\chapter{Preface}

This chapter serves to motivate the following thesis.

In most matters of life, it is of principal concern to represent and transform
data. In high profile applications, such as where human fortunes are at stake,
it is important that data transformations occur in a reliable manner.  This
means both that the data is transformed in an expected way, but also that this
is done in a timely manner.

The advent of digital computers has delivered much reliability, provided that
their human programmers can deliver correct and efficient programs for the
matter at hand. Much has been done in the field of data representation and
transformation, i.e. Computer Science, in terms of guaranteeing the correctness
and judging the efficiency of programs. This thesis is about guaranteeing
both the correctness and efficiency of programs.

In support of the fundamental human faculty of managing data, it is natural in
Computer Science to consider data representations and transformations as data
in themselves. Accompanied by additional descriptors, this leads way for basic
buildings blocks for the more elaborate management of data. Such
``abstractions'' in Computer Science both alleviate the mundane details of data
representation and transformation, and enable more efficient communication
about programs amongst the humans themselves.

\begin{example} \ 

It may be sufficient, for a Alice to know that Bob has written a program
\textsc{Add}, which given two integers, $x$ and $y$, in the range $[0;31]$,
after a handful machine instructions, returns an integer, $z$, such that $z=x+y
\mod 32$.

Alice could now use \textsc{Add} to compute the sum of an array of integers in
the range $[0;31]$, in a number of machine instructions proportional to the
number of elements in the array. That is Alice, didn't need to know the
internal details of how \textsc{Add} works, merely some high-level descriptors
of what it does.

\end{example}

We say that the program \textsc{Add} computes the function $f(x,y)=x+y \mod
32$.

% static vs dynamic

Abstractions incorporate into the interface between the human and computer.

% The subject matter of this thesis is to guarantee that Bob has not used more
% than a handful machine instructions.

% A good programming language puts the tools of the language designer at the
% programmers fingertips. And so anything written in a programming language
% defines a programming language.

% We approach the problem from a programming languages perspective. A digital
% computer typically understands only one language --- machine language. As
% programming is principally a human activity, 

% Programming languages serve to communicate the intent of the programmer both
% to a computer, other programmers, and the programmer themselves.

\begin{itemize}

\item Programming Languages

\item Human-computer interaction

\item Computability and Complexity

\end{itemize}

\section{Audience} \label{sec:preface:audience}

The audience of this thesis is anyone interested in the connection of
computability and complexity to the theory and practice of programming
languages.

In particular, what the admittance of useful programming language constructs
implies for the time and space complexity of the programs that you can write. A
programming language construct is ``useful'' if its admission permits to write
a practical class of programs in an elegant manner.

\def\ch{\textsection}
\def\chs{\ch\ch}

The thesis is directed towards the level of a Computer Science graduate student
at the time of writing: The reader is assumed to be familiar with the basics of
discrete mathematics, as in \cite{sipser-2013}, \ch0, and
\cite{cormen-et-al-2009}, Appendices A, B, and C. The analysis of time and
space complexity of algorithms, as in \cite{cormen-et-al-2009}, \chs1--17 and
\chs21--24. Regular and context-free languages, as in \cite{sipser-2013}
\chs1--2, and their use for programming language design, as in
\cite{mogensen-2010}. The reader should also be familiar with Logic in Computer
Science, as in \cite{huth-ryan-2004}, \chs1--4.

\section{Internal references} \label{sec:preface:internal-references}

All references within this document to parts of this document are qualified
with the page on which the referenced part begins. For instance, we refer to
this section as \refSection{preface:internal-references}.

\section{Named segments}

Named segments are text segments with a role and an address. The role seeks to
distinguish the text segment from prose, and the address permits to
subsequently reference the segment.

Addresses are of the format \mbox{$x$.$y$.$z$}, where $z$ is the index of the
named segment in section indexed $y$, in chapter indexed $x$. Where
appropriate, we'll give a named segment a name for even easier reference, for
instance, when defining the concept of a Turing machine. The various roles are
described below, showcasing also how each named segment will be formatted.

\begin{notion} An informal definition of a concept, appealing either to the
reader's intuition or formerly stated notions. \end{notion}

\begin{hypothesis} An informal statement, proposed as an explanation for a
particular phenomenon. A hypothesis does not admit itself to formal proof, but
can be disproven by observation. A hypothesis is always followed by a
discussion of its validity. \end{hypothesis}

Some notions are ``primitive'', in the sense that they underpin subsequent
formal definitions. It is a fundamental epistemological impasse, that in any
formal system, some concepts must remain distinguished, but undefined, these
are the \textbf{primitive notions}, or \textbf{axioms}, of the formal system.

\begin{definition} A formal definition of a concept, appealing to either
formerly stated primitive notions or definitions. Definitions seek to underpin
subsequent formal definitions and analyses. \end{definition}

This thesis only seeks to be self-contained beyond what can be expected of the
intended audience. Some notions, hypotheses, and definitions appeal to
knowledge outside this thesis. See also \refSection{preface:audience}.

\begin{theorem} \label{thm:theorem} A formal statement using formerly
defined concepts. A theorem is always proven to hold by an immediately
proceeding proof.\end{theorem}

\begin{proof} A formal or informal proof the preceding theorem. \end{proof}

As proofs occur immediately after a theorem, proofs are not decorated with an
address. A proof is then referred to by the address of the theorem it proves.
For instance, we refer to the proof above as \refProof{theorem}.

\begin{example} An example of a concept discussed above. Examples seek to
facilitate the reader's understanding on the concept in question.
\end{example}

\begin{notation}

A choice of notation for some formally or informally defined concepts.
Typically, notation is introduced to aid further discussion.

\end{notation}

A concept is put in quotations, for instance ``computable'', if it is
introduced informally as part of a notion. A concept is put in bold, for
instance \textbf{computable}, if it is introduced formally as part of a
definition.

\section{Preliminaries}

\begin{notion}

Let the \textbf{natural numbers}, $\mathbb{N}$, be the nonnegative integers,
i.e.  $\mathbb{N}=\set{0,1,2,\ldots}$.

\end{notion}

\begin{definition}

A \textbf{language} $\mathcal{L}=\chev{\mathcal{C},\mathcal{F}, \mathcal{R}}$
is given by specifying a set of constant symbols $\mathcal{C}$, a set of
function symbols $\mathcal{F}$, and a set of relation symbols $\mathcal{R}$.

\end{definition}

% a sequence is something formed by the concatenation of elements.

\begin{definition}

An \textbf{alphabet} is a non-empty finite set.

\end{definition}

\begin{definition}

A \textbf{string} is a finite sequence of elements drawn from an alphabet.

\end{definition}

\begin{definition}

A \textbf{string language} is a set of \textbf{strings}.

\end{definition}

..

\begin{definition}

A \textbf{closure} of $A$ under the operations $O$, is the smallest class $C$,
containing $A$, and such that the operations of $O$, operating on elements of
$C$ yield elements of $C$.

\end{definition}

\begin{definition}

A \textbf{function algebra} is a closure of a class $A$.

\end{definition}

\begin{definition}

A characteristic function 

\end{definition}
