% vim: set spell:

\section{Preliminaries}

\begin{definition} Let $\mathbb{N}$ denote the type of natural numbers,
starting with $0$.\end{definition}

\begin{definition} Given a type $\Sigma$, let $\Sigma^*$ be the type of
\textbf{symbolic strings} over $\Sigma$, formed using either the \textbf{empty
string} or the \textbf{string concatenation} operator:
\begin{align*}\label{def:empty-string}
\judgement{}{\varepsilon_\Sigma:\Sigma^*}
\quad
\judgement{s_0:\Sigma \quad s:\Sigma^*}{s_0 \cdot s:\Sigma^*}
\end{align*}
We refer to $\Sigma$ as an \textbf{alphabet}, and to its
terms as \textbf{symbols}.
\end{definition}

We omit $\Sigma$ in $\varepsilon_\Sigma$, when it is clear from context, e.g.
$\varepsilon:\Sigma^*$.

\begin{definition} We say that a term $s\equiv s_0 \cdot s_1 \cdots s_{n-1}
\cdot \varepsilon : \Sigma^*$  is a string of length $n:\mathbb{N}$ over the
alphabet $\Sigma$.\end{definition}

\begin{definition} Let $\card{\Sigma}$ be the number of symbols in alphabet
$\Sigma$, called it's \textbf{cardinality}. \end{definition}

If the cardinality of a alphabet is finite, strings over the alphabet are
\textbf{recursively enumerable}, i.e. there exists a bijection
$f:\Sigma^*\rightarrow\mathbb{N}$.

\begin{theorem} If $\card{\Sigma} = n$ for some $n\in\mathbb{N}$, then
$\card{\Sigma^*}=\card{\mathbb{N}}$.\end{theorem}

\begin{proof} We have $\card{\Sigma} = n$. The terms of $\Sigma$ can be
arranged in a sequence such that for each $s:\Sigma$ we assign a unique natural
number $g(s)$, such that $1 \leq g(s) \leq \card{\Sigma}$. We now inductively
define the function $f:\Sigma^*\rightarrow \mathbb{N}$.

\begin{align}
f\p{\varepsilon} &= 0 \\
f\p{s_0\cdot s} &= g\p{s_0} + f\p{s} \cdot \card{\Sigma}
\end{align}

To show that the function is a bijection, we also define its inverse by
induction:

\begin{align}
f^{-1}\p{0} &= \varepsilon \\
f^{-1}\p{n} &= g^{-1}\p{n \mod \card{\Sigma}} \cdot f^{-1}\p{\floor{n/\card{\Sigma}}} & \text{for $n>0$}
\end{align}

\end{proof}

This means that strings over any finite alphabet can be used to represent
strings over any other finite alphabet. One such basic alphabet that has proven
useful in practice is the binary alphabet, consisting of e.g. the symbols
\texttt{0} and \texttt{1}.

\begin{definition} Let $\mathbb{B}$ denote the type of booleans.\end{definition}

Define the set $\mathbb{B}$, and the usual boolean connectives.

\begin{definition} An infix function $\leq : A\times A \rightarrow \mathbb{B}$ defines a
\textbf{total order} on the type $A$ iff for all $x,y,z : A$:
\begin{align}
x \leq y \wedge y \leq x \Rightarrow y = x  & \quad \text{(antisymmetry)} \\
x \leq y \wedge y \leq z \Rightarrow x \leq z & \quad \text{(transitivity)} \\
x \leq y \vee y \leq x & \quad \text{(totality)}
\end{align}
\end{definition}

\begin{definition} Given a total order on $\Sigma$, we define the
\textbf{lexicographic order} on $\Sigma^*$ as follows:
\begin{align}
\judgement{}{\varepsilon \leq s_0\cdot s}
\quad
\judgement{s_0 = t_0 \quad s \leq t}{s_0\cdot s \leq t_0 \cdot t}
\quad
\judgement{s_0 \neq t_0 \quad s_0 \leq t_0}{s_0 \cdot s \leq t_0 \cdot t}
\end{align}
\end{definition}

\begin{theorem} A lexicographic order is a total order. \end{theorem}

\begin{proof} \end{proof}

\begin{definition} An infix function $<:A\times A \rightarrow \mathbb{B}$
defines a \textbf{strict total order} on the type $A$, iff 
\begin{align}
\judgement[F-Less]{\neg \p{y \leq x}}{x < y}
\end{align}
\end{definition} 

We say that $y$ has a higher \textbf{value} than $x$, whenever $x<y$, and equal
in value whenever $x=y$.

\begin{definition} A function $f$ is defined by \textbf{primitive recursion}
from the functions $h,g_1,g_2,\ldots,g_n$ for $n:\mathbb{N}$, iff
\begin{align}
f\p{\varepsilon, \vect{y}} &= h\p{\vect{y}} \\
f\p{s_i\p{x}, \vect{y}} &= g_i\p{x,\vect{y},f\p{x,\vect{y}}} \\
x &< s_i\p{x}
\end{align}
\end{definition}

That is, a function is defined by primitive recursion, if on every invocation
of the function, we recurse on at most one formal parameter, and only recurse
after the actual parameter has been decreased in value. It follows that
primitive recursion demands a strict total order on the value type in question.

\begin{definition} A function is \textbf{primitive recursive} if it is
non-recursive, or defined by primitive recursion from non-recursive, or
primitive recursive functions.  \end{definition}

Primitive recursive functions are not necessarily polytime functions.

\begin{example}
Unary addition over binary notation is primitive recursive, but not polytime.

Let $\leq$ be the lexicographic order on binary notation.

We now define unary addition over binary notation using primitive recursion:

\begin{align}
add'\p{0,0,0} = \p{0,0} \\
add'\p{1,0,0} = \p{1,0} \\
add'\p{0,1,0} = \p{1,0} \\
add'\p{0,0,1} = \p{1,0} \\
add'\p{0,0,1} = \p{1,0} \\
add'\p{1,1,0} = \p{0,1} \\
add'\p{1,1,1} = \p{1,1}
\end{align}

\end{example}
