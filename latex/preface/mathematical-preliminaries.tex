\section{Mathematical Preliminaries}

\label{sec:preface:mathematical-preliminaries}

It is beyond the scope of this thesis to develop a notion of sets. We take this
notion as given, subject to the following specifications.

\begin{specification} A \textbf{set} is a mathematical object that is distinct
from, but completely determined by its \textbf{elements}. \end{specification}

\begin{notation} When we consider an element $a$ of a set $A$, we write $a\in
A$. \end{notation}

\begin{definition} The \textbf{union} of sets $A$ and $B$, written $A\cup B$,
is the set of all $a\in A$ and all $b\in B$. \end{definition}

\begin{specification} A \textbf{property} is a mathematical statement that is
either true or false. \end{specification}

\begin{notion}

Let the \textbf{natural numbers}, $\mathbb{N}$, be the nonnegative integers,
i.e.  $\mathbb{N}=\set{0,1,2,\ldots}$.

\end{notion}

\begin{definition}

A \textbf{language} $\mathcal{L}=\chev{\mathcal{C},\mathcal{F}, \mathcal{R}}$
is given by specifying a set of constant symbols $\mathcal{C}$, a set of
function symbols $\mathcal{F}$, and a set of relation symbols $\mathcal{R}$.

\end{definition}

% a sequence is something formed by the concatenation of elements.

\begin{definition}

An \textbf{alphabet} is a non-empty finite set.

\end{definition}

\begin{definition}

A \textbf{string} is a finite sequence of elements drawn from an alphabet.

\end{definition}

\begin{definition}

A \textbf{string language} is a set of \textbf{strings}.

\end{definition}

..

\begin{definition}

A \textbf{closure} of $A$ under the operations $O$, is the smallest class $C$,
containing $A$, and such that the operations of $O$, operating on elements of
$C$ yield elements of $C$.

\end{definition}

\begin{definition}

A \textbf{function algebra} is a closure of a class $A$.

\end{definition}

\begin{definition}

A characteristic function 

\end{definition}

\begin{notation} We denote a tuple by listing the elements separated by commas,
and enclosed in parentheses. We use this notation out of inspiration from
Haskell syntax. We can regard this as in line with regular mathematical
notation as the arguments to a non-curried function are typically passed in
parentheses. That is unless a more convenient notation is given for particular
kinds of tuples, e.g. the configurations of a Turing machine.\end{notation}

\begin{definition} An \textbf{alphabet} is a finite set of \textbf{symbols}.
\end{definition}

\begin{definition} A \textbf{string} over an alphabet is a finite sequence of
symbols from that alphabet. \end{definition}


