\chapter{Implicit Computational Complexity} \label{sec:icc}

In \refSec{computational-complexity}, we characterised a range of complexity
classes in terms of the costs incurred on a particular machine model. This
reflects perhaps a real-world motivation for the theory of computational
complexity --- some interesting programs actually get executed by machines.
Such characterisations have proven theoretically well-founded as well, as
certain complexity classes appear nearly equivalent, regardless of the choice
of machine model, e.g. PTIME.

The field of \emph{Implicit Computational Complexity} (ICC) seeks instead to
give ``machine-independent'' characterisations of these complexity classes.
The motivation for such characterisations is twofold\cite{baillot-et-al-2006}:

\begin{enumerate}

\item to provide more ``natural'' (also called ``foundational'' and ``resource
free'' in \cite{bellantoni-phd-1992}) characterisations of complexity classes,
and

% At its outset, researchers in computational complexity were concerned with
% finding ``machine-independent'' measures of functional
% complexity\cite{blum-1967}. At the time, it was unclear whether the choice of
% machine model could wither away results in computational complexity, that is,
% until e.g. the Hartmanis-Baker conjecture
% (\refCjt{hartmanis-baker})\cite{young-1988}.
%
% It's more that the results seek to hold regardless of choice of machine
% model.

\item to design systems suitable for static verification of program complexity.
 
% it can serve as a foundation for the design of programming languages where
% all programs are guaranteed by construction to lie within the characterised
% complexity class; this is in contrast to the common practice of working with
% algorithms, where the complexity of an algorithm is analyzed after it is
% written.

\end{enumerate}

There is a deep-rooted interest in ``machine-independence'' in Computational
Complexity. The general independence of certain important complexity classes
from the choice of machine model, e.g. PTIME as by the Hartmanis-Baker
conjecture, has helped to establish Computational Complexity as a field. This
strive for ``machine-independence'' is illustrated by several seminal works in
Computational Complexity, e.g. \cite{cobham-1965, blum-1967}.

% The other motivation is that such characterisations are perhaps more natural
% and more general than those based on a particular machine model.

% considering language restrictions or logical/computational principles entailing
% complexity properties

The second motivation is perhaps more recent. In the context reliable computer
systems, it is sometimes important to certify not just that a computation has
the intended input/output effect, but also that it does not exceed the
allocated time or space resources. Implicit computational complexity works in
this direction by seeking to characterise the complexity classes in a manner
general enough to ``capture'' commonly used algorithms.

We can say that in classical, i.e. explicit, computational complexity, we look
for the boundaries where the increase in time or space available for a
computation leads to a considerably larger class of problems that we can solve.
In implicit computational complexity, we instead look for the boundaries where
the change in program syntax or semantics necessarily leads to the same.

In \refSec{computability}, where we introduced the notion of a Turing machine,
we also gave a ``machine-independent'' characterisation of what it is that a
Turing machine computes. This was a ``mathematical'' characterisation. With
such a ``machine-avoidant'' foundation as classical mathematics, such
characterisations are of little use, but it gave a motivating example. ICC is
instead founded in either:

\begin{itemize}

\item recursion theory\cite{cobham-1965, clote-1990,
bellantoni-cook-1992, leivant-1995, ishihara-1999, kristiansen-2005,
marion-2009},

\item logic\cite{fagin-1974,immerman},

\item finite model theory\cite{?},

\item proof theory\cite{girard-scedorov-scott-1992, leivant-1994, girard-1998,
asperti-roversi-2002, lafont-2004, dal-lago-hofmann-2010},

\item term rewriting
systems\cite{beckmann-weiermann-1996, bonfante-et-al-1999, avanzini-et-al-2012,
de-carvalho-simonsen-2014},

\item programming languages\cite{voda-1994, jones-1999,
kristiansen-niggl-2004, kristiansen-voda-2005, niggl-wunderlich-2006},

\item type systems\cite{hofmann-2000b, crary-weirich-2000,
bellantoni-et-al-2000, hofmann-2003}

\hspace{-0.12in}\vdots

\end{itemize}

This diversity in foundations suggests that it is worthwhile to review and
evaluate  at least some of these characterisations. The wealth of citations
suggests that it is beyond the scope of this thesis to cover all of them.
Instead, we will be guided by the surveys of \cite{clote-1999},
\cite{hofmann-2000a}, and \cite{dal-lago-2012}, drawing on insights from
\cite{jones-1999}, \cite{jones-2001}, \cite{niggl-2005},
\cite{niggl-wunderlich-2006}, and \cite{jones-kristiansen-2009}.

In general, we will evaluate a characterisation of a complexity class in terms
of two formal, and one informal parameter.
 
\begin{definition} \label{def:icc-soundness} A characterisation $S$, of a
complexity class $C$, is \textbf{extensionally sound} if all functions
computable in $S$ lie in the complexity class $C$.\end{definition} 

For instance, we could prove that a characterisation $S$, of a complexity class
$C$, is extensionally sound, by exhibiting a method of simulating a Turing
machine with resources bounded to $C$ in $S$.

\begin{definition} \label{def:icc-completeness} A characterisation $S$, of a
complexity class $C$, is \textbf{extensionally complete} if all functions in
the complexity class $C$ are computable in $S$.\end{definition}

For instance, we could prove that a characterisation $S$, of a complexity class
$C$, is extensionally complete, by exhibiting a method of simulating $S$ on a
Turing machine with resources bounded to $C$.

\begin{definition} \label{def:icc-capture} A characterisation $S$, of a
complexity class $C$, \textbf{captures} the complexity class $C$, if $S$ is
both extensionally sound and complete.  \end{definition}

\refDef{icc-soundness} and \refDef{icc-completeness} appeal to the computation
of (mathematical) functions rather than the expression of algorithms. Of
course, there are (uncountably) infinitely many algorithms computing the same
function. The vast majority of them may be rejected by a system $S$ in virtue
of its characterisation of $C$. Perhaps so much so, that the eloquent
programmer may find $S$ to be rather ineloquent.

% It is commonplace to seek both \textbf{efficient} and
% \textbf{elegant} algorithms. Let an \textbf{eloquent} algorithm be one which is
% both elegant and efficient, at least from the point of view of some community
% of programmers.

This leads way for the last, informal parameter which we'll use for comparing
ICC characterisations:

% Programming paradigms

\begin{notion} A system $S$ is \textbf{intensionally expressive} if it permits
eloquent solutions to some set of practical programming problems.\end{notion}

% ``Eloquence'' is both a matter aesthetics, and a matter of raw efficiency. It may be 

A program that is easy to understand often fails to be efficient, and an
efficient program often fails to be easy to understand.

\begin{notion} A system $S$ is \textbf{relatively expressive} if there is no
system $T$ which captures the same complexity class, but permits the expression
of more efficient algorithms. (Due to \cite{jones-2001}.) \end{notion}

It is not uncommon to seek a baseline characterisation of a complexity class
first, and then later to seek to increase its ``intensionality'', which in
itself may lead to new characterisations. This is indeed the motivation in e.g.
\cite{hofmann-1999, niggl-2005, marion-pechoux-2008, jones-kristiansen-2009}.

% not really aka, but in the spirit of the previous chapter, Implicit
% Complexity, and hence Implicit Complexity Theory

% Also, better suited as an introduction as it doesn't really say anything
% about the choice of polynomial time from the get go. That's more a
% coincidence, the following definitions will hold regardless. Perhaps it is
% worth it calling this a Part I: Introduction and Background

% Fragment of a logic means that we are restricted to a subset of the syntax,
% but retain the same semantics. If we choose the combination of words -
% subsystem of system, we are perhaps a little more free, but in either case,
% we probably have to further specify what we keep and what we take out.

% for the purposes of this thesis, we may regard logical systems equivalent to
% programming languages - draw inspiration from Types of Crash Prevention - it
% has a nice, human readable introduction to the whole thing.

% A system that is both intensionally sound and complete is often not
% recursively enumerable think, EQ(TM) which is neither Turing-recognisable,
% nor co-Turing-recognisable.

% Thus in practice, we strive to improve the intensional expressivitiy by
% capturing important classes of examples and patterns.
