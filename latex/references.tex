\begin{thebibliography}{9} % 9 if < 10 references, 99 if < 100 references, etc.

\label{sec:references}

\section*{Published in Journals}

% \bibitem[Grzegorczyk (1953)]{grzegorczyk-1953}

% Andrzej Grzegorczyk. Institut Matematiczny Polskiej Akademii Nauk. \emph{Some
% classes of recursive functions}. 1953. Rozprawy Matematyczne: IV. Edited by
% Karol Borsuk, et al. Published by Polskie Towarzystwo Matematyczne Warszawa
% in Poland. Printed in Poland, by Wroclawska Drukarnia Naukowa.

% \backrefprint

\bib{Di Paola \& Heller (1987)}{di-paola-heller-1987}{%
%
Robert A. Di Paola and Alex Heller. \emph{Dominical Categories: Recursion
Theory without Elements}. September 1987. The Journal of Symbolic Logic, Vol.
52, No. 3, pp.  594--635. Published by the Association of Symbolic Logic.
%
}

\bib{Simmons (1988)}{simmons-1988}{%
%
Harold Simmons. University of Aberdeen. \emph{The Realm of Primitive
Recursion}. 1988. Archive for Mathematical Logic, Vol. 27, No. 2, pp. 177--188.
Published by Springer-Verlag.
%
}

\bib{Bellantoni \& Cook (1992)}{bellantoni-cook-1992}{%
%
Stephen Bellantoni and Stephen Cook. \emph{A new recursion-theoretic
characterizaton of the polytime functions}. June 1992. Computational
Complexity. Vol. 2, No. 2, pp. 97--110. Published by Birkh\"auser-Verlag Basel.
%
}

\bib{Girard, Scedorov \& Scott (1992)}{girard-scedorov-scott-1992}{%
%
Jean-Yves Girard, Andre Scedorov, Philip J. Scott. \emph{Bounded Linear Logic:
A Modular Approach to Polynomial Time Computability}. April 1992. Theoretical
Computer Science, Vol. 97, No. 1, pp. 1--66. Published by Elsevier Science
Publishers B.V.
%
}

\bib{Leivant (1995)}{leivant-1995}{%
%
Daniel Levant (Indiana University). \emph{Ramified Recurrence and Computational
Complexity I: Word Recurrence and Poly-time}. January 1995.  In Peter Clote and
Jeffrey B. Remmel (eds.), Feasible Mathematics II, Progress in Computer Science
and Applied Logic, Vol. 13, pp.  320--343. Published by Birkh\"auser Boston.
ISBN: 978-1-4612-7582-4. \\
%
\emph{A generalization of \cite{leivant-1990} and \cite{bellantoni-cook-1992}
from two-tiered to many-tiered recursion. This is the complete version of the
earlier conference paper on ``ramified'', aka. ``stratified'', recursion
\cite{leivant-1993}.}
%
}

\bib{Girard (1998)}{girard-1998}{%
%
Jean-Yves Girard. Institut de Math\'ematisques de Luminy. \emph{Light Linear
Logic}. June 1998. Information and Computation, Vol. 143, No. 2, pp. 175--204.
Published by Academic Press.
%
}

\bib{Beckmann \& Weiermann (1996)}{beckmann-weiermann-1996}{%
%
Arnold Beckmann and Andreas Weiermann (Westf\"alischen Wilhelms-Universit\"at).
\emph{A term rewriting characterization of the polytime functions and related
complexity classes}. December 1996. Archive for Mathematical Logic, Vol. 36,
No. 1, pp.  11--30. 
%
}

\bib{Jones (1999)}{jones-1999}{%
%
Neil D. Jones (University of Copenhagen). \emph{LOGSPACE and PTIME
characterized by programming languages}. October 1999. Theoretical Computer
Science, Vol. 228, No. 1--2, pp. 151--174. Published by Elsevier Science B.V. 
%
}

\bib{Hofmann (2000)}{hofmann-2000}{%
%
Martin Hofmann. University of Edinburgh. \emph{Programming languages capturing
complexity classes}.  2000.  ACM SIGACT News, Vol. 31, No. 1, pp. 31--42.
Published by the ACM in New York, USA.
%
\emph{A concise survey of a range approaches to implcit complexity theory
between 1965 and 2000.}
%
}

\bib{Asperti \& Roversi (2002)}{asperti-roversi-2002}{%
%
Andrea Asperti and Luca Roversi. Universit\'a di Bologna and Universit\'a di
Torino. \emph{Intuitionistic Light Affine Logic}. January 2002. ACM
Transactions on Computational Logic, Vol. 3, No. 1, pp. 137--175. Published by
ACM New York.
%
}

\bib{Lafont (2004)}{lafont-2004}{%
%
Yves Lafont (Universite de le Mediterranee). \emph{Soft linear logic and
polynomial time}. June 2004. Theoretical Computer Science: Implicit
Computational Complexity, Vol. 318, No. 1--2, pp. 163--180. Published by
Elsevier B.V.
%
}

\bib{Marion (2009)}{marion-2009}{%
%
Yves Marion (Nancy Universit\'e). \emph{On tiered small jump operators}. March
2009. Logical Methods in Computer Science, Vol. 5, No. 1, pp. 1--19. Published
under the Creative Commons Attibution-NonDerivs License.
%
}

\bib{dal Lago \& Hofmann (2010)}{dal-lago-hofmann-2010}{%
%
Ugo dal Lago and Martin Hofmann. Universet\'a di Bologna and LMS M\"unchen.
\emph{Bounded Linear Logic, Revisited}.  December 2010. Logic Methods in
Computer Science, Vol. 6, No. 4, pp. 1--31. Published under a Creative Commons
license.
%
}

\bib{Avanzini et al. (2012)}{avanzini-et-al-2012}{%
%
Martin Avanzini, Naohi Eguchi, and Georg Moser (University of Innsbruck and
Tohoku University). \emph{A New Order-Theoretic Characterisation of the
Polytime Computable Functions}. 2012. Programming Languages and Systems,
Lecture Notes in Computer Science, Vol. 7705, pp. 280--295.
%
}

\bib{dal Lago (2012)}{dal-lago-2012}{%
%
Ugo dal Lago. Universet\'a di Bologna. \emph{A Short Introduction to Implicit
Computational Complexity}. ESSLLI 2010/2011 Lectures, Lecture Notes in Computer
Science, Vol. 7388, pp. 89--109. Published by Springer-Verlag Berlin
Heidelberg.
%
}

\bib{de Carvalho \& Simonsen (2014)}{de-carvalho-simonsen-2014}{%
%
Daniel de Carvalho and Jakob Grue Simonsen (University of Copenhagen). \emph{An
Implicit Characterization of the Polynomial-Time Decidable Sets by Cons-Free
Rewriting}. 2014.  Rewriting and Typed Lambda Calculi, Lecture Notes in
Computer Science, Vol. 8560, pp. 179--193.
%
}

\section*{Books}

\bibitem[Beman (1901)]{beman-1901}

Richard Dedekind. \emph{Theory of Numbers: The Nature and Meaning of Numbers}.
Authorized translation by Wooster Woodruff Beman, Professor of Mathemtics at
the University of Michigan, USA. Published in Chicago, by The Open Court
Publishing Company.

Digitized by the Internet Archive in 2007 with funding from Microsoft
Corporation. Retrieved from
\url{http://www.archive.org/details/essaysintheoryofOOdedeuoft} on July 7,
2014.

\backrefprint

\bib{Wittgenstein (1953)}{wittgenstein-1953}{%
%
Ludwig Wittgenstein. \emph{Philosophical Investigations}. The German text, with
a revised English translation.  Translated by G.E.M. Ancombe. Third Edition.
Published by Blackwell Publishing in 1953, 1958, and 2001. ISBN:
978-0-631-23159-2.
%
}

\bib{Markov (1954)}{markov-1954}{%
%
Andrey Andreyevich Markov Jr. Academy of Sciences of the U.S.S.R. \emph{Theory
of Algorithms}. Works of the Mathematical Institute Im. V.A. Steklov. Vol.
XLII. Published by Izdatel'stvo Akademii Nauk SSSR in 1954. Translated by
Jacques J. Schorr-Kon and PST Staff. Second Impression. Published by the Israel
Program of Scientific Translations Ltd. in 1962. Printed in Jerusalem.
%
}

\bib{Tourlakis (1984)}{tourlakis-1984}{%
%
George J. Tourlakis. \emph{Computability}. Published by Reston Publishin
Company, Inc., A Prentice-Hall Company, in 1984. Printed in the United States.
ISBN: 0-8359-0876-3.
%
}

\bib{Jones (1997)}{jones-1997}{%
%
Neil. D. Jones. \emph{Computability and Complexity: From a Programming
Perspective}. Published by The MIT Press in 1997.  Printed in the United
States. ISBN: 0-262-10064-9.
%
}

% \bibitem[Graham et al. (1998)]{graham-et-al-1998}

% Ronald Lewis Graham, Donald Ervin Knuth, Oren Patashnik. \emph{Concrete
% Mathematics}. Second Edition. 1998. Published by Addison-Wesley Publishing
% Company, Inc. ISBN: 978-0-201-55802-9. 24th printing. 2011. Printed in
% Westford, Massachusetts, USA.

% \backrefprint

\bib{Huth \& Ryan (2004)}{huth-ryan-2004}{%
%
Michael Huth and Mark Ryan. Imperial College London and University of
Birmingham, United Kingdom. \emph{Logic in Computer Science: Modelling and
Reasoning about Systems}, Second Edition. Published by Cambridge University
Press in New York, USA in 2004. ISBN: 978-0-521-54310-1. 7th printing.  2011.
Printed in the United Kingdom by University Press, Cambridge.
%
}

\bib{Cormen et al. (2009)}{cormen-et-al-2009}{%
%
Thomas H. Cormen, Charles E. Leiserson, Ronald L. Rivest, and Clifford Stein.
Massachusetts Institute of Technology.  \emph{Introduction to Algorithms}, 3rd
ed. Published by The MIT Press in 2009. Printed in the United States. ISBN:
978-0-262-03384-8.
%
}

\bib{Mogensen (2010)}{mogensen-2010}{%
%
Torben Ægidius Mogensen. Department of Computer Science, University of
Copenhagen. \emph{Basics of Compiler Design}. Anniversary Edition. 2010.
Published through \texttt{lulu.com}. ISBN: 978-87-993154-0-6.
%
}

\bib{Rose (1984)}{rose-1984}{%
%
H. E. Rose, School of Mathematics, University of Bristol. \emph{Subrecursion:
functions and hierarchies}. 1984. Oxford Logix Guides: 9. Typeset by Joshua
Associates, Oxford.  Published by Clarendon Press, division of Oxford
University Press, in New York, USA. ISBN 0-19-853189. Printed in Great Britain,
by The Thetford Press Ltd.
%
}

\bib{Homer \& Selman (2011)}{homer-selman-2011}{%
%
Steven Homer and Alan L. Selman. Boston University and University at Buffalo.
\emph{Computability and Complexity Theory}. Second Edition. Published by
Springer in 2011. ISBN: 978-1-4614-0681-5.
%
}

\bib{Sipser (2013)}{sipser-2013}{
%
Michael Sipser. \emph{Introduction to the Theory of Computation}. Third
Edition. Published by Cengage Learning in 2013. Printed in the United States.
ISBN: 978-1-133-18779-0.
%
}

\bib{Dickson (1919)}{dickson-1919-goldbach}{%
%
Leonard Eugene Dickson. Professor of Mathematics in the University of Chicago.
\emph{Goldbach's Empirical Theorem: Every Even Integer is a Sum of Two Primes}.
In History of the Theory of Numbers, Volume 1: Divisibility and Primality, pp.
421--424.  Published by Carnegie Institute of Washington in 1919, 1920, and
1923.  Reprinted in New York in 1992. ISBN: 0-8218-1934-8.
%
}

\section*{In Proceedings}

\bib{Turing (1936-7)}{turing-1936-7}{%
%
Alan M. Turing. Princeton University. \emph{On computable numbers with an
application to the Entscheidungsproblem.} 1936-7. In Proceedings of The London
Mathematical Society, Second Series, Vol. 42. pp. 230–265. Originally printed
in Great Britain. Reprinted in Belgium in 1968 for The London Mathematical
Society.
%
}

\bib{Gifford \& Lucassen (1986)}{gifford-lucassen-1986}{%
%
David K. Gifford and John M. Lucassen. MIT Laboratory for Computer Science.
\emph{Integrating Functional and Imperative Programming.} August 1986. In
Proceedings of the 1986 ACM conference on LISP and functional programming.
Published by ACM New York. ISBN: 0-89791-200-4.
%
}

\bib{Leivant (1990)}{leivant-1990}{%
%
Daniel Leivant. \emph{Subrecursion and lambda representation over free algebras
(Preliminary summary)}. 1990. In Samuel L. Buss and Philip J. Scott (Eds.),
Feasible Mathematics, Progress in Computer Science and Applied Logic, Vol. 9,
pp. 281--291.  Proceedings of a Mathematical Sciences Institute Workshop,
Ithaca, New York, USA, 1989.  Published by Birkh\"auser Boston. ISBN:
978-1-4612-3466-1.
%
}

\bib{Leivant (1993)}{leivant-1993}{%
%
Daniel Leivant. Indiane University. \emph{Stratified functional programs and
computational complexity}. March 1993. In Proceedings of the 20th ACM
SIGPLAN-SIGACT symposium on Principles of Programming Languages (POPL'93), pp.
325--333.  Published by ACM New York. ISBN: 0-89791-560-7.
%
}

\bib{Bonfante et al. (1999)}{bonfante-et-al-1999}{%
%
G. Bonfante, A. Cichon, J. Y Marion, H. Touzet. \emph{Complexity Classes and
Rewrite Systems with Polynomial Interpretation}. 1999. In Proceedings of the
1998 Annual Conference of the European Association for Computer Science Logic
(EACSL), held in Brno, Czech Republic. Computer Science Logic, Lecture Notes in
Computer Science, Vol. 1584. Published by SpringerISBN 978-3-540-48855-2.
%
}

\bib{Cobham (1965)}{cobham-1965}{%
%
Alan Cobham. IBM Research Center, New York. \emph{The intrinsic computational
difficulty of functions}. 1965. In Proceedings of the 1964 International
Congress for Logic, Methodology and Philosophy of Science, pp. 24--30. Edited
by Yehoshua Bar-Hillel. Published by North-Holland Publishing Company in
Amsterdam, Holland.  Printed in Israel, by Jerusalem Academic Press Ltd.
%
}

% \bibitem[Sazonov (1987)]{sazonov-1987}

% V.Yu. Sazonov. Institute of Mathematics, Novosibirsk, USSR. \emph{Bounded set
% theory and polynomial computability}. 1987. In Proceedings of the 1987
% International Conference on Fundamentals of Computation Theory. Lecture Notes
% in Computer Science, Vol. 278, pp. 391--395. Editied by L. Budach, R.G.
% Bukharajev, and O.B. Lupanov. Published by Springer-Verlag in Berlin,
% Germany.  Printed by Druckhaus Beltz.

% \backrefprint

\section*{Other}

\bibitem[Joyce (2005)]{joyce-2005}

David E. Joyce. Clark University. USA. \emph{Notes on Richard Dedekind's Was
sind und was sollen de Zallen?} 2005. Retrieved from
\url{http://aleph0.clarku.edu/~djoyce/numbers/dedekind.pdf} on July 6, 2014.

Archived by WebCite\textsuperscript{\textregistered}\ at
\url{http://www.webcitation.org/6U9PH93PS}.

\backrefprint

% \bibitem[Bennett (1962)]{bennett-1962}

% J.H. Bennett. Ph.D.Thesis, Princeton University. \emph{On Spectra}. 1962.
% Princeton, New Jersey, USA.

% \emph{Referenced by \cite{cobham-1965}, not attained.}

% \backrefprint

\bib{Roversi (2007)}{roversi-2007}{%
%
Luca Roversi. Universit\'a di Torino. \emph{Weak Affine Light Typing: Polytime
intensional expressivity, soundness and completeness}. 2007. Computing Research
Repository. Revised as of December 2013. arXiv:0712.4222v2 [cs.LO].
%
}

\end{thebibliography}
