\begin{thebibliography}{9} % 9 if < 10 references, 99 if < 100 references, etc.

\section*{Published in Journals}

% \bibitem[Grzegorczyk (1953)]{grzegorczyk-1953}

% Andrzej Grzegorczyk. Institut Matematiczny Polskiej Akademii Nauk. \emph{Some
% classes of recursive functions}. 1953. Rozprawy Matematyczne: IV. Edited by
% Karol Borsuk, et al. Published by Polskie Towarzystwo Matematyczne Warszawa
% in Poland. Printed in Poland, by Wroclawska Drukarnia Naukowa.

% \backrefprint

% \bibitem[Hofmann (2000)]{hoffmann-2000}

% Martin Hoffmann. Laboratory for Foundations of Computer Science, University
% of Edinburgh. \emph{Programming languages capturing complexity classes}.
% 2000. ACM SIGACT News, Vol. 31, No. 1, pp. 31--42. Published by the ACM in
% New York, USA.

% \emph{A concise survey of a range approaches to implcit complexity theory
% between 1965 and 2000.}

% \backrefprint

\section*{Books}

\bibitem[Beman (1901)]{beman-1901}

Richard Dedekind. \emph{Theory of Numbers: The Nature and Meaning of Numbers}.
Authorized translation by Wooster Woodruff Beman, Professor of Mathemtics at
the University of Michigan, USA. Published in Chicago, by The Open Court Publishing Company.

Digitized by the Internet Archive in 2007 with funding from Microsoft
Corporation. Retrieved from
\url{http://www.archive.org/details/essaysintheoryofOOdedeuoft} on July 7,
2014.

\backrefprint

\bibitem[Tourlakis (1984)]{tourlakis-1984}

George J. Tourlakis. \emph{Computability}. Published by Reston Publishin
Company, Inc., A Prentice-Hall Company, in 1984. Printed in the United States.
ISBN: 0-8359-0876-3.

\backrefprint

\bibitem[Jones (1997)]{jones-1997}

Neil. D. Jones. \emph{Computability and Complexity: From a Programming
Perspective}. Published by The MIT Press in 1997.  Printed in the United
States. ISBN: 0-262-10064-9.

\backrefprint

% \bibitem[Graham et al. (1998)]{graham-et-al-1998}

% Ronald Lewis Graham, Donald Ervin Knuth, Oren Patashnik. \emph{Concrete
% Mathematics}. Second Edition. 1998. Published by Addison-Wesley Publishing
% Company, Inc. ISBN: 978-0-201-55802-9. 24th printing. 2011. Printed in
% Westford, Massachusetts, USA.

% \backrefprint

\bibitem[Huth \& Ryan (2004)]{huth-ryan-2004}

Michael Huth and Mark Ryan. Imperial College London and University of
Birmingham, United Kingdom. \emph{Logic in Computer Science: Modelling and
Reasoning about Systems}, Second Edition. Published by Cambridge University
Press in New York, USA in 2004. ISBN: 978-0-521-54310-1. 7th printing.  2011.
Printed in the United Kingdom by University Press, Cambridge.

\backrefprint

\bibitem[Cormen et al. (2009)]{cormen-et-al-2009}

Thomas H. Cormen, Charles E. Leiserson, Ronald L. Rivest, and Clifford Stein.
Massachusetts Institute of Technology.  \emph{Introduction to Algorithms}, 3rd
ed. Published by The MIT Press in 2009. Printed in the United States. ISBN:
978-0-262-03384-8.

\backrefprint

\bibitem[Mogensen (2010)]{mogensen-2010}

Torben Ægidius Mogensen. Department of Computer Science, University of
Copenhagen. \emph{Basics of Compiler Design}. Anniversary Edition. 2010.
Published through \texttt{lulu.com}. ISBN: 978-87-993154-0-6.

\backrefprint

% \bibitem[Rose (1984)]{rose-1984}

% H. E. Rose, School of Mathematics, University of Bristol. \emph{Subrecursion:
% functions and hierarchies}. 1984. Oxford Logix Guides: 9. Typeset by Joshua
% Associates, Oxford.  Published by Clarendon Press, division of Oxford
% University Press, in New York, USA. ISBN 0-19-853189. Printed in Great
% Britain, by The Thetford Press Ltd.

% \backrefprint

\bibitem[Homer \& Selman (2011)]{homer-selman-2011}

Steven Homer and Alan L. Selman. Boston University and University at Buffalo.
\emph{Computability and Complexity Theory}. Second Edition. Published by
Springer in 2011. ISBN: 978-1-4614-0681-5.

\backrefprint

\bibitem[Sipser (2013)]{sipser-2013}

Michael Sipser. \emph{Introduction to the Theory of Computation}. Third
Edition. Published by Cengage Learning in 2013. Printed in the United States.
ISBN: 978-1-133-18779-0.

\backrefprint



\bibitem[Dickson (1919)]{dickson-1919-goldbach}

Leonard Eugene Dickson. Professor of Mathematics in the University of Chicago.
\emph{Goldbach's Empirical Theorem: Every Even Integer is a Sum of Two Primes}.
In History of the Theory of Numbers, Volume 1: Divisibility and Primality, pp.
421--424.  Published by Carnegie Institute of Washington in 1919, 1920, and
1923.  Reprinted in New York in 1992. ISBN: 0-8218-1934-8.

\backrefprint

\section*{In Proceedings}

\bibitem[Turing (1936-7)]{turing-1936-7}

Alan M. Turing. Princeton University, USA. \emph{On computable numbers with an
application to the Entscheidungsproblem.} 1936-7. In Proceedings of The London
Mathematical Society, Second Series, Vol. 42. pp. 230–265. Originally printed
in Great Britain. Reprinted in Belgium in 1968 for The London Mathematical
Society.

\backrefprint

% \bibitem[Cobham (1965)]{cobham-1965}

% Alan Cobham. IBM Research Center, Yorktown Heights, New York, USA. \emph{The
% intrinsic computational difficulty of functions}. 1965. In Proceedings of the
% 1964 International Congress for Logic, Methodology and Philosophy of Science,
% pp. 24--30. Edited by Yehoshua Bar-Hillel. Published by North-Holland
% Publishing Company in Amsterdam, Holland.  Printed in Israel, by Jerusalem
% Academic Press Ltd.

% \backrefprint

% \bibitem[Sazonov (1987)]{sazonov-1987}

% V.Yu. Sazonov. Institute of Mathematics, Novosibirsk, USSR. \emph{Bounded set
% theory and polynomial computability}. 1987. In Proceedings of the 1987
% International Conference on Fundamentals of Computation Theory. Lecture Notes
% in Computer Science, Vol. 278, pp. 391--395. Editied by L. Budach, R.G.
% Bukharajev, and O.B. Lupanov. Published by Springer-Verlag in Berlin,
% Germany.  Printed by Druckhaus Beltz.

% \backrefprint

\section*{Other}

\bibitem[Joyce (2005)]{joyce-2005}

David E. Joyce. Clark University. USA. \emph{Notes on Richard Dedekind's Was
sind und was sollen de Zallen?} 2005. Retrieved from
\url{http://aleph0.clarku.edu/~djoyce/numbers/dedekind.pdf} on July 6, 2014.

Archived by WebCite\textsuperscript{\textregistered}\ at
\url{http://www.webcitation.org/6U9PH93PS}.

\backrefprint

% \bibitem[Bennett (1962)]{bennett-1962}

% J.H. Bennett. Ph.D.Thesis, Princeton University. \emph{On Spectra}. 1962.
% Princeton, New Jersey, USA.

% \emph{Referenced by \cite{cobham-1965}, not attained.}

% \backrefprint

\end{thebibliography}
