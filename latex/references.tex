\begin{thebibliography}{9} % 9 if < 10 references, 99 if < 100 references, etc.

\label{sec:references}

\section*{Journal articles}

\bib{Grzegorczyk (1953)}{grzegorczyk-1953}{%
%
Andrzej Grzegorczyk. Institut Matematiczny Polskiej Akademii Nauk. \emph{Some
classes of recursive functions}. 1953. Rozprawy Matematyczne: IV. Edited by
Karol Borsuk, et al. Published by Polskie Towarzystwo Matematyczne Warszawa in
Poland. Printed in Poland, by Wroclawska Drukarnia Naukowa.
%
}

\bib{Ritchie (1963)}{ritchie-1963}{%
%
Robert W. Ritchie, Princeton University and Dartmouth College. \emph{Classes of
predictably computable functions}. Transactions of the American Mathematical
Society 106(1), pp.  139--170. AMS, January 1963.  Online.
%
}

\bib{Feferman (1964)}{feferman-1964}{%
%
Solomon Fererman. \emph{Systems of Predicative Analysis}. The Journal of Symbolic Logic 29(1), pp. 1--30. Association for Symbolic Logic, March 1964. Online.
%
}

\bib{Blum (1967)}{blum-1967}{%
%
Michael Blum. \emph{A Machine-Independent Theory of the Complexity of Recursive
Functions}. Journal of the ACM 14(2), pp. 322--336. ACM, April 1967.  Online.
%
}

\bib{Cook (1971)}{cook-1971}{%
%
Stephen A. Cook, University of California, Berkley. \emph{Characterizations of
Pushdown Machines in Terms of Time-Bounded Computers}. Journal of the ACM
18(1), pp. 4--18. ACM, January, 1971. Online.
%
}


\bib{Immerman (1986)}{immerman-1986}{%
%
Neil Immerman. \emph{Relational Queries Computable in Polynomial Time}. Yale
University. January--March 1986. Information and Control, Vol. 68, No. 1--3,
pp.  86--104.  Published by Academic Press, Inc. Online.
%
\bibremark{A revised version of the conference paper \cite{immerman-1982}.} 
}

\bib{Di Paola \& Heller (1987)}{di-paola-heller-1987}{%
%
Robert A. Di Paola and Alex Heller. \emph{Dominical Categories: Recursion
Theory without Elements}. September 1987. The Journal of Symbolic Logic, Vol.
52, No. 3, pp.  594--635. Published by the Association of Symbolic Logic.
Online.
%
}

\bib{Immerman (1987)}{immerman-1987}{%
%
Neil Immerman. \emph{Languages that Capture Complexity Classes}. UMass,
Armherst. August 1987. SIAM Journal on Computing archive, Vol. 16, No. 4, pp.
760--778. Published by the Society for Industrial and Applied Mathematics
Philadelphia. Online.
%
}

\bib{Simmons (1988)}{simmons-1988}{%
%
Harold Simmons. \emph{The Realm of Primitive Recursion}. 1988. Archive for
Mathematical Logic, Vol. 27, No. 2, pp. 177--188.  Published by
Springer-Verlag. University of Aberdeen.
%
}

\bib{Moggi (1991)}{moggi-1991}{%
%
Eugenio Moggi. \emph{Notions of computation and monads}. July 1991. In
Selections from 1989 IEEE Symposium on Logic in Computer Science, Information
and Computation, Vol. 93, No. 1, pp. 55--92. Published by Academic Press, Inc.
University of Edinburgh.
%
}

\bib{Bellantoni \& Cook (1992)}{bellantoni-cook-1992}{%
%
Stephen J. Bellantoni and Stephen A. Cook. \emph{A new recursion-theoretic
characterizaton of the polytime functions}. University of Toronto. June 1992.
computational complexity, Vol. 2, No. 2, pp. 97--110. Published by
Birkh\"auser-Verlag Basel.
%
}

\bib{Dornic et al. (1992)}{dornic-et-al-1992}{%
%
Vincent Dornic, Pierre Jouvelot, and David K. Gifford.  \emph{Polymorphic time
systems for estimating program complexity}. March 1992.  ACM Letters on
Programming Languages and Systems (LOPLAS), Vol 1, No. 1, pp.  33--45.
%
}

\bib{Girard, Scedorov \& Scott (1992)}{girard-scedorov-scott-1992}{%
%
Jean-Yves Girard, Andre Scedorov, Philip J. Scott. \emph{Bounded Linear Logic:
A Modular Approach to Polynomial Time Computability}. April 1992. Theoretical
Computer Science, Vol. 97, No. 1, pp. 1--66. Published by Elsevier Science
Publishers B.V.
%
}

\bib{Cook \& Urquhart (1993)}{cook-urquhart-1993}{%
%
Stephen A. Cook and Alasdair Urquhart. \emph{Functional interpretations of
feasibly constructive arithmetic}. September 1993. Annals of Pure and Applied
Logic, Vol. 63, No. 2, pp. 103--200. Published by Elsevier Science Publishers
B.V.
%
}

\bib{Leivant (1994)}{leivant-1994}{%
%
Daniel Leivant. \emph{A Foundational Delineation of Poly-time}. 1994. Indiana
University.  Information and Computation, Vol. 110, No. 2, pp. 391--440.
Published by Academic Press Inc.
%
\bibremark{Characterizes \FPTIME{} as the functions which can be proven
convergent in second order logic with comprehension for positive quantifier
free, or existential formulas.}
%
}

\bib{Girard (1998)}{girard-1998}{%
%
Jean-Yves Girard. \emph{Light Linear Logic}. June 1998. Institut de
Math\'ematisques de Luminy.  Information and Computation, Vol. 143, No. 2, pp.
175--204.  Published by Academic Press Inc.
%
}

\bib{Beckmann \& Weiermann (1996)}{beckmann-weiermann-1996}{%
%
Arnold Beckmann and Andreas Weiermann.  \emph{A term rewriting characterization
of the polytime functions and related complexity classes}. December 1996.
Westf\"alischen Wilhelms-Universit\"at. Archive for Mathematical Logic, Vol.
36, No. 1, pp.  11--30. 
%
}

\bib{Ishihara (1999)}{ishihara-1999}{%
%
Hajime Ishihara. \emph{Function Algebraic Characterizations Of The Polytime
Functions}. December 1999.  computational complexity, Vol. 8, No. 4, pp.
346--356. Published by Birkh\"auser-Verlag Basel. Japan Advanced Institute of
Science and Technology. 
%
\bibremark{Considers a stronger version of concatenation recursion on notation
(as in \cite{clote-1990}), and gives a range of characterisations of PTIME.}
%
}

\bib{Jones (1999)}{jones-1999}{%
%
Neil D. Jones. \emph{LOGSPACE and PTIME characterized by programming
languages}. University of Copenhagen. October 1999. Theoretical Computer
Science, Vol. 228, No. 1--2, pp.  151--174. Published by Elsevier Science B.V.
%
}

\bib{Bellantoni, Niggl \& Schwichtenberg (2000)}{bellantoni-et-al-2000}{%
%
Stephen J. Bellantoni, Karl-Heinz Niggl, and Helmut Schwichtenberg.
\emph{Higher type recursion, ramification and polynomial time}.  Annals of Pure
and Applied Logic 104(1--3), pp. 17--30. Elsevier Science B.V., July 2000.
%
}

\bib{Hofmann (2000a)}{hofmann-2000a}{%
%
Martin Hofmann. \emph{Programming languages capturing complexity classes}.
University of Edinburgh. March 2000.  ACM SIGACT News, Vol. 31, No. 1, pp.
31--42.  Published by the ACM in New York, USA.
%
\bibremark{A concise survey of a range approaches to implcit complexity theory
between 1965 and 2000.}
%
}

\bib{Hofmann (2000b)}{hofmann-2000b}{%
%
Martin Hofmann. \emph{Safe recursion with higher types and BCK-algebra}.
University of Edinburgh. July 2000. Annals of Pure and Applied Logic Vol. 104,
No. 1--3, pp. 113--166. Published by Elsevier Science B.V.
%
\bibremark{An abridged and updated version of \cite{hofmann-1999}.}
%
}

\bib{Jones (2001)}{jones-2001}{%
%
Neil D. Jones. \emph{The expressive power of higher-order types or, life
without CONS}. March 2001. Journal of Functional Programming, Vol. 11, No. 1,
pp. 55--94. Published by Cambridge University Press. Online. University of
Copenhagen.
%
}

\bib{Aehlig \& Scwichtenberg (2002)}{aehlig-schwichtenberg-2002}{%
%
Klaus Aehlig and Helmut Schwichtenberg. \emph{A Syntactical Analysis of
Non-Size-Increasing Polynomial Time Computation}. Universit\"at M\"unchen. July
2002. ACM Transactions on Computational Logic (TOCL), Vol. 3, No. 3, pp.
383--401. Published by ACM New York. Online.
%
\bibremark{Provides an alternative approach to proving some results in
\cite{hofmann-2003}. Exhibits a lambda calculus, where any strategy to reduce a
term to a $\beta$-normal form is polynomially bounded.}
%
}

\bib{Asperti \& Roversi (2002)}{asperti-roversi-2002}{%
%
Andrea Asperti and Luca Roversi. Universit\'a di Bologna and Universit\'a di
Torino. \emph{Intuitionistic Light Affine Logic}. January 2002. ACM
Transactions on Computational Logic, Vol. 3, No. 1, pp. 137--175. Published by
ACM New York.
%
}

\bib{Hofmann (2003)}{hofmann-2003}{%
%
Martin Hofmann. \emph{Linear types and non-size-increasing polynomial time
computation}. May 2003. In Proceedings of International Workshop on Implicit
Computational Complexity (ICC'99), Information and Computation, Vol. 183, No.
1, pp.  57--85. Published by Elsevier Science (USA). Online.
Ludwig-Maximilians-Universit\"at M\"unchen.
%
\bibremark{Although this has the title of proceedings, according to Martin
Hofmann's home page this is indeed an ``expanded journal version''. Introduces
a special (non-constructible) ``resource type'', which is used for accounting
resources, in addition to using linear types to restrain reuse of variables.}
%
}

\bib{Marion (2003)}{marion-2003}{%
%
Jean-Yves Marion. \emph{Analysing the implicit complexity of programs}. In
Proceedings of International Workshop on Implicit Computational Complexity
(ICC'99), Information and Computation, Vol. 183, No.  1, pp.  2--18. Published
by Elsevier Science (USA). Online.
%
\bibremark{Combines a termination analysis technique from term rewriting with
predicative recursion. Gives a characterisation of FPTIME over free algebras
with ``simple constructors'', i.e. constructors of type $\tau_1 \rightarrow
\tau_2 \rightarrow \cdots \rightarrow \tau_n \rightarrow \tau$, where $\tau$
occurs at most once among $\tau_i$, and all $\tau$ and $\tau_i$ are atomic
types.}
%
}

\bib{Kristiansen \& Niggl (2004)}{kristiansen-niggl-2004}{%
%
Lars Kristiansen and Karl-Heinz Niggl. \emph{On the computational complexity of
imperative programming languages}. June 2004. Implicit Computational
Complexity, Theoretical Computer Science, Vol. 318, No. 1--2, pp. 139--161.
Published by Elsevier B.V. Oslo University College and Technische Universit\"at
Ilmenau.
%
}

\bib{Lafont (2004)}{lafont-2004}{%
%
Yves Lafont (Universite de le Mediterranee). \emph{Soft linear logic and
polynomial time}. June 2004. Theoretical Computer Science: Implicit
Computational Complexity, Vol. 318, No. 1--2, pp. 163--180. Published by
Elsevier B.V.
%
}

\bib{Kristiansen (2005)}{kristiansen-2005}{%
%
Lars Kristiansen. \emph{Neat function algebraic characterizations of LOGSPACE
and LINSPACE}. April 2005. computational complexity, Vol. 14, No. 1, pp.
72--88.  Published by Birkh\"auser-Verlag Basel.
%
\bibremark{Characterises the complexity classes without either explicit bounds
(as in \cite{cobham-1965}), variable segregation (as in
\cite{bellantoni-cook-1992, leivant-1995}), or ``complicated'' initial
functions (as in \cite{clote-1990,ishihara-1999}).}
%
}

\bib{Kristiansen \& Voda (2005)}{kristiansen-voda-2005}{%
%
Lars Kristiansen and Paul J. Voda. \emph{Programming languages capturing
complexity classes}. April 2005. Nordic Journal of Computing, Vol. 12, No. 2,
pp. 89--115. Published by Association Nordic Journal of Computing.
%
}

\bib{Niggl (2005)}{niggl-2005}{%
%
Karl-Heinz Niggl. \emph{Control structures in programs and computational
complexity}. Tehnische Universit\"at Ilmenau. May 2005. Festschrift on the
occasion of Helmut Schwichtenberg’s 60th birthday, Ein Bogen von der
Beweistheorie zur Informatik, Annals of Pure and Applied Logic, Vol. 133, No.
1--3, pp. 247--273.  Published by Elsevier B.V. Online. 
%
\bibremark{Summarises the results in \cite{bellantoni-et-al-2000,
kristiansen-niggl-2004}, among others. Considers three types of programs:
classical primitive recursive definitions, loop and stack programs, and
programs with higher-order types.}
%
}

\bib{Niggl \& Wunderlich (2006)}{niggl-wunderlich-2006}{%
%
Karl-Heinz Niggl and Henning Wunderlich. \emph{Certifying polynomial time and
linear/polynomial space for imperative programs}. Technische Universit\"at
Ilmenau. March 2006. SIAM Journal on Computing, Vol. 35, No. 5, pp. 1122--1147.
Published by the Society for Industrial and Applied Mathematics. Online.
%
\bibremark{Generalizes the techniques presented in \cite{niggl-2005} to
programming languages with mixed data types (rather than singly-typed), and
arbitrary polynomial time- or size-bounded basic instructions. This amounts to
a new method of certifying time and space bounds in general. The technique is
slightly weaker, but also slightly more general than
\cite{jones-kristiansen-2009}.}
%
}

\bib{Kristiansen (2008)}{kristiansen-2008}{%
%
Lars Kristiansen. \emph{Complexity-Theoretic Hierarchies Induced by Fragments
of G\"odel's T}. December 2008. Theory of Computing Systems, Vol. 43, No. 3--4,
pp. 516--541. Published by Springer Science+Business Media. Online.
%
}

\bib{Jones \& Kristiansen (2009)}{jones-kristiansen-2009}{%
%
Neil D. Jones and Lars Kristiansen. \emph{A Flow Calculus of $mwp$-Bounds for
Complexity Analysis}. University of Copenhagen and University of Oslo. August
2009. ACM Transactions on Computational Logic (TOCL), Vol. 10, No. 4, Article
28, 41 pages. Online. 
%
\bibremark{Introduces a novel method of detecting the polynomial-size bounded
variables in a program. Only imperative programs on natural numbers are
considered: generalizing the technique to richer languages is left as an open
question. The technique is slightly stronger than
\cite{niggl-wunderlich-2006}.}
%
}

\bib{Marion (2009)}{marion-2009}{%
%
Yves Marion. \emph{On tiered small jump operators}. Nancy Universit\'e. March
2009. Logical Methods in Computer Science, Vol. 5, No. 1, pp. 1--19. Published
under the Creative Commons Attibution-NonDerivs License.
%
}

\bib{dal Lago \& Hofmann (2010)}{dal-lago-hofmann-2010}{%
%
Ugo dal Lago and Martin Hofmann. Universet\'a di Bologna and LMS M\"unchen.
\emph{Bounded Linear Logic, Revisited}.  December 2010. Logic Methods in
Computer Science, Vol. 6, No. 4, pp. 1--31. Published under a Creative Commons
license.
%
}

\bib{Berger \& Moerdijk (2011)}{berger-moerdijk-2011}{%
%
Clemens Berger and Ieke Moerdijk. \emph{On an extension of the notion of Reedy
category}. 2011. Mathematische Zeitschrift, Vol. 269, pp. 977--1004.
%
}

\bib{Avanzini et al. (2012)}{avanzini-et-al-2012}{%
%
Martin Avanzini, Naohi Eguchi, and Georg Moser (University of Innsbruck and
Tohoku University). \emph{A New Order-Theoretic Characterisation of the
Polytime Computable Functions}. 2012. Programming Languages and Systems,
Lecture Notes in Computer Science, Vol. 7705, pp. 280--295.
%
}

\bib{dal Lago (2012)}{dal-lago-2012}{%
%
Ugo dal Lago. Universet\'a di Bologna. \emph{A Short Introduction to Implicit
Computational Complexity}. ESSLLI 2010/2011 Lectures, Lecture Notes in Computer
Science, Vol. 7388, pp. 89--109. Published by Springer-Verlag Berlin
Heidelberg.
%
}

\bib{de Carvalho \& Simonsen (2014)}{de-carvalho-simonsen-2014}{%
%
Daniel de Carvalho and Jakob Grue Simonsen (University of Copenhagen). \emph{An
Implicit Characterization of the Polynomial-Time Decidable Sets by Cons-Free
Rewriting}. 2014.  Rewriting and Typed Lambda Calculi, Lecture Notes in
Computer Science, Vol. 8560, pp. 179--193.
%
}

\section*{Books}

\bib{Beman (1901)}{beman-1901}{%
%
Richard Dedekind. \emph{Theory of Numbers: The Nature and Meaning of Numbers}.
Authorized translation by Wooster Woodruff Beman, Professor of Mathemtics at
the University of Michigan, USA. Published in Chicago, by The Open Court
Publishing Company.
%
\bibremark{Digitized by the Internet Archive in 2007 with funding from
Microsoft Corporation. Retrieved from
\url{http://www.archive.org/details/essaysintheoryofOOdedeuoft} on July 7,
2014.}
%
}

\bib{Wittgenstein (1953)}{wittgenstein-1953}{%
%
Ludwig Wittgenstein. \emph{Philosophical Investigations}. The German text, with
a revised English translation.  Translated by G.E.M. Ancombe. Third Edition.
Published by Blackwell Publishing in 1953, 1958, and 2001. ISBN:
978-0-631-23159-2.
%
}

\bib{Markov (1954)}{markov-1954}{%
%
Andrey Andreyevich Markov Jr. Academy of Sciences of the U.S.S.R. \emph{Theory
of Algorithms}. Works of the Mathematical Institute Im. V.A. Steklov. Vol.
XLII. Published by Izdatel'stvo Akademii Nauk SSSR in 1954. Translated by
Jacques J. Schorr-Kon and PST Staff. Second Impression. Published by the Israel
Program of Scientific Translations Ltd. in 1962. Printed in Jerusalem.
%
}

\bib{Tourlakis (1984)}{tourlakis-1984}{%
%
George J. Tourlakis. \emph{Computability}. 1984. Published by Reston Publishing
Company, Inc., A Prentice-Hall Company. ISBN: 0-8359-0876-3.
%
}

\bib{Minsky (1967)}{minsky-1967}{%
%
Marvin L. Minsky, MIT. \emph{Computation: Finite and Infinite Machines}.
Series in Automatic Computation. Prentice-Hall, 1967. Print.
%
}

\bib{Odifreddi (1989)}{odifreddi-1989}{%
%
Piergiorgio Odifreddi, University of Turin. \emph{Classical Recursion Theory:
The Theory of Functions and Sets of Natural Numbers}. Studies in Logic and The
Foundations of Mathematics 125. Elsevier, 1989, 1992.
%
}

\bib{Jones (1997)}{jones-1997}{%
%
Neil. D. Jones. \emph{Computability and Complexity: From a Programming
Perspective}. In series M. Garey and A. Meyer (Eds.), Foundations of Computing.
The MIT Press, 1997. ISBN: 0-262-10064-9. Online.
%
\bibremark{Retrieved on June 16, 2014, from \\
\url{http://www.diku.dk/~neil/comp2book2007/book-whole.pdf}.\\
%
Archived by WebCite\textsuperscript{\textregistered}\ at
\url{http://www.webcitation.org/6VsEoaGr0}.}
%
}

% \bibitem[Graham et al. (1998)]{graham-et-al-1998}

% Ronald Lewis Graham, Donald Ervin Knuth, Oren Patashnik. \emph{Concrete
% Mathematics}. Second Edition. 1998. Published by Addison-Wesley Publishing
% Company, Inc. ISBN: 978-0-201-55802-9. 24th printing. 2011. Printed in
% Westford, Massachusetts, USA.

% \backrefprint

\bib{Huth \& Ryan (2004)}{huth-ryan-2004}{%
%
Michael Huth and Mark Ryan. Imperial College London and University of
Birmingham, United Kingdom. \emph{Logic in Computer Science: Modelling and
Reasoning about Systems}, Second Edition. Published by Cambridge University
Press in New York, USA in 2004. ISBN: 978-0-521-54310-1. 7th printing.  2011.
Printed in the United Kingdom by University Press, Cambridge.
%
}

\bib{Cormen et al. (2009)}{cormen-et-al-2009}{%
%
Thomas H. Cormen, Charles E. Leiserson, Ronald L. Rivest, and Clifford Stein.
Massachusetts Institute of Technology.  \emph{Introduction to Algorithms}, 3rd
ed. Published by The MIT Press in 2009. Printed in the United States. ISBN:
978-0-262-03384-8.
%
}

\bib{Mogensen (2010)}{mogensen-2010}{%
%
Torben Ægidius Mogensen. Department of Computer Science, University of
Copenhagen. \emph{Basics of Compiler Design}. Anniversary Edition. 2010.
Published through \texttt{lulu.com}. ISBN: 978-87-993154-0-6.
%
}

\bib{Rose (1984)}{rose-1984}{%
%
H. E. Rose, School of Mathematics, University of Bristol. \emph{Subrecursion:
functions and hierarchies}. 1984. Oxford Logix Guides: 9. Typeset by Joshua
Associates, Oxford.  Published by Clarendon Press, division of Oxford
University Press, in New York, USA. ISBN 0-19-853189. Printed in Great Britain,
by The Thetford Press Ltd.
%
}

\bib{Immerman (1999)}{immerman-1999}{%
%
Neil Immerman. \emph{Descriptive Complexity}. 1999. In D. Gries and F.B.
Schneider (Eds.), Graduate Texts in Computer Science. Published by Springer
Verlag New York, Inc. ISBN: 0-387-98600-6. Print.
%
}

\bib{Homer \& Selman (2011)}{homer-selman-2011}{%
%
Steven Homer and Alan L. Selman. Boston University and University at Buffalo.
\emph{Computability and Complexity Theory}. Second Edition. Published by
Springer in 2011. ISBN: 978-1-4614-0681-5.
%
}

\bib{Sipser (2013)}{sipser-2013}{
%
Michael Sipser. \emph{Introduction to the Theory of Computation}. Third
Edition. Published by Cengage Learning in 2013. Printed in the United States.
ISBN: 978-1-133-18779-0.
%
}

\bib{Dickson (1919)}{dickson-1919-goldbach}{%
%
Leonard Eugene Dickson. Professor of Mathematics in the University of Chicago.
\emph{Goldbach's Empirical Theorem: Every Even Integer is a Sum of Two Primes}.
In History of the Theory of Numbers, Volume 1: Divisibility and Primality, pp.
421--424.  Published by Carnegie Institute of Washington in 1919, 1920, and
1923.  Reprinted in New York in 1992. ISBN: 0-8218-1934-8.
%
}

\section*{Refereed articles in Books}

\bib{Klop \& Vrijer (2003)}{klop-vrijer-2003}{%
%
Jan Willem Klop and Roel de Vrijer. \emph{First-order term rewriting systems}.
In Terese (Ed.), Term Rewriting Systems, Cambridge Tracts in Theoretical
Computer Science, vol. 55. Cambridge University Press, 2003.
%
}

\section*{Refereed articles in Proceedings}

\bib{Russell (1907)}{russell-1907}{%
%
Bertrand Russell. \emph{On Some Difficulties in the Theory of Transfinite
Numbers and Order Types}. In Proceedings of The London Mathematical Society,
Second Series, Vol. 4, No. 1, pp. 29--53. Reprinted in Belgium in 1966 for The
London Mathematical Society. Online.
%
}

\bib{Turing (1936-7)}{turing-1936-7}{%
%
Alan M. Turing. \emph{On computable numbers with an application to the
Entscheidungsproblem}. Princeton University. 1936-7. In Proceedings of The
London Mathematical Society, Second Series, Vol. 42. pp. 230--265. Originally
printed in Great Britain. Reprinted in Belgium in 1968 for The London
Mathematical Society. Online.
%
}

\bib{Cobham (1965)}{cobham-1965}{%
%
Alan Cobham. \emph{The intrinsic computational difficulty of functions}.  IBM
Research Center. 1965. In Yehoshua Bar-Hillel (Ed.), Proceedings of the 1964
International Congress for Logic, Methodology and Philosophy of Science, pp.
24--30. Published by North-Holland Publishing Company. Print.
%
\bibremark{Seminal article on Computational Complexity: hypothesising that
PTIME functions form the practically feasible functions; introducing ``bounded
recursion on notation''. Alternate presentations and proofs appear in
\cite{rose-1984, tourlakis-1984}. }
%
}

\bib{Fagin (1974)}{fagin-1974}{%
%
Ronald Fagin. \emph{Generalized First-Order Spectra and Polynomial-Time
Recognizable Sets}. 1974. In Proceedings of a symposium in applied mathematics
of the American Mathematical Society and the Society for Industrial and Applied
Mathematics (SIAM-AMS), Vol. 7, pp. 43--73. Published by the American
Mathematical Society. ISBN: 0-8218-1327-7. Print.
%
\bibremark{Presents what is now called ``Fagin's theorem'', stating that the
complexity class NP is equivalent to existential second-order boolean logic. An
accessible presentation and proof of the theorem appears in
\cite{immerman-1999}.}
%
}

\bib{Cook (1975)}{cook-1975}{%
%
Stephen A. Cook, University of Toronto. \emph{Feasibly constructive proofs and
the propositional calculus (Preliminary Version)}. In Proceedings of the 7th
annual ACM SIGACT Symposium on Theory of Computing (STOC'75), pp. 83--97. ACM
New York, May 1975. Online.
%
}

\bib{Immerman (1982)}{immerman-1982}{%
%
Neil Immerman. \emph{Relational Queries Computable in Polynomial Time (Extended
Abstract)}. May 1982. In Proceedings of the 14th ACM SIGACT Symposium on Theory
of Computing (STOC'82), pp. 147--152. Published by ACM New York. Online.
%
\bibremark{Revised in \cite{immerman-1986}. Referenced for historical reasons:
some results are discovered independently, in the same volume, see
\cite{vardi-1982}.}
%
}

\bib{Vardi (1982)}{vardi-1982}{%
%
Moshe Y. Vardi. \emph{The Complexity of Relational Query Languages}. Stanford
University. May 1982. In Proceedings of the 14th ACM SIGACT Symposium on Theory
of Computing (STOC'82), pp. 137--146. Published by ACM New York. Online.
%
\bibremark{Some results are discovered independently in \cite{immerman-1982}.}
%
}

\bib{Gifford \& Lucassen (1986)}{gifford-lucassen-1986}{%
%
David K. Gifford and John M. Lucassen. \emph{Integrating Functional and
Imperative Programming.} MIT Laboratory for Computer Science.  August 1986. In
Proceedings of the 1986 ACM conference on LISP and functional programming
(LFP'86).  Published by ACM New York. ISBN: 0-89791-200-4.
%
\bibremark{Introducing ``effect types''.}
%
}

\bib{Clote (1990)}{clote-1990}{%
%
Peter G. Clote. \emph{Sequential, machine-independent characterizations of the
parallel complexity classes AlogTIME, AC\textsuperscript{k} ,
NC\textsuperscript{k} and NC}. 1990. In Samuel L. Buss and Philip J. Scott
(Eds.), Feasible Mathematics, Progress in Computer Science and Applied Logic,
Vol. 9, pp. 49--69.  Proceedings of a Workshop on Feasible Mathematics, Cornell
University, 1989.  Published by Birkh\"auser Boston. ISBN: 978-1-4612-3466-1.
\vspace{0.1in}\\
%
\emph{Introducing ``concatenation recursion on notation'', which uses neither
explicit bounds (as in \cite{cobham-1965}), nor variable segregation (as in
\cite{bellantoni-cook-1992,leivant-1995}). Mentions how the complexity class NC
can also be characterised by a subset of the Pascal programming language,
citing Technical Report BCCS-88-07 at Boston College for further details.}
%
}

\bib{Leivant (1990)}{leivant-1990}{%
%
Daniel Leivant. \emph{Subrecursion and lambda representation over free algebras
(Preliminary summary)}. 1990. In Samuel L. Buss and Philip J. Scott (Eds.),
Feasible Mathematics, Progress in Computer Science and Applied Logic, Vol. 9,
pp. 281--291. Proceedings of a Workshop on Feasible Mathematics, Cornell
University, 1989.  Published by Birkh\"auser Boston. ISBN: 978-1-4612-3466-1.
%
}

\bib{Leivant (1993)}{leivant-1993}{%
%
Daniel Leivant. \emph{Stratified functional programs and computational
complexity}. March 1993. In Proceedings of the 20th ACM SIGPLAN-SIGACT
symposium on Principles of Programming Languages (POPL'93), pp.  325--333.
Published by ACM New York. Indiana University.
% ISBN: 0-89791-560-7.
%
\bibremark{Extended and revised in \cite{leivant-1995}.} }

\bib{Reistad \& Gifford (1994)}{reistad-gifford-1994}{%
%
Brian Reistad and David K. Gifford. \emph{ Static dependent costs for
estimating execution time}. June 1994. In Proceedings of the 1994 ACM
conference on LISP and functional programming (LFP'94), pp. 65--78. Published
by ACM New York. MIT.
% ISBN: 0-89791-643-3.
%
}

\bib{Voda (1994)}{voda-1994}{%
%
Paul J. Voda. \emph{Subrecursion as a Basis for a Feasible Programming
Language}. September 1994. In Leszek Pacholski Jerzy Tiuryn (eds.),
\emph{Selected Papers from the 8th International Workshop on Computer Science
Logic} (CSL'94), pp. 324--338.  Published by Springer Verlag London. University
of Bratislava.
% ISBN: 3-540-60017-5.
%
}

\bib{Leivant (1995)}{leivant-1995}{%
%
Daniel Leivant. \emph{Ramified Recurrence and Computational Complexity I: Word
Recurrence and Poly-time}. Indiana University. January 1995.  In Peter Clote
and Jeffrey B. Remmel (eds.), Feasible Mathematics II, Progress in Computer
Science and Applied Logic, Vol. 13, pp.  320--343. Proceedings of the Second
Workshop on Feasible Mathematics, Cornell University, 1992. Published by
Birkh\"auser Boston.  ISBN: 978-1-4612-7582-4.
%
\bibremark{A generalization of \cite{leivant-1990} and
\cite{bellantoni-cook-1992} from two-tiered to many-tiered recursion. This is
the complete version of the earlier conference paper on tiered, aka.
``ramified'', or ``stratified'' recursion \cite{leivant-1993}.}
%
}

\bib{Bonfante et al. (1999)}{bonfante-et-al-1999}{%
%
G. Bonfante, A. Cichon, J. Y Marion, H. Touzet. \emph{Complexity Classes and
Rewrite Systems with Polynomial Interpretation}. 1999. In Proceedings of the
1998 Annual Conference of the European Association for Computer Science Logic
(EACSL). Computer Science Logic, Lecture Notes in Computer Science, Vol. 1584,
pp. 372--384. Published by Springer-Verlag Berlin Heidelberg. ISBN
978-3-540-48855-2.
%
}

\bib{Crary \& Weirich (2000)}{crary-weirich-2000}{%
%
Karl Crary and Stephnie Weirich. \emph{Resource bound certification}. Carnegie
Mellon and Cornell University. January 2000. In Proceedings of the 27th ACM
SIGPLAN-SIGACT symposium on Principles of programming languages (POPL'00), pp.
184--198.  Published by ACM New York. ISBN:1-58113-125-9.
%
}

\bib{Marion \& Moyen (2000)}{marion-moyen-2000}{%
%
Jean-Yves Marion and J.-Y. Moyen. \emph{Efficient First Order Functional
Program Interpreter with Time Bound Certifications}. Nancy-Universit\'e. 2000.
In M. Parigot and A. Voronkov (Eds.), Proceedings of the Seventh International
Conference on Logic for Programming and Automated Reasoning (LPAR 2000),
Lecture Notes in Artificial Intelligence (LNAI), Vol. 1955, pp. 25--42.
Published by Springer Verlag Berlin Heidelberg.
%
\bibremark{Expands on the work in \cite{marion-2003}, by replacing predicative
recursion with an additional semantic requirement. This admits primitive
recursion over non-size increasing base and step  functions, capturing e.g. LCS
and insertion sort.}
%
}

\bib{Hofmann (2002)}{hofmann-2002}{%
%
Martin Hofmann, LMU M\"unchen. \emph{The strength of non-size increasing
computation}. In Proceedings of the 29th ACM SIGPLAN-SIGACT symposium on
Principles of programming languages (POPL'02), pp. 260--269. ACM New York,
January 2002. Online.
%
}

\bib{Hofmann \& Jost (2003)}{hofmann-jost-2003}{%
%
Martin Hofmann and Steffen Jost, LMU M\"unchen. \emph{Static Prediction of Heap
Space Usage for First-Order Functional Programs}. In Proceedings of the 30th
ACM SIGPLAN-SIGACT symposium on Principles of programming languages (POPL'03),
pp. 185--197. ACM New York, January 2003. Online.
%
}

\bib{Kristiansen \& Jones (2005)}{kristiansen-jones-2005}{%
%
Lars Kristiansen and Neil D. Jones. \emph{The Flow of Data and the Complexity
of Algorithms}. 2005. In S.B. Cooper, B. Löwe, and L. Torenvliet (Eds.),
Proceedings of the First Conference on Computability in Europe (CiE 2005), New
Computational Paradigms, Lecture Notes in Computer Science, Vol. 3526, pp.
263--274. Published by Springer-Verlag Berlin Heidelberg.
%
\bibremark{Seeking a better understanding of the relationship between
syntactical constructions and computational complexity (rather than seeking
implicit characterisations of particular classes of computational complexity
per se).}
%
}

\bib{Marion \& P\'echoux (2008)}{marion-pechoux-2008}{%
%
Jean-Yves Marion and Romain Péchoux. \emph{Characterizations of polynomial
complexity classes with a better intensionality}. Nancy-Universit\'e. July
2008. In Proceedings of the 10th international ACM SIGPLAN conference on
Principles and practice of declarative programming (PPDP'08), pp. 79--88.
%
}

% \bibitem[Sazonov (1987)]{sazonov-1987}

% V.Yu. Sazonov. Institute of Mathematics, Novosibirsk, USSR. \emph{Bounded set
% theory and polynomial computability}. 1987. In Proceedings of the 1987
% International Conference on Fundamentals of Computation Theory. Lecture Notes
% in Computer Science, Vol. 278, pp. 391--395. Editied by L. Budach, R.G.
% Bukharajev, and O.B. Lupanov. Published by Springer-Verlag in Berlin,
% Germany.  Printed by Druckhaus Beltz.

% \backrefprint

\section*{Other}

\bib{G\"odel (1944)}{goedel-1944}{%
%
Kurt G\"odel. \emph{Russell's Mathematical Logic}. 1944. In P. Benacerraf and
H. Putnam (Eds.), Philosophy of Mathematics: Selected Readings, 2nd edition,
March 1984, pp. 447--469. Reprinted with comments from P.A. Schlipp (Ed.), The
Philosophy of Bertrand Russell, 1944, pp. 125--153. Online.
%
}

\bib{Smullyan (1961)}{smullyan-1961}{%
%
Raymond M. Smullyan. \emph{Theory of Formal Systems}. Annals of Mathematics
Studies 47. Princeton University Press, 1961. Print.
%
\bibremark{A revision of the author's doctoral thesis.}
%
}

\bib{Buss (1985-6, PhD thesis)}{buss-phd-1985-6}{%
%
Samuel R. Buss. \emph{Bounded Arithmetic}. 1986. Revision of the 1985 PhD
thesis. Princeton University. Online.
%
}

\bib{Bellantoni (1992, PhD thesis)}{bellantoni-phd-1992}{%
%
Stephen J. Bellantoni. \emph{Predicative Recursion and Computational
Complexity}. September 1992. PhD Thesis. University of Toronto. Online.
%
}

\bib{Dornic (1993)}{dornic-1993}{%
%
Vincent Dornic (Yale University). \emph{Ordering Times}. April 1993. Research
Report YALEU/DCS/RR-956.
%
}

\bib{Caseiro (1996)}{caseiro-1996}{%
%
Vuokko-Helena Caseiro. Series of tech reports. University of Oslo. Online.
%
\begin{itemize}
%
\item Tech Report 224. \emph{Some General Criteria on Equations to Guarantee
Poly-time Functions}. November 1996.
%
\item Tech Report 225. \emph{Criticality Conditions on Equations to Ensure
Poly-time Functions}. November 1996.
%
\item Tech Report 226. \emph{An Equational Characterization of the Poly-time
Functions on any Constructor Data Structure}. December 1996.
%
\end{itemize}
%
\emph{Introducing the ``Don't Double Criticals'' technique, which is slightly
more general than tiered recursion, but not quite as general as non-size
increasing computation.}
%
}

\bib{Clote (1999)}{clote-1999}{%
%
Peter G. Clote. \emph{Computation models and function algebras}. 1999. In E.
Griffor (Ed.), Handbook of computability theory, pp.  589--681. Published by
Elsevier Science B.V. ISBN: 0-444-89882-4. Print.
Ludwig-Maximilians-Universit\"at M\"unchen.
%
}

\bib{Hofmann (1999, Habilitation thesis)}{hofmann-1999}{%
%
Martin Hofmann. \emph{Type systems for polynomial-time computation}. February
1999. Habilitation thesis. Unviersit\"at Darmstadt. Appeared as a Laboratory
for Foundations of Computer Science at the University of Edinburgh (LFCS)
Technical Report ECS-LFCS-99-406. Online.
%
\bibremark{An abridged and updated version appears in \cite{hofmann-2000b}.}
%
}

\bibitem[Joyce (2005)]{joyce-2005}

David E. Joyce. Clark University. USA. \emph{Notes on Richard Dedekind's Was
sind und was sollen de Zallen?} 2005. Retrieved from
\url{http://aleph0.clarku.edu/~djoyce/numbers/dedekind.pdf} on July 6, 2014.

Archived by WebCite\textsuperscript{\textregistered}\ at
\url{http://www.webcitation.org/6U9PH93PS}.

\backrefprint

% \bibitem[Bennett (1962)]{bennett-1962}

% J.H. Bennett. Ph.D.Thesis, Princeton University. \emph{On Spectra}. 1962.
% Princeton, New Jersey, USA.

% \emph{Referenced by \cite{cobham-1965}, not attained.}

% \backrefprint

\bib{Baillot et al. (2006)}{baillot-et-al-2006}{%
%
Patrick Baillot, Jean-Yves Marion, and Simona Ronchi Della Rocca. \emph{Special
Issue on Implicit Computational Complexity}. February 2006. In preface to the
Proceedings of a Workshop on Implicit Computational Complexity - Geometry of
Computation (GEOCAL'06), ACM Transactions on Computational Logic, Vol. 10, No.
4. 
%
}

\bib{Roversi (2007)}{roversi-2007}{%
%
Luca Roversi. Universit\'a di Torino. \emph{Weak Affine Light Typing: Polytime
intensional expressivity, soundness and completeness}. 2007. Computing Research
Repository. Revised as of December 2013. arXiv:0712.4222v2 [cs.LO].
%
}

\bib{dal Lago et al. (2010)}{dal-lago-et-al-2010}{%
%
Ugo Dal Lago, Simone Martini, and Margherita Zorzi. \emph{General Ramified
Recurrence is Sound for Polynomial Time}. 2010. In P. Baillot (Ed.),
Proceedings of the International Workshop on Developments in Implicit
Computational complExity (DICE 2010), EPTCS, Vol. 23, pp. 47--62. Published
under a Creative Commons Attribution License. Online.
%
}

\end{thebibliography}
