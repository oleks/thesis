\chapter{Effect Types}

The idea of effect types is to include in the type system not just information
about the input/output relationship of computations, but also the effects of
computations.

In the functional programming community, the notion ``side-effects'' is often
neglected and outright feared. It is hopefully non-negotiable however, that
computations as a ``side-effect'' consume time and space. We thus find it only
natural to employ the notion of effect types to decorate our type system with
information about the complexities of our computations.

In support of this motion, we may quote the seminal article by Gifford and
Lucassen, entitled ``Integrating Functional and Imperative Programming'',
regarding the intended purpose of effect types, in general:

\begin{quotation}

\footnotesize\sffamily\itshape

Every expression has an effect class; just as the \emph{type} of an expression
describes the \emph{value} computed by the expression, the \emph{effect class}
of an expression describes \emph{how} that value is computed. [\ldots]

The effect class of an expression determines the sublanguage to which the
expression must be confined, which in turn determines which language facilities
the expression may use and what subroutines it may call. At the same time, the
side-effect specification of a subroutine determines which sublanguages may
call it.

\begin{flushright}

\footnotesize\sffamily

--- \cite{gifford-lucassen-1986}

\end{flushright}

\end{quotation}
