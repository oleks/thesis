\part{Subpolynomialism}

We turn now to the consideration of subpolynomial bounds. We wish for a
baseline system which captures polynomial time, and admits subsystems with
explicitly stated subpolynomial bounds.

For lack of a better word, we draw inspiration from the term ``colonialism''.

\begin{quotation}

Colonialism is a practice of domination, which involves the subjugation of one
people to another \ldots The term colony comes from the Latin word
\emph{colonus}, meaning farmer. This root reminds us that the practice of
colonialism usually involved the transfer of population to a new territory,
where the arrivals lived as permanent settlers while maintaining political
allegiance to their country of origin.

\begin{flushright}

\footnotesize\sffamily

\upshape

--- The Stanford Encyclopedia of Philosophy \itshape (Spring 2014 Edition)

\end{flushright}

\end{quotation}

``Subpolynomailism'' is similar in this regard, as this too involves the
``transfer of population (programmers) to a new territory'', where they can
only write programs within a given subpolynomial bound, ``maintaining political
allegiance to their country of origin'', in the sense that they can return to
the baseline system, or move to another subsystem, for other lines of work.

We consider two approaches to subpolynomialism. The approaches differ in a
fundamental way:

\begin{enumerate}

\item The first takes a substructural approach, where we review earlier
attempts to integrate bounds into the logic of the type system itself. In
particular, we briefly review various bounded, light, and soft linear and
affine logics.

\item The second takes a substructural approach only to guarantee that a
baseline system is bounded to PTIME. Subpolynomial bounds are achieved for
subsystems by sprinkling effect types on top of this baseline system. To our
knowledge, this is a novel approach.

\end{enumerate}

\chapter{Substructural Bounds}

\begin{quotation}

\footnotesize\sffamily\itshape

\begin{flushright}

The abuse of structural rules may have damaging complexity effects.

\smallbreak

\upshape

--- Jean-Yves Girard, {\itshape Light Linear Logic} (1998)

\end{flushright}

\end{quotation}

We review literature on bounded linear logic, as first introduced in
\cite{girard-scedorov-scott-1992}, and revisited in
\cite{dal-lago-hofmann-2010}.


\chapter{Effect Types}

The idea of effect types is to include in the type system not just information
about the input/output relationship of computations, but also the effects of
computations.

In the functional programming community, the notion ``side-effects'' is often
neglected and outright feared. It is hopefully non-negotiable however, that
computations as a ``side-effect'' consume time and space. We thus find it only
natural to employ the notion of effect types to decorate our type system with
information about the complexities of our computations.

In support of this motion, we may quote the seminal article by Gifford and
Lucassen, entitled ``Integrating Functional and Imperative Programming'',
regarding the intended purpose of effect types, in general:

\begin{quotation}

Every expression has an effect class; just as the \emph{type} of an expression
describes the \emph{value} computed by the expression, the \emph{effect class}
of an expression describes \emph{how} that value is computed \ldots

% Examples of effect classes include ``OBSERVER'' and ``FUNCTION'': an
% expression that is an OBSERVER can observe side-effects, but it can not cause
% them, while a FUNCTION can neither cause nor observe side-effects.

The effect class of an expression determines the sublanguage to which the
expression must be confined, which in turn determines which language facilities
the expression may use and what subroutines it may call. At the same time, the
side-effect specification of a subroutine determines which sublanguages may
call it.

\begin{flushright}

\footnotesize\sffamily

--- \cite{gifford-lucassen-1986}

\end{flushright}

\end{quotation}

