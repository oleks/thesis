\part{Subpolynomialism}

\label{section:subpolynomialism}

We turn now to the consideration of subpolynomial bounds. We wish for a
baseline system which captures polynomial time, and admits subsystems with
explicitly stated subpolynomial bounds.

For lack of a better word, we draw inspiration from the term ``colonialism'':

\begin{quotation}

\footnotesize\sffamily\itshape

Colonialism is a practice of domination, which involves the subjugation of one
people to another [\ldots] The term colony comes from the Latin word
\emph{colonus}, meaning farmer. This root reminds us that the practice of
colonialism usually involved the transfer of population to a new territory,
where the arrivals lived as permanent settlers while maintaining political
allegiance to their country of origin.

\begin{flushright}

\upshape

--- The Stanford Encyclopedia of Philosophy \itshape (Spring 2014 Edition)

\end{flushright}

\end{quotation}

``Subpolynomailism'' is similar in this regard, as this too involves the
``transfer of population'' (programmers) ``to a new territory'', where their
programs can have at most a particular subpolynomial complexity, ``maintaining
political allegiance to their country of origin'', in the sense that they can
return to the baseline system, or move to another subsystem, for other lines of
work.

We consider two approaches to subpolynomialism. The approaches differ in a
fundamental way:

\begin{enumerate}

\item The first takes a substructural approach, where we review earlier
attempts to integrate bounds into the logic of the type system itself. In
particular, we briefly review various bounded, weak, light, and soft linear and
affine logics.

\item The second takes a substructural approach only to guarantee that a
baseline system is bounded to PTIME. Subpolynomial bounds are achieved for
subsystems by sprinkling effect types on top of this baseline system. To our
knowledge, this is a novel approach.

\end{enumerate}

We compare these approaches according to two formal, and one informal criteria.
We define these criteria as follows:


Although the word ``extensional'' suggests perhaps a dependence upon the base
system, extensional soundness can be proven independent of the base system.
For instance, we may prove extensional soundness by simulating $\mathcal{S_C}$
on a Turing machine with resources bounded to $\mathcal{C}$.

\begin{definition} A subsystem $\mathcal{S_C}$, for some complexity class
$\mathcal{C}$, is \textbf{intensionally complete} if all the programs in the
base system which lie in $\mathcal{C}$ are in $\mathcal{S_C}$. \end{definition}

% why is intensional completeness often not recursively enumerable?

% we will not consider whether a system can embed other existing system
% characterising the same classes, although this does increase confidence in
% their expressiveness, perhaps this _is_ worth doing for the review of
% polytime characterisations.

% we do not consider type inference, not least because type inference will
% oftentimes be undecidable when complexities are involved in the types.

Following the criticisms made by various authors\cite{bellantoni-cook-1992,
hofmann-2000, dal-lago-hofmann-2010}, it can be said that when bounds appear
stated explicitly, then we are no longer dealing with truly \emph{implicit}
computational complexity.

We ignore this aspect partly out of practical motivations for this thesis, and
partly as a simplifying assumption. We seek a programming language where bounds
are initially stated explicitly for user-defined functions, and are known á
priori for built-in functions. Having this in order, we conjecture that it
could be a worth-while venture to consider a system of \emph{type inference}
for, if nothing else, the lower order subpolynomial complexity classes. Here,
we can employ the wealth of existing literature on the approximation of the
complexity classes for user-defined programs. We conjecture, that the guarantee
that all programs terminate, as otherwise there would be programs outside
PTIME, can lift many hurdles of the inference of complexity classes.

Although our ways involve explicit bounds, none include ``machine models'' as such.

\chapter{Effect Types}

The idea of effect types is to include in the type system not just information
about the input/output relationship of computations, but also the effects of
computations.

In the functional programming community, the notion ``side-effects'' is often
neglected and outright feared. It is hopefully non-negotiable however, that
computations as a ``side-effect'' consume time and space. We thus find it only
natural to employ the notion of effect types to decorate our type system with
information about the complexities of our computations.

In support of this motion, we may quote the seminal article by Gifford and
Lucassen, entitled ``Integrating Functional and Imperative Programming'',
regarding the intended purpose of effect types, in general:

\begin{quotation}

\footnotesize\sffamily\itshape

Every expression has an effect class; just as the \emph{type} of an expression
describes the \emph{value} computed by the expression, the \emph{effect class}
of an expression describes \emph{how} that value is computed. [\ldots]

The effect class of an expression determines the sublanguage to which the
expression must be confined, which in turn determines which language facilities
the expression may use and what subroutines it may call. At the same time, the
side-effect specification of a subroutine determines which sublanguages may
call it.

\begin{flushright}

\footnotesize\sffamily

--- \cite{gifford-lucassen-1986}

\end{flushright}

\end{quotation}

The use of effect types to denote the time or space complexity of a computation
is not new, aka. cost systems: a mix of time systems and sized types, is not
new. It has previously been used in connection with \emph{estimating} such
complexities\cite{dornic-et-al-1992,dornic-1993,reistad-gifford-1994}. Their
use for space complexity analysis is however somewhat
disputed\cite{crary-weirich-2000}. The use in connection with ICC is (to our
knowledge) new.

