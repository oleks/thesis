\chapter{Polytime}

Better title: Implicit Computational Complexity

% Also, better suited as an introduction as it doesn't really say anything
% about the choice of polynomial time from the get go. That's more a
% coincidence, the following definitions will hold regardless. Perhaps it is
% worth it calling this a Part I: Introduction and Background

% Fragment of a logic means that we are restricted to a subset of the syntax,
% but retain the same semantics. If we choose the combination of words -
% subsystem of system, we are perhaps a little more free, but in either case,
% we probably have to further specify what we keep and what we take out.

% for the purposes of this thesis, we may regard logical systems equivalent to
% programming languages - draw inspiration from Types of Crash Prevention - it
% has a nice, human readable introduction to the whole thing.

The correspondence between an ICC system and a complexity class is extensional,
i.e. the class of function (or problems) representable in the system equals the
complexity class.

% by this point we should have more precisely defined the notions of function
% and problem.

The systems that we will consider here will be subsystems of a larger base
system, in which other functions, besides those in the complexity class of the
system can be represented.

\begin{definition} \textit{Intensional soundness}

Any program representable in the ICC system, is within the given complexity
class. Proven by showing equivalence of the class of problems to a complexity
class.

\end{definition}

\begin{definition} \textit{Intensional completeness}

All functions in the complexity class are representable in the ICC system.

\end{definition}

\begin{definition} \textit{Extensional soundness}

All functions in the complexity class are representable in the base system.

\end{definition}

\section{Natural numbers}

We begin by defining a theory of natural numbers, $\mathcal{N}$. This is
distinct from the set of natural numbers, $\mathbb{N}$, which may form the
universe of a model of $\mathcal{N}$.

\begin{definition} \textit{The theory of natural numbers, $\mathcal{N}$.}

Let the language
$\mathcal{L}_{\mathcal{N}}=\chev{\set{0},\set{s(x)},\set{x<y,x=y}}$, consist of
a constant symbol $0$ (read: zero), a unary function $s$ (read: successor), and
a binary relation $<$ (read: less than). The class of natural numbers is
axiomatized by the $\mathcal{L}_{\mathcal{N}}$-sentences

\begin{align}
&\vdash 0 < s(x) \\
x < y &\vdash s(x) < s(y)
\end{align}

\end{definition}

\begin{example} \textit{A model $\mathcal{M}$ of $\mathcal{N}$.}

Let the universe be the set of natural numbers, $\mathbb{N}$. Let the constant
zero be the numeral $0$, and the successor function be the mathematical
function $s(x)=x+1$. Let the less than relation be the usual $<$ relation on
$\mathbb{N}$.

Notably, if $\mathcal{M}$ is a model of $\mathcal{N}$, satisfying the above
axioms, $s$ is an \emph{injective} function, and $<$ is an \emph{irreflexive}
and \emph{transitive} relation.

\end{example}

\begin{remark}

The theory of natural numbers as defined above, closely resembles the notion of
natural numbers as originally given by Dedekind\cite{beman-1901,joyce-2005}.
Unlike the more conventional Peano/Dedekind axioms, which you might be familiar
with, this notion focuses on the ordered nature of the natural numbers.

\end{remark}

\begin{definition} \emph{The theory of strings over an alphabet $\Sigma$}

Let the language
$\mathcal{L}_\Sigma=\chev{\set{\varepsilon}\cup\Sigma,\set{x\cdot y},\set{}}$,
consist of the constant symbols $\Sigma$, the empty symbol $\varepsilon \notin
\Sigma$, and a binary function $\cdot$ (read: concatenate). The class of
strings over an alphabet $\Sigma$ is axiomatized by the
$\mathcal{L}_\Sigma$-sentences

\begin{align}
&\vdash \varepsilon \cdot x = x & \text{(left identity)} \\
&\vdash x \cdot \varepsilon = x & \text{(right identity)} \\
&\vdash \p{ x \cdot y } \cdot z = x \cdot \p{ y \cdot z } & \text{(associativity)}
\end{align}

\end{definition}

% we want natural numbers, but we don't want the successor and predecessor
% operations to always be blindly permitted.

% primitive recursion is really defined in terms of a predecessor operation, at
% least if we assume the successor operation to merely be something that can
% construct a "bigger" value. otherwise, it is undefined how in primitive
% recursion we first compute the lower-order value.

% we run into the same problem with s_1 and s_0! Does it at all make sense to
% define a successor without a predecessor? At least not if we want to be able
% to define functions recursively! Any time we define a function computing a
% bigger value having a lower value, we need to specify it in terms of
% predecessors!

% decrement on binary notation is a polynomial-time operation if we only assume
% s_0 and s_1 (or rather, their predecessors). We can assume dec to be a
% unit-cost operation, and that is where the problems of primitive recursion
% creep up.

% although we can always convert a unique string to a unique natural number, it
% may take a long time to do so. Furthermore, the successor operation is a polynomial time oper

Any more definite attempt at a definition of the natural numbers seems to
arrive at a philosophical impasse.

The natural numbers have never-the-less shown to be of prime importance due to
the concept of recursive enumerability.

Use sets rather than ``language'' to avoid mixing up inputs/outputs of machines
and the programming languages that we design. Languages refer specifically to
sets of strings.

\begin{definition}

Define enumerator.

\end{definition}

\begin{definition}

Define recursively enumerable sets.

\end{definition}

Dealing with natural numbers is bad as we would quickly regard the successor
(and predecessor) operations to be paired and unit-cost operations.

% You cannot take the successor operation to be a unit-cost operation, nor can
% you assume that it is paired with a predecessor.

% we are dealing with words? basic data elements on any machine model?

% an element of state in a machine model. it has an initial state, and some
% succeeding states - preceding states don't really make sense.

% a successor does not lead you to another state

% at least size is not defined in terms of the number states that you've passed.

% for all classical machine models, the class of polynomial-time computable
% functions is equivalent.

% the values of any practical data structure can be recursively enumerated; so
% although the fact that a set is recursively enumerable opens a pandoras box,
% only a practical subset of it will be used in practice.

% this is an explanation of godel encoding.

% strings or states of a data structure?

% due to godel encoding, it is okay to deal in natural numbers, provided that
% we do not (necessarily) permit a unit-cost unique successor operation.

% so we're actually redefining natural numbers again? - yes.

% the natural numbers that you do have in any programming language, depend on
% the constructs available in your language. indeed, due to godel encoding, any
% practical programming languages comes with manifolds of ways of defining
% natural numbers.

This makes it enticive to accept the natural numbers as a basic primitive

% is all this because we want to deal in the natural numbers?

Our choice of available successor and predecessor

In practice, we often deal with recursive sets, as by the Church-Turing thesis,
these are  the computable functions. This makes it attractive to deal merely in
the natural numbers as introduced above. Although this is perhaps a useful
definition of natural numbers, the pairing of a recursively enumerable set with
the natural numbers can be a rather complex procedure.

We intend to model this complexity by permitting a more elaborate definition of
natural numbers:

\begin{notion}

A \textbf{value} is either $0$, a successor, or a predecessor of a value. A
successor and predecessor need not take $O(1)$ to compute.

\end{notion}

The notions of successor and predecessor are defined below.

% It is intentional that we do not denote the syntax of $x$.

% what is a data type?

\begin{definition}

The \textbf{notation} of a particular data type is a specification of how
values of that data type may be constructed and deconstructed.

\end{definition}

\section{Successors}

% multiple possible successor and predecessor functions.

% TODO: define <, =, \vdash

% to define <, we need to define size; so < is defined for some measure of size.

% The rules with < occurring so early is not a problem as it is a statement
% that these rules must hold, either as part of the definition of a successor
% s, or as an admissible rule.

\begin{definition}

A function $s$ is a \textbf{successor} function iff

\begin{align}
&\vdash 0 < s(x) \\
x < y &\vdash s(x) < s(y)
\end{align}

\end{definition}

Successor functions play a key role in the remainder of the thesis, as they
form the basis of our measure of the size of a value.

\begin{definition}

The \textbf{size} of a value $x$, the minimum number of successor applications
necessary to construct $x$.

\end{definition}

For instance, a language $\mathcal{B}$ that uses binary notation may have two
successors, one that doubles the value, $s_0$, and another that doubles the
value and adds a one, $s_1$. The value $5$ is then represented in $\mathcal{B}$
as $s_1\p{s_0\p{s_1\p{0}}}$, and so the size of the value $5$ in $\mathcal{B}$
is $3$. Analytically, the size of a value $x$ in $\mathcal{B}$ is
$\ceil{\log_2\p{x}}$.

\section{Predecessors}

\begin{definition}

A function $p$ is a \textbf{predecessor} function iff

% no cycles?

\begin{align}
&\vdash p\p{0} = 0 \\
&\vdash p\p{x} < x & x \neq 0 \\
x < y &\vdash p\p{x} < p\p{y}
\end{align}

\end{definition}

For instance, the language $\mathcal{B}$ above, may have a predecessor function
$dec$ (read: decrement) defined as follows:

\begin{align}
dec'\p{x}=y, trim\p{y}=z &\vdash dec\p{x} = z \\
dec'\p{x} = y &\vdash dec'\p{s_1\p{x}} = s_1\p{y} \\
dec''\p{x} = y &\vdash dec'\p{s_1\p{x}} = s_0\p{y} \\
dec'\p{x} = y &\vdash dec'\p{s_0\p{x}} = s_0\p{y} \\
dec''\p{x} = y &\vdash dec''\p{s_0\p{x}} = s_1\p{y} \\
&\vdash dec''\p{s_0\p{0}} = s_1\p{0} \\
&\vdash dec'\p{s_1\p{0}} = s_0\p{0} \\
&\vdash dec'\p{0} = 0 \\
trim\p{y} = z &\vdash trim\p{s_0\p{y}} = z \\
&\vdash trim\p{s_1\p{y}} = s_1\p{y} \\
&\vdash trim\p{0} = 0
\end{align}

For instance, $s_1\p{s_0\p{s_1\p{0}}}$, i.e. 5, decremented, is
$s_1\p{s_0\p{s_0\p{0}}}$, i.e. 4:

$$
\judgement{
  \judgement{
    \judgement{
      \axiom{
        dec'\p{s_1\p{0}} = s_0\p{0}
      }
    }{
      dec'\p{s_0\p{s_1\p{0}}} = s_0\p{s_0\p{0}}
    }
  }{
    dec'\p{s_1\p{s_0\p{s_1\p{0}}}} = s_1\p{s_0\p{s_0\p{0}}}
  }
  \quad
  \axiom{
    trim\p{s_1\p{s_0\p{s_0\p{0}}}} = s_1\p{s_0\p{s_0\p{0}}}
  }
}{
  dec\p{s_1\p{s_0\p{s_1\p{0}}}} = s_1\p{s_0\p{s_0\p{0}}}
}
$$

and $s_1\p{s_0\p{s_0\p{0}}}$, i.e. 4, decremented, is $s_1\p{s_1\p{0}}$, i.e. $3$:

$$
\judgement{
  \judgement{
    \judgement{
      \axiom{
        dec''\p{s_0\p{0}} = s_1\p{0}
      }
    }{
      dec''\p{s_0\p{s_0\p{0}}} = s_1\p{s_1\p{0}}
    }
  }{
    dec'\p{s_1\p{s_0\p{s_0\p{0}}}} = s_0\p{s_1\p{s_1\p{0}}}
  }
  \quad
  \judgement{
    \axiom{
      trim\p{s_1\p{s_1\p{0}}} = s_1\p{s_1\p{0}}
    }
  }{
    trim\p{s_0\p{s_1\p{s_1\p{0}}}} = s_1\p{s_1\p{0}}
  }
}{
  dec\p{s_1\p{s_0\p{s_0\p{0}}}} = s_1\p{s_1\p{0}}
}
$$

\section{Successors and predecessors}

\begin{definition}

A predecessor function $p$ is \textbf{paired} with a successor function $s$ iff

\begin{align}
\vdash p\p{s\p{x}} &= x
\end{align}

\end{definition}

It is not a requirement for a predecessor to be paired with a successor. A
language may have predecessors more powerful than any of its successors.

\begin{remark}

A successor or predecessor function may or may not take $O(1)$ time. This
depends on the machine model.

\end{remark}

\section{Primitive recursion}

\begin{definition}

A function is defined by \textbf{primitive recursion} (PR), from PR functions
$g$ and $h$, and a PR predecessor function $p$, iff

\begin{align}
f\p{0,\vect{y}} &= g\p{\vect{y}} \\
f\p{x, \vect{y}} &= h\p{x, \vect{y}, f\p{p\p{x},\vect{y}}}
\end{align}

\end{definition}

The running time of a PR function $f$, depends on the running times of $g$,
$h$, and $p$, as well as how quickly $p$ decreases $x$ to $0$.

Mere primitive recursion is not confined to polytime computation.

\begin{theorem}
PR is not confined to polytime computation.
\end{theorem}

\begin{proof} Assume that $g$, $h$, and $p$ are all computed in $O(1)$ time.
Consider the language $\mathcal{B}$ above, having just the successors $s_0$ and
$s_1$. Let $p$ be the function $dec$ above.\end{proof}

The trouble is that $p$ does not decrease $x$ fast enough throughout the
recursion --- the number of times $p$ has to be applied to $x$ to reach $0$, is
superpolynomial in $x$.

One attempt to mitigate for this could be to have $p$ decrease the value
faster. For instance, sticking with the language $\mathcal{B}$ above, we could
let $p$ be a function that halves $x$. This would pair $p$ with $s_0$.

\begin{theorem}

Primitive recursion defined with a predecessor function which is paired with a
successor function is not confined to polytime computation.

\end{theorem}

\begin{proof} Let a predecessor function $p$ be paired with a successor
function $s$.

Consider a function $dbl$, which doubles its input:

\begin{align}
dbl\p{0} &= 0 \\
dbl\p{x} &= s\p{s\p{dbl\p{p\p{x}}}}
\end{align}

Consider furthermore a function $iterdbl$, which calls $dbl$ iteratively, the
same number of times as the length of its input:

\begin{align}
iterdbl\p{0} &= s\p{0} \\
iterdbl\p{x} &= dbl\p{iterdbl\p{p\p{x}}}
\end{align}

Calling $iterdbl$ with a string of length $n$, we obtain a string of length
$2^n$ due to iterated doubling of the input. It follows that $iterdbl$ does a
superpolynomial amount of work in the size of its input.\end{proof}

\begin{example}
We illustrate the above proof with an example:
\begin{align}
\underline{iterdbl\p{s\p{s\p{0}}}}
  &\leadsto dbl\p{iterdbl\p{\underline{p\p{s\p{s\p{0}}}}}} \\
  &\leadsto dbl\p{\underline{iterdbl\p{s\p{0}}}} \\
  &\leadsto dbl\p{dbl\p{iterdbl\p{\underline{p\p{s\p{0}}}}}} \\
  &\leadsto dbl\p{dbl\p{\underline{iterdbl\p{0}}}} \\
  &\leadsto dbl\p{\underline{dbl\p{s\p{0}}}} \\
  &\leadsto dbl\p{s\p{s\p{dbl\p{\underline{p\p{s\p{0}}}}}}} \\
  &\leadsto dbl\p{s\p{s\p{\underline{dbl\p{0}}}}} \\
  &\leadsto \underline{dbl\p{s\p{s\p{0}}}} \\
  &\leadsto s\p{s\p{dbl\p{\underline{p\p{s\p{s\p{0}}}}}}} \\
  &\leadsto s\p{s\p{\underline{dbl\p{s\p{0}}}}} \\
  &\leadsto s\p{s\p{s\p{s\p{dbl\p{\underline{p\p{s\p{0}}}}}}}} \\
  &\leadsto s\p{s\p{s\p{s\p{\underline{dbl\p{0}}}}}} \\
  &\leadsto s\p{s\p{s\p{s\p{0}}}} \\
\end{align}
\end{example}

In the proof above, we fall prey to the admission for the function $h$ in a
primitive recursive function $f$ to generate value larger in size than its
input.

We could try to mitigate for this, but this would be getting ahead of
ourselves. Bounded primitive recursion bounds the primitive recursion in a more
general way, by bounding $f$ itself.

\section{Bounded primitive recursion}
