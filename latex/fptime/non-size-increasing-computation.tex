\chapter{Non-Size Increasing Computation}

\begin{quotation}

\footnotesize\sffamily\itshape

\begin{flushright}

there is a relationship between the absence and presence of successor-like
functions and the computational complexity of programs

\smallbreak

\upshape

--- \cite{jones-kristiansen-2009}

\end{flushright}

\end{quotation}

Although it might seem due to name this section ``Impredicative Recursion'', it
is perhaps too general a name for this section to end up concise. Besides,
drawing such a cleavage might charm us in the wrong direction.

Some impredicative definitions cause us trouble --- but certainly not all.  In
a similar vein, a purely ``predicative'' theory of sets did not catch on
either, in lieu of softer restrictions one could impose on set comprehension
(for more, see \cite{feferman-1964}). An adequate choice of axioms for a theory
of sets, remains a context-dependent question.

An early investigation into when impredicative recursion causes trouble is
perhaps due to \cite{caseiro-1996}. She notes that ``the real problem [...] is
not that [...] recurs on its input, but that [...] \emph{doubles} its input''.
This leads to the development of a technique she calls ``Don't Double
Criticals'' (DDC), where the critical arguments are inferred automatically, and
the recurrence function is \emph{semantically} prohibited in doubling the
result of a recursive call. A proof that DDC captures \FPTIME{} does appear in
\cite{caseiro-1996}, but the work was never published. Instead, in its attempt
to ``investigate how to treat critical arguments linearly'', it has inspired
the many works of others \cite{bellantoni-et-al-2000,
aehlig-schwichtenberg-2002, hofmann-2003}.

At roughly the same time, \cite[\textsection~24]{jones-1997}, characterised
\PTIME{} using first-order read-only programs. The characterisation was later
published in \cite{jones-1999}, and extended to arbitrary finite data orders
and tail-recursive programs in \cite{jones-2001}. The divergent paths later
conflate in e.g. \cite{hofmann-2002}, as it uses an important proof technique
of \cite{jones-2001}.

The work we focus on here is \cite{aehlig-schwichtenberg-2002}, which is
considered an alternate take on the results of \cite{hofmann-2003}.

\begin{definition} Let $\NSIFPTIME{}$ be the class of those functions $f \in
\FPTIME{}$, where $f : A \rightarrow B$, for which $\card{f\p{a}} \leq
\card{a}$ for all $a \in A$. That is, the output of the function is no greater
in size than the input of the function. \end{definition}

\subsection{Discussion}

The insights of Caseiro et al., also led to a generalisation of the results of
\cite{marion-2003} in the conference paper \cite{marion-moyen-2000}. In both,
the authors deal in simple algebras, i.e.  algebras with types $\sigma_1 \times
\cdots \times \sigma_n \rightarrow \sigma$, where each $\sigma_i$ occurs at
most once among $\sigma_1 \times \cdots \times \sigma_n$ (so a tree algebra is
not a simple algebra). \todo{The conference paper captures \FPTIME{} with a
semantic argument, and shows that $\proc{LCS}$ and $\proc{Insertion-Sort}$ are
representable in the system.} The results never made it past the conference,
perhaps due to the use of simple algebras, in lieu of the more general results
of e.g.  \cite{aehlig-schwichtenberg-2002} and \cite{hofmann-2003}.
