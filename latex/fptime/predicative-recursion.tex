\chapter{Predicative Recursion}

\begin{quotation}

\footnotesize\sffamily\itshape

\begin{flushright}

What's in a name? that which we call a rose \\
By any other name would smell as sweet;

\smallbreak

\upshape

--- WILLIAM SHAKESPEARE, \emph{Romeo and Juliet}, Act II, Scene II

\end{flushright}

\end{quotation}

That which we call ``predicative'' recursion is also referred to as
``tiered''\cite{leivant-1990}, ``stratified''\cite{leivant-1993}, or
``ramified''\cite{leivant-1995} recursion. (The conflation of these terms
appears prevalent in mathematics.) Our motivation is due to
\cite{hofmann-2000a}, categorising in this fell swoop, both the seminal works
\cite{bellantoni-cook-1992} and \cite{leivant-1995}. Hofmann was in turn
motivated by \cite{bellantoni-phd-1992}.

The use of the term ``predicative'' itself is due to \cite{russell-1907}, who
suggested a ``predicative'' theory of sets to deal with the plaguing paradoxes
in na\"ive set theory, e.g. Russell's or liar paradox. Informally, the
definition of an entity is ``impredicative''\cite{goedel-1944}, if it refers
(directly, or indirectly) to a totality to which the entity being defined
itself belongs.  Many recursive definitions fall squarely into such totalities. 

\todo{It is forbidden to do recursion on a critical argument, i.e.
specifically, the result of a self-recursive call.}

The refinement due in this section is based on the idea that the values we
recur on in a recursive definition are of a ``different nature'' from the
rest\cite{caseiro-1996}. Consequently, a segregation of values must commence.
\todo{Segregation is done on the input values. Distinction between two-tiered
and many-tiered recursion (and mu measure?). See also \cite{leivant-1995}.}

% Two classical examples of an impredicative definition gone rogue are
% Russell's and the Liar paradoxes: 

% \begin{description}

% \item [Russell's paradox] Define the set $R$ to be the set of all sets which
% are not members of themselves. If $R$ is a member of itself, then it should
% not be a member of itself; if it is not a member of itself, then it should be
% a member of itself. Paradox by indirect self-reference.

% \item [Liar paradox] Define the liar sentence ``This sentence is false.'' If
% the sentence is true, then it is false; if it is false, then it is true.
% Paradox by direct self-reference.

% \end{description} 

\subsection{Two-tiered recursion}

Two-tiered recursion, also referred to as ``safe'' or ``controlled'' recursion,
was independently introduced in \cite{simmons-1988}, \cite{leivant-1990}, and
\cite{bellantoni-cook-1992}, and generally attributed to the latter.

The idea, as indicated above, is to segregate the values into ``normal'' and
``safe'' values. The normal values are assumed known in their totality. The
safe values are those (perhaps) obtained by impredicative means, i.e. via
recursion\cite{bellantoni-cook-1992, clote-1999}. Safely, as in, without
exceeding \FPTIME{}.

The variables are segregated purely syntactically, e.g. using a semicolon. In
particular, if $f$ is an $m+n$-ary symbol, with $m$ normal inputs and $n$ safe
inputs, it is written as $f\p{x_1,\ldots,x_m \semic{} x_{m+1},\ldots,x_{m+n}}$,
where $x_1,\ldots,x_m$ are the $m$ normal inputs, and $x_{m+1},\ldots,x_{m+n}$
are the $n$ safe inputs. If the eloquent programmer finds the explicit
segregation ineloquent, the segregation can be inferred from how the values
$x_1,\ldots,x_n$ are used\cite{caseiro-1996}.

The characterisation in \cite{bellantoni-cook-1992} is again given in terms of
a single-sorted word signature, in particular, the natural numbers in binary
notation. We give a characterisation equivalent up to choice of words and
notation, except that

\begin{enumerate}[label=(\arabic*)]

\item we give a (syntax-directed) operational semantics of the operators to
emphasise the syntactic nature of the characterisation; and

\item we make the distinction between normal and safe inputs explicit in the
type system, as suggested in \cite[\textsection\ 5]{bellantoni-cook-1992}.

\end{enumerate}

\begin{definition} \cite{bellantoni-cook-1992}

\begin{enumerate}[label=(\arabic*)]

\item Let $\mathbb{N}$ and $\Box\mathbb{N}$ denote the normal and safe data
types, respectively.

\item Let $\mathtt{z} : \Box\mathbb{N}$ be a \textbf{constant} operator, having
the following semantics:$$
%
\axiom{ \mathtt{z} \goesto \repr{0} }
%
$$

\item Let $\mathtt{s}_0,\mathtt{s}_1 : \Box\mathbb{N} \rightarrow
\Box\mathbb{N}$ be (binary) \textbf{successor} operators, having the following
semantics:$$
%
\judgement{ a \goesto \repr{n} }{
%
  \mathtt{s}_0\p{\semic{} a} \goesto \repr{2n}
%
} \quad
%
\judgement{ a \goesto \repr{n} }{
%
  \mathtt{s}_1\p{\semic{} a} \goesto \repr{2n + 1}
%
}
%
$$

\item Let $\mathtt{p} : \Box\mathbb{N} \rightarrow \Box\mathbb{N}$ be a
(binary) \textbf{predecessor} operator, having the following semantics: $$
%
\axiom{
%
  \mathtt{p}\p{\semic{} \mathtt{z}} \goesto \repr{0}
%
}
%
\quad\axiom{
%
  \mathtt{p}\p{\semic{} \mathtt{s}_i\p{\semic{} a}} \goesto a
%
}\p{\text{for $i \in \set{0,1}$}}
%
$$

\item Let $\pi_j : \mathbb{N}^m \times \Box\mathbb{N}^n \rightarrow
\Box\mathbb{N}$ be a generalised (safe) \textbf{projection} operator, for all
$1 \leq j \leq m+n$, given by the following semantics:$$
%
\axiom{ \pi_j\p{x_1,\ldots,x_m \semic{} x_{m+1},\ldots,x_{m+n}} \goesto x_j }
%
$$

Note, in particular, $\pi_j$ projects the input safe, and so $\pi_j$ is
\emph{not} a mere operational interpretation of the generalised projection
operator $\sem{\mathtt{PROJ}_j}$.

\item Let $\mathtt{cond} : \Box\mathbb{N} \times \Box\mathbb{N} \times
\Box\mathbb{N} \rightarrow \Box\mathbb{N}$ be a (safe) \textbf{conditional}
operator, having the following semantics:$$
%
\judgement{
%
  a \goesto \repr{n} \quad b \goesto m
%
}{
%
  \mathtt{cond}\p{\semic{} a, b, c} \goesto m
%
}\p{n \bmod 2 = 0}
%
\quad\judgement{
%
  a \goesto \repr{n} \quad c \goesto m
%
}{
%
  \mathtt{cond}\p{\semic{} a, b, c} \goesto m
%
}\p{n \bmod 2 \neq 0}
%
$$

\item Let $\mathtt{sprn}$ be a \textbf{safe primitive recursion on notation}
operator, having the following semantics, where $f \triangleq
\mathtt{sprn}\p{g, h_0, h_1}$:\begin{align*}
%
\judgement{
%
  g\p{\vect{y} \semic{} \vect{a}} \goesto c
%
}{
%
  f\p{\mathtt{z}, \vect{y} \semic{} \vect{a}} \goesto c
%
}\end{align*}\begin{align*}
%
\judgement{
%
  h_i\p{x, \vect{y} \semic{} \vect{a}, f\p{x,\vect{y} \semic{} \vect{a}}}
\goesto c
%
}{
%
  f\p{\mathtt{s}_i\p{x}, \vect{y} \semic{} \vect{a}} \goesto c
%
}\p{\text{for $i\in\set{0,1}$}}
%
\end{align*}

\item For every $f : \mathbb{N}^n \rightarrow \Box\mathbb{N}$, where $n \in
\mathbb{N}$, there is a corresponding $g : \mathbb{N}^n \rightarrow
\mathbb{N}$, in accordance with the following axiom:\begin{align*}
%
\axiom{f\p{x_1,\ldots,x_n} \goesto g\p{x_1,\ldots,x_n}}
%
\end{align*}

This permits the coersion of safe outputs to normal outputs.

\item Let $\mathcal{B} = \p{\set{\mathtt{s}_0, \mathtt{s}_1, \mathtt{p}, \pi_j,
\mathtt{cond}},\set{\mathtt{comp}, \mathtt{sprn}}} $, where $\mathtt{comp}$ is
defined as in \refSec{generalised-composition}, with $S=\mathbb{N}$ and
$T=U=\Box\mathbb{N}$.

\end{enumerate}

\end{definition}

\begin{remark} The size function is defined as before. In particular, for
binary notation, $\card{n}=\ceil{\log_2\p{n+1}}$ for all $n \in \mathbb{N}$.
\end{remark}

\begin{lemma} \label{lem:b-fptime-sound} $\mathcal{B}$ is \FPTIME{}-sound.
\end{lemma}

\begin{proof} (Sketch, see also \cite{bellantoni-cook-1992}.) By simulating
$\mathcal{B}$ in $\mathcal{L}$.  \todo{Rest of sketch missing.} \end{proof}

\begin{lemma} \label{lem:b-fptime-complete} $\mathcal{B}$ is
\FPTIME{}-complete.  \end{lemma}

\begin{proof} (Sketch, see also \cite{bellantoni-cook-1992}.) By simulating
$\mathcal{L}$ in $\mathcal{B}$. \todo{Rest of sketch missing.} \end{proof}

\begin{theorem} $\mathcal{B}$ captures \FPTIME{}. \end{theorem}

\begin{proof} Follows directly from \refLem{b-fptime-sound} and
\refLem{b-fptime-complete}. \end{proof}

\subsubsection{Example}

\subsubsection{Discussion}

\subsubsection{$\mu$-measure}

Before we turn to many-tiered recursion..

\subsection{Many-tiered recursion}

The generalization of two-tiered recursion to many-tiered recursion is
generally attributed to \cite{leivant-1995}. The intuition is that inputs come
bearing varying computational weights --- some are too heavy to drive a
recurrence.

In the style of \cite{leivant-1990}, the motivation for this work is to give an
algebraic, rather than a numeric, account of tiered recursion, drawing
parallels to programming with algebraic data types. As a ``technical
advantage'', the work also boasts an implicit characterisation of
$O\p{n^k}$-\TIME{} for any given $k\in \mathbb{N}$. This is considered useful
to Part II.

The characterisation in \cite{leivant-1995} is given in terms of free algebras.
The proofs of capture are given, ``without loss of generality'', in terms of
word algebras. Unfortunately, the means of proof do not generalise to tree
algebras\cite{caseiro-1996, dal-lago-et-al-2010}.

The conference paper \cite{dal-lago-et-al-2010} presents an alternative proof,
generalising tiered recursion to single-sorted free algebras. Their
characterisation is given below, equivalent up to choice of words and notation,
except:

\begin{enumerate}[label=(\arabic*)]

\item we give a (syntax-directed) operational semantics of the oprators to
emphasise the syntactic nature of the characterisation;

\item we let the tiers be a part of the type system rather than they stand on
their own, in leau of what we did for two-tiered recursion; and

\item we present the characterisation in terms of a many-sorted, rather than a
single-sorted signature, in leau of the above, and generalising slightly.

\end{enumerate}

\begin{definition} \cite{dal-lago-et-al-2010}

\begin{enumerate}[label=(\arabic*)]

\item Let $\Sigma$ be a many-sorted signature, decorated with tiers, and let
$\sigma : \Sigma \rightarrow \mathbb{N}$ give the arities of the symbols in
$\Sigma$.

\todo{A type signature?}

\item Let $s : S_1^i \times \cdots \times S_n^i \rightarrow T^i$ be the type of
any symbol $s \in \Sigma$. That is, the constructors of the algebra operate on
same-tier inputs, and preserve tier.

\item Let $\mathtt{proj}_j$ be a generalised projection operator, as in
\refSec{generalised-projection},  parametrised by the types $S_1^{i_1}, \ldots,
S_n^{i_n}$, and $S_j^{i_j}$, where $1 \leq j \leq n$.

\item Let $\mathtt{comp}_j$ be a generalised composition operator, as in
\refSec{generalised-composition}, parametrised by the types $S_1^{i_1}, \ldots,
S_n^{i_n}$, $T_1^{j_1}, \ldots, T_m^{j_m}$, and $U^k$.

Note, in particular, that for any $f : V^0 \rightarrow V^1$ and $g : V^0
\rightarrow V^0$, $\mathtt{comp}\p{f,g}$ is well-typed, whereas
$\mathtt{comp}\p{g,f}$ is not.

\item Let $\mathtt{trec}$ be a \textbf{tiered primitive recursion} operator,
having the following semantics: \begin{align*}
%
f \triangleq \mathtt{trec}\p{g_1,\ldots,g_m} \quad\quad
%
\vect{x} \triangleq x_1, \ldots, x_{\sigma\p{s_k}} \quad\quad
%
\vect{y} \triangleq y_1, \ldots, y_n \\
%
\judgement{
%
  g_l\p{\vect{x}, \vect{y}, f\p{x_1, \vect{y}}, \ldots, f\p{x_{\sigma\p{s_k}},
\vect{y}}} \goesto z
%
}{
%
  f\p{s_l\p{\vect{x}},\vect{y}} \goesto z
%
}\p{\text{for $k \in \set{1,\ldots,m}$}}
%
\end{align*} where, for $i<j$, the types are as follows, having $\mathbf{T}
\triangleq T_1^{k_1} \times \cdots \times T_n^{k_n}$:\begin{align*}
%
f &: S^j \times \mathbf{T} \rightarrow U^i \\
%
g_k &:  S_1^j \times \cdots \times S_{\sigma\p{s_l}}^j \times \mathbf{T} \times
\underbrace{U^i \times \cdots \times U^i}_{\text{$\sigma\p{s_l}$ times}}
\rightarrow U^i
%
\end{align*}

The key point with $\mathtt{trec}$ is that the tier of the value we recur on,
must be strictly greater than the tier of the value we produce with
$\mathtt{trec}$ ($i < j$).

\end{enumerate}

\end{definition}

The notable change with the tiered primitive recursion operator
($\mathtt{trec}$), in contrast to the previously introduced bounded primitive
recursion operator ($\mathtt{BPRN}$), and safe primitive recursion operator
($\mathtt{sprn}$), is that $\mathtt{trec}$ makes multiple recursive calls ---
once for each $x_i$ in $s_i\p{x_1,\ldots,x_n}$.

This streamlines the operational semantics, and does originally appear in
\cite{leivant-1995}. However, the paper goes on to prove the capture of
polynomial complexities in terms of word algebras only, i.e. where $\sigma\p{s}
\leq 1$ for every symbol $s$.

\begin{definition} A \textbf{single-sorted} algebra is an algebra over the
signature $\p{\set{B}, \mathcal{T}, \mathcal{V}}$, i.e. there is just a single
base type, $B$. \end{definition}

\begin{definition} A \textbf{word} algebra is an algebra over the signature
$\p{\mathbf{B}, \set{\rightarrow}, \mathcal{V}}$, where each $v \in
\mathcal{V}$ either has the kind $v : *$ or $v : *\rightarrow *$.
\end{definition}

\begin{example} A single-sorted word algebra is an algebra over the signature
$\p{\set{B},\set{\rightarrow}, \mathcal{V}}$, where for each $v \in
\mathcal{V}$ we either have $v : B$ or $v : B\rightarrow B$. \end{example}

\begin{definition} A \textbf{Peano numbers} algebra is a single-sorted word
algebra over the signature $\p{\set{N}, \set{\rightarrow},\set{\mathtt{0} : N,
\mathtt{s} : N \rightarrow N}}$. \end{definition}


This is sufficient to deal in all finite types and string types, e.g.  the
peano numbers or binary strings. Making our way towards mixed data types and
tuples, \cite{marion-2003} generalised tiered recursion to simple algebras.

\begin{definition} A \textbf{simple algebra} is an algebra over the signature
$\p{\mathbf{B}, \set{\rightarrow}, \mathcal{C}}$, where each value constructor
$c \in \mathcal{C}$ has the type $\tau_1 \rightarrow \tau_2 \rightarrow \cdots
\rightarrow \tau_n \rightarrow \tau$, for some $n\in \mathbb{N}$, such that
$\tau$ occurs at most once amongst the $\tau_i$.  \end{definition}

If we curry our eyes a bit, we can see how this leads way to ``simple'' tuples.
Indeed, \cite{marion-2003} employs the more concise notation
$\p{\tau_1,\tau_2,\cdots,\tau_n}\rightarrow \tau$ instead.  Although this
admits a slightly more eloquent class of data types, we need more ``complex''
tuples to even model good old trees.

Worse yet, the results in \cite{leivant-1995} rely on a size function where the
size of a term is the ``height of the parse-tree'' of the term. Leading to the
plausible deniability that tiered recursion still captures $\mathtt{FPTIME}$
when defined over any free algebra\cite{caseiro-1996}.

The generalization to \emph{any} free algebra comes about in
\cite{dal-lago-et-al-2010}. Their insight is that there is no need to copy (or
recompute) a value if it appears more than once in a term --- we could just
\emph{refer} to the same value. In a programming language without destructive
assignment, this is perhaps a natural optimization technique. Hence ``trees''
are really represented as graphs.

\begin{remark} With this generalisation we move away from the conventional
notion of the \emph{length} of a term $t$ being the cumulative number of
occurrences of symbols and variables in $t$, as in first-order term
rewriting\cite{klop-vrijer-2003}. \end{remark}

\begin{example} \cite{dal-lago-et-al-2003} Consider a program $\p{\p{\p{N,T}, N
\cup T, \set{z : N, s : N \rightarrow N, \mathtt{leaf} : T, \mathtt{node} : T
\times T \rightarrow T}}, \set{f}}$, where $f$ is given by the following set of
clauses:

\begin{align}
f\p{\symb{z}}       &\leadsto \symb{leaf} \\
f\p{\symb{s}\p{w}}  &\leadsto \symb{node}\p{f\p{w}, f\p{w}}
\end{align}

\end{example}
