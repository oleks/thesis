\chapter{A Generalised Framework}

\begin{quotation}

\footnotesize\sffamily\itshape

\begin{flushright}

{\upshape PETER HILTON} \\
Christ!! What happens now?!\\
{\upshape ALAN TURING} \\
It should tell use the day's Enigma settings!\\
{\upshape HUGH ALEXANDER} \\
How long!?!

\smallbreak

\upshape

--- Graham More, The Imitaiton Game (2014)

\end{flushright}

\end{quotation}


The fundamental insight highlighted by the above quotation underpins much work
done in \cite{jones-1999, jones-2001, kristiansen-voda-2005, hofmann-2003,
kristiansen-2008}, and of course, \cite{jones-kristiansen-2009}, among others.
We draw on this insight in a fundamental and perhaps extreme way, by taking the
notion of ``successor-like function'' to roughly mean ``enumerator'', and
consider the presence and absence of both particular successor-like \emph{and}
predecessor-like functions.

Implicit characterizations of complexity classes, e.g. PTIME, often start out
by considering first-order programs on natural numbers. That is, the
computation of (mathematical) functions of type $\mathbb{N} \rightarrow
\mathbb{N}$. (For a counterexample, see \cite{hofmann-2003}.) 

In lieu of \refThm{tm-total}, such results apply even if we replace
$\mathbb{N}$ (on either side) by any countable or countably infinite set. That
is, the results still hold even if we turn to the computation of (mathematical)
functions of type $A \rightarrow B$ for some countable or countably infinite
sets $A$ and $B$.

Crucially, \refThm{tm-total} relies on \emph{mathematical} functions, not
real-world transformations. Realistically, all we get is this: assuming that we
have a representation of a value of type $A$ as a natural number, we can
compute a natural number representation of the corresponding value of type $B$,
within the characterised bounds.

This outset of ``first-order programs on natural numbers'' ignores the
real-life hurdle of dealing with \emph{data structures} rather than natural
numbers. So such an exposition is arguably impractical, or more dramatically,
ignores certain \emph{complexities} involved in dealing with data.

Instead, we formalise a notion of ``successor'' and ``predecessor'', and
present seminal results in the characterisation of PTIME in terms of this
general framework. Informally, a successor or predecessor is any operation that
may change the size of any value in the program (including time?). In practice,
essentially any conceivable operation, except perhaps stalling.

\section{Programs}

% Many high-level programming languages have data types, and these data types
% cannot be coerced.

In the interest of capturing a wide range of programming paradigms in one
discussion, we give only a general specification of what we deem a program to
be. This specification will later be specialized for more concrete purposes. We
make no claim however, that in so doing, we capture all conceivable programming
paradigms.

\begin{specification}

A \textbf{program} $P = \p{P_\mathbf{D}, P_\mathbf{F}}$ is a finite set of
\textbf{data types} $P_\mathbf{D}$, and a finite set of \textbf{functions}
$P_\mathbf{F}$, subject to the following laws:

\begin{enumerate}

\item [P-1] Every data type $A \in P_\mathbf{D}$ forms a set.

\item [P-2] Every function $f \in P_\mathbf{F}$ computes a partial function $f
: A \rightharpoonup B$, for some $A, B\in P_\mathbf{D}$. We write $f : A
\rightharpoonup B \in P_\mathbf{F}$.

\item [P-3] For every data type $A \in P_\mathbf{D}$ there is a \textbf{size
function} $\card{\cdot} : A \rightarrow \mathbb{N}$, so for every $a \in A$ we
have $\card{a} \in \mathbb{N}$.

\end{enumerate}

\end{specification}

\begin{remark} We do not require that a function $f \in P_\mathbf{F}$ can
compute a partial function $f : A \rightharpoonup B$ between \emph{any} given
data types $A, B \in P_\mathbf{D}$, merely that it is a partial function
between \emph{some} given data types $A, B \in P_\mathbf{D}$.  \end{remark}

\begin{remark} The purpose of the size function is primarily semantic, and so
it need not be in $P_\mathbf{F}$, but of course, it could be. \end{remark}

\subsection{Successors and Predecessors}

\begin{definition} A partial function $s : X \rightharpoonup Y \in
P_\mathbf{F}$ is a \textbf{successor function} for some $x \in X$ iff $s\p{x}$
is defined and $\card{x} < \card{s\p{x}}$. \end{definition}

\begin{definition} A partial function $p : X \rightharpoonup Y \in
P_\mathbf{F}$ is a \textbf{predecesor function} for some $x \in X$ iff $p\p{x}$
is defined and $\card{p\p{x}} < \card{x}$. \end{definition}

That is, a successor function produces an output \emph{strictly greater} in
size than its input, and a predecessor function produces an output
\emph{strictly lesser} in size than its input.

Note in particular, that whether a given partial function $f : X
\rightharpoonup Y \in P_\mathbf{F}$ is a successor or predecessor function, if
either, \emph{depends} on the value $x \in X$ which $f$ is applied to. If $X$
is a countably infinite data type, there may exist functions which are
successor functions, \emph{independent} of the value that they are applied to.

\begin{definition} A partial function $c : X \rightharpoonup Y \in
P_\mathbf{F}$, is a \textbf{constructor} iff $X$ is the domain of definition of
$c$, so $c : X \rightarrow Y \in P_\mathbf{F}$, and $c$ is a successor function
\emph{for all} $x \in X$. \end{definition}

On the other hand, for any data type $X$, there must exist some value $x \in
X$, for which any given function is \emph{not} a predecessor function, as
otherwise there would be a natural number less than $0$.

\subsection{Defining Functions}

For the purposes of a further discussion about programming, we adopt an
equational programming style, in line with classical recursion theory and
functional programming.

\begin{definition} For any program $P = \p{P_\mathbf{D}, P_\mathbf{F}}$, let
$P_\mathbf{F}$ form the \textbf{signature} of $P$, that is, for every $f : X
\rightharpoonup Y \in P_\mathbf{F}$, we say that $f$ is a \textbf{symbol}, and
$X \rightharpoonup Y$ is the \textbf{arity} of symbol $f$. \end{definition}

\begin{definition} A \textbf{variable} $v$ of type $Y \in P_\mathbf{D}$, is a
symbol $v$, which does not occur in $P_\mathbf{F}$. \end{definition}

\begin{definition} A \textbf{term} $t$ of type $Y$, written $t : Y$, is either
a variable $t \equiv v$, or a \textbf{compound term} $t \equiv f\p{x}$, where
$f : X \rightharpoonup Y$ and $x \in X$. \end{definition} 

\begin{definition} A \textbf{ground term} $t : Y$, is a either a variable $t
\equiv v$, or a compound term $t \equiv f\p{x}$, where $f : X \rightharpoonup
Y$ is a \emph{constructor} and $x : X$ is also a \emph{ground term}.
\end{definition}

\begin{definition} If $x$ is a ground term and $y$ is a term, then $x \leadsto
y$ is a \textbf{clause}.  \end{definition}

\begin{remark} We deviate from the classical use of the symbol $=$ (equals),
frequent in recursion theory and functional programming. We are not concerned
with any notion of ``equality'' between the left- and right-hand side of a
clause. Instead, a clause should state how the left-hand side \emph{leads to}
the right hand side. That is, we have chosen which way evaluation flows. To the
acute reader this sounds an awful lot like term-rewriting, and though it is, we
see no need to dive further into term-rewriting terminology. \end{remark}

\begin{definition} A function is given by a finite set of clauses, where no
pair of left-hand sides unify using the conventional algorithm.
\end{definition}

\subsection{Examples}

\begin{example} Peano numbers in one programming language or another. Model
theory? \end{example}

\subsection{Programming with algebraic data types}

To this end we adopt the notation of \cite{marion-2003}.

\begin{definition} A \textbf{sort} $\mathcal{S}$ is a set. \end{definition}

\begin{specification} The set of \textbf{types} $\mathcal{T}$ over a sort
$\mathcal{S}$ is  built from either the elements of $\mathcal{S}$ or
\textbf{type constructors}. \end{specification}

\begin{definition} A \textbf{vocabulary} is a set of symbols \end{definition} 

We follow the approach and notation of \cite{clote-1999}, where many of the
below results are presented using a single-sorted word algebra. We consider
generalizing the presentation to many-sorted algebras where this has already
been done by others, and otherwise.

A recursion scheme will be given over a particular algebra, where an algebra is
given over a signature.

\begin{definition} A \textbf{signature} is a triple $\Sigma = \p{\mathbf{B},
\mathbf{T}, \mathbf{C}}$, where

\begin{enumerate}

\item [$\mathbf{B}$] is a set of \textbf{base types},

\item [$\mathbf{T}$] is a set of \textbf{type constructors}, and

\item [$\mathbf{V}$] is a set of \textbf{value constructors}.

\end{enumerate}

\end{definition}

The set of base types and type constructors can be deduced from the set of
value constructors in a signature. We choose to mark these elements in a
distinct manner to make the following notions straight forward to discuss.

\begin{definition} A signature is \textbf{single-sorted} if it has exactly one
base type $B$, written $\Sigma = \p{\set{B}, -, -}$. A signature that is not
single-sorted is \textbf{many-sorted}.  \end{definition}

\begin{definition} For a set of base types $\mathbf{B}$, let $*$ denote any
base type in $\mathbf{B}$. \end{definition}

\begin{definition} A \textbf{word} signature is a signature having just the
type constructors $*$ and $* \rightarrow *$, written $\p{\mathbf{B}, \set{*, *
\rightarrow *}, \mathbf{V}}$.  \end{definition}

% \begin{example} A single-sorted word signature is a signature $\p{\set{B},
% \set{*, * \rightarrow *}, \mathbf{V}}$. That is, for each $v \in \mathbf{V}$
% we either have $v : B$ or $v : B\rightarrow B$. \end{example}

As an example, we now turn to the Peano numbers:

\begin{definition} The \textbf{Peano numbers} is the single-sorted word
signature

$$\Pi \triangleq \p{\set{N}, \set{*, * \rightarrow *}, \set{\mathtt{z} : N,
\mathtt{s} : N \rightarrow N}}.$$

\end{definition}

% \begin{definition} A \textbf{tree} algebra is an algebra having the type
% constructors $\mathbf{T} = \set{*, \mathbf{T} \rightarrow \mathbf{T}}$,
% written $\p{\mathbf{B}, \mathbf{T} = \set{*, \mathbf{T} \rightarrow
% \mathbf{T}}, \mathbf{V}}$. \end{definition}

% \begin{example} A single-sorted binary tree algebra is an algebra over the
% signature $\p{\set{B}, \set{*, * \rightarrow * \rightarrow *}, \mathbf{V}}$,
% where for each $v \in \mathbf{V}$ we either have $v : B$ or $v : B\rightarrow
% B \rightarrow B$. \end{example}

\begin{definition} The set of \textbf{$n$-ary} value constructors over $\Sigma
= \p{-, -, \mathbf{V}}$ is

$$\mathbf{V}_n \triangleq \set{v \in \mathbf{V} \st{v : *_0 \rightarrow *_1
\rightarrow \cdots \rightarrow *_n}}$$

\end{definition}

\begin{definition} The set of \textbf{constructor terms} over
$\Sigma=\p{-, -, \mathbf{V}}$ is

$$\mathcal{C}_\Sigma \triangleq \set{v\p{t_1,t_2,\ldots,t_n} \st{n \in
\mathbb{N} \wedge v \in \mathbf{V}_n \wedge t_1,t_2,\ldots,t_n \in
\mathcal{C}_\Sigma}}$$

\end{definition}

\begin{example} The set of constructor terms over the Peano numbers is

$$\mathcal{C}_\Pi = \set{\mathtt{z}, \mathtt{s}\p{\mathtt{z}},
\mathtt{s}\p{\mathtt{s}\p{\mathtt{z}}},
\mathtt{s}\p{\mathtt{s}\p{\mathtt{s}\p{\mathtt{z}}}}, \ldots}$$

\end{example}

% recursive call stack

% purely functional programs? no data other than what you put in.

% Perhaps a neat way of categorise the wealth of data structures 

% We take a programming languages approach, and in the process relax a couple
% terms from modern programming nomenclature.

% \begin{definition} A \textbf{program} is a countable set of functions.
% \end{definition}

% \begin{definition} For a given program, let the \textbf{$A$-valued functions}
% be all the functions in $P$ with codomain $A$. \end{definition}

% \begin{definition} A \textbf{data type} $A$ is given by a subset $S$ of the
% $A$-valued functions $T$, such that all functions in $T \setminus S$ return
% values constructed using functions in $S$. The functions in $S$ are called
% \textbf{value constructors}. \end{definition}

% \begin{remark} Typically, the value constructors are defined in a language
% using designated syntax. \end{remark}

% \begin{remark} For the functional programmer this is a degeneration of the
% common term ``value constructor''. For her, a data type is defined by a
% finite set of well-placed value constructors, and can only be constructed
% using these value constructors. Value constructors are $O(1)$-time and
% $O(1)$-space operations, each having an inverse ``value deconstructor'' for
% use in e.g.  pattern matching. For the imperative programmer, this is a
% degeneration of the common term ``constructor''. Here we lend the term from
% C++, where a type can have multiple constructors taking an arbitrarily long
% time and space, and having at least one common (but perhaps multiple)
% inverses. \end{remark}

% Typically a value constructor has an inverse: a value deconstructor.

% In the presence of successors without pairing predecessors, we must do away
% with such notions.

% \begin{notation} Let $P_\mathbf{D}$ denote the data types induced by the
% functions of program $P$, that is, the set of all codomains in $P$.
% \end{notation}

% \begin{definition} A value constructor is a \textbf{successor} \end{definition} 

% Cobham: Let a function $f$ be defined by a range of base case clauses, and a
% range of deconstructor clauses. The recursive clauses may use various
% predecessor operations.

% Size is always given relative to a particular predecessor operation.

% Consider the predecessor (among those used in the function) that decreases
% the value by the least amount. Assume that the predecessors are always
% applicable to a given value, and that by iterated use of the predecessor, we
% reach a base case. A function is defined by bounded primitive recursion, if
% the number of applications of this most-basic predecessor is no more than as
% given by the given bound.

% Also, better suited as an introduction as it doesn't really say anything
% about the choice of polynomial time from the get go. That's more a
% coincidence, the following definitions will hold regardless. Perhaps it is
% worth it calling this a Part I: Introduction and Background

% Fragment of a logic means that we are restricted to a subset of the syntax,
% but retain the same semantics. If we choose the combination of words -
% subsystem of system, we are perhaps a little more free, but in either case,
% we probably have to further specify what we keep and what we take out.

% for the purposes of this thesis, we may regard logical systems equivalent to
% programming languages - draw inspiration from Types of Crash Prevention - it
% has a nice, human readable introduction to the whole thing.

% The correspondence between an ICC system and a complexity class is
% extensional, i.e. the class of function (or problems) representable in the
% system equals the complexity class.

% by this point we should have more precisely defined the notions of function
% and problem.

% The systems that we will consider here will be subsystems of a larger base
% system, in which other functions, besides those in the complexity class of
% the system can be represented.

% \begin{definition} \textit{Intensional soundness}

% Any program representable in the ICC system, is within the given complexity
% class. Proven by showing equivalence of the class of problems to a complexity
% class.

% \end{definition}

% \begin{definition} \textit{Intensional completeness}

% All functions in the complexity class are representable in the ICC system.

% \end{definition}

% \begin{definition} \textit{Extensional soundness}

% All functions in the complexity class are representable in the base system.

% \end{definition}




% we want natural numbers, but we don't want the successor and predecessor
% operations to always be blindly permitted.

% primitive recursion is really defined in terms of a predecessor operation, at
% least if we assume the successor operation to merely be something that can
% construct a "bigger" value. otherwise, it is undefined how in primitive
% recursion we first compute the lower-order value.

% we run into the same problem with s_1 and s_0! Does it at all make sense to
% define a successor without a predecessor? At least not if we want to be able
% to define functions recursively! Any time we define a function computing a
% bigger value having a lower value, we need to specify it in terms of
% predecessors!

% decrement on binary notation is a polynomial-time operation if we only assume
% s_0 and s_1 (or rather, their predecessors). We can assume dec to be a
% unit-cost operation, and that is where the problems of primitive recursion
% creep up.

% although we can always convert a unique string to a unique natural number, it
% may take a long time to do so. Furthermore, the successor operation is a
% polynomial time oper

% Any more definite attempt at a definition of the natural numbers seems to
% arrive at a philosophical impasse.

% The natural numbers have never-the-less shown to be of prime importance due to
% the concept of recursive enumerability.

% Use sets rather than ``language'' to avoid mixing up inputs/outputs of
% machines and the programming languages that we design. Languages refer
% specifically to sets of strings.

% \begin{definition}

% Define enumerator.

% \end{definition}

% \begin{definition}

% Define recursively enumerable sets.

% \end{definition}

% Dealing with natural numbers is bad as we would quickly regard the successor
% (and predecessor) operations to be paired and unit-cost operations.

% You cannot take the successor operation to be a unit-cost operation, nor can
% you assume that it is paired with a predecessor.

% we are dealing with words? basic data elements on any machine model?

% an element of state in a machine model. it has an initial state, and some
% succeeding states - preceding states don't really make sense.

% a successor does not lead you to another state

% at least size is not defined in terms of the number states that you've passed.

% for all classical machine models, the class of polynomial-time computable
% functions is equivalent.

% the values of any practical data structure can be recursively enumerated; so
% although the fact that a set is recursively enumerable opens a pandoras box,
% only a practical subset of it will be used in practice.

% this is an explanation of godel encoding.

% strings or states of a data structure?

% due to godel encoding, it is okay to deal in natural numbers, provided that
% we do not (necessarily) permit a unit-cost unique successor operation.

% so we're actually redefining natural numbers again? - yes.

% the natural numbers that you do have in any programming language, depend on
% the constructs available in your language. indeed, due to godel encoding, any
% practical programming languages comes with manifolds of ways of defining
% natural numbers.

% is all this because we want to deal in the natural numbers?

% In practice, we often deal with recursive sets, as by the Church-Turing thesis,
% these are  the computable functions. This makes it attractive to deal merely in
% the natural numbers as introduced above. Although this is perhaps a useful
% definition of natural numbers, the pairing of a recursively enumerable set with
% the natural numbers can be a rather complex procedure.

% We intend to model this complexity by permitting a more elaborate definition of
% natural numbers:

% \begin{notion}

% A \textbf{value} is either $0$, a successor, or a predecessor of a value. A
% successor and predecessor need not take $O(1)$ to compute.

% \end{notion}

% The notions of successor and predecessor are defined below.

% It is intentional that we do not denote the syntax of $x$.

% what is a data type?

% \begin{definition}

% The \textbf{notation} of a particular data type is a specification of how
% values of that data type may be constructed and deconstructed.

% \end{definition}

% \section{Successors}

% multiple possible successor and predecessor functions.

% TODO: define <, =, \vdash

% to define <, we need to define size; so < is defined for some measure of size.

% The rules with < occurring so early is not a problem as it is a statement
% that these rules must hold, either as part of the definition of a successor
% s, or as an admissible rule.

% \begin{definition}

% A function $s$ is a \textbf{successor} function iff

% \begin{align}
% &\vdash 0 < s(x) \\
% x < y &\vdash s(x) < s(y)
% \end{align}

% \end{definition}

% Successor functions play a key role in the remainder of the thesis, as they
% form the basis of our measure of the size of a value.

% \begin{definition}

% The \textbf{size} of a value $x$, the minimum number of successor
% applications necessary to construct $x$.

% \end{definition}

% For instance, a language $\mathcal{B}$ that uses binary notation may have two
% successors, one that doubles the value, $s_0$, and another that doubles the
% value and adds a one, $s_1$. The value $5$ is then represented in
% $\mathcal{B}$ as $s_1\p{s_0\p{s_1\p{0}}}$, and so the size of the value $5$
% in $\mathcal{B}$ is $3$. Analytically, the size of a value $x$ in
% $\mathcal{B}$ is $\ceil{\log_2\p{x}}$.

% \section{Predecessors}

% \begin{definition}

% A function $p$ is a \textbf{predecessor} function iff

% no cycles?

% \begin{align}
% &\vdash p\p{0} = 0 \\
% &\vdash p\p{x} < x & x \neq 0 \\
% x < y &\vdash p\p{x} < p\p{y}
% \end{align}

% \end{definition}

For instance, the language $\mathcal{B}$ above, may have a predecessor function
$dec$ (read: decrement) defined as follows:

\begin{align}
dec'\p{x}=y, trim\p{y}=z &\vdash dec\p{x} = z \\
dec'\p{x} = y &\vdash dec'\p{s_1\p{x}} = s_1\p{y} \\
dec''\p{x} = y &\vdash dec'\p{s_1\p{x}} = s_0\p{y} \\
dec'\p{x} = y &\vdash dec'\p{s_0\p{x}} = s_0\p{y} \\
dec''\p{x} = y &\vdash dec''\p{s_0\p{x}} = s_1\p{y} \\
&\vdash dec''\p{s_0\p{0}} = s_1\p{0} \\
&\vdash dec'\p{s_1\p{0}} = s_0\p{0} \\
&\vdash dec'\p{0} = 0 \\
trim\p{y} = z &\vdash trim\p{s_0\p{y}} = z \\
&\vdash trim\p{s_1\p{y}} = s_1\p{y} \\
&\vdash trim\p{0} = 0
\end{align}

For instance, $s_1\p{s_0\p{s_1\p{0}}}$, i.e. 5, decremented, is
$s_1\p{s_0\p{s_0\p{0}}}$, i.e. 4:

$$
\judgement{
  \judgement{
    \judgement{
      \axiom{
        dec'\p{s_1\p{0}} = s_0\p{0}
      }
    }{
      dec'\p{s_0\p{s_1\p{0}}} = s_0\p{s_0\p{0}}
    }
  }{
    dec'\p{s_1\p{s_0\p{s_1\p{0}}}} = s_1\p{s_0\p{s_0\p{0}}}
  }
  \quad
  \axiom{
    trim\p{s_1\p{s_0\p{s_0\p{0}}}} = s_1\p{s_0\p{s_0\p{0}}}
  }
}{
  dec\p{s_1\p{s_0\p{s_1\p{0}}}} = s_1\p{s_0\p{s_0\p{0}}}
}
$$

and $s_1\p{s_0\p{s_0\p{0}}}$, i.e. 4, decremented, is $s_1\p{s_1\p{0}}$, i.e. $3$:

$$
\judgement{
  \judgement{
    \judgement{
      \axiom{
        dec''\p{s_0\p{0}} = s_1\p{0}
      }
    }{
      dec''\p{s_0\p{s_0\p{0}}} = s_1\p{s_1\p{0}}
    }
  }{
    dec'\p{s_1\p{s_0\p{s_0\p{0}}}} = s_0\p{s_1\p{s_1\p{0}}}
  }
  \quad
  \judgement{
    \axiom{
      trim\p{s_1\p{s_1\p{0}}} = s_1\p{s_1\p{0}}
    }
  }{
    trim\p{s_0\p{s_1\p{s_1\p{0}}}} = s_1\p{s_1\p{0}}
  }
}{
  dec\p{s_1\p{s_0\p{s_0\p{0}}}} = s_1\p{s_1\p{0}}
}
$$

\section{Successors and predecessors}

\begin{definition}

A predecessor function $p$ is \textbf{paired} with a successor function $s$ iff

\begin{align}
\vdash p\p{s\p{x}} &= x
\end{align}

\end{definition}

It is not a requirement for a predecessor to be paired with a successor. A
language may have predecessors more powerful than any of its successors.

\begin{remark}

A successor or predecessor function may or may not take $O(1)$ time. This
depends on the machine model.

\end{remark}

\section{Primitive recursion}

\begin{definition}

A function is defined by \textbf{primitive recursion} (PR), from PR functions
$g$ and $h$, and a PR predecessor function $p$, iff

\begin{align}
f\p{0,\vect{y}} &= g\p{\vect{y}} \\
f\p{x, \vect{y}} &= h\p{x, \vect{y}, f\p{p\p{x},\vect{y}}}
\end{align}

\end{definition}

The running time of a PR function $f$, depends on the running times of $g$,
$h$, and $p$, as well as how quickly $p$ decreases $x$ to $0$.

Mere primitive recursion is not confined to polytime computation.

\begin{theorem}
PR is not confined to polytime computation.
\end{theorem}

\begin{proof} Assume that $g$, $h$, and $p$ are all computed in $O(1)$ time.
Consider the language $\mathcal{B}$ above, having just the successors $s_0$ and
$s_1$. Let $p$ be the function $dec$ above.\end{proof}

The trouble is that $p$ does not decrease $x$ fast enough throughout the
recursion --- the number of times $p$ has to be applied to $x$ to reach $0$, is
superpolynomial in $x$.

One attempt to mitigate for this could be to have $p$ decrease the value
faster. For instance, sticking with the language $\mathcal{B}$ above, we could
let $p$ be a function that halves $x$. This would pair $p$ with $s_0$.

\begin{theorem}

Primitive recursion defined with a predecessor function which is paired with a
successor function is not confined to polytime computation.

\end{theorem}

\begin{proof} Let a predecessor function $p$ be paired with a successor
function $s$.

Consider a function $dbl$, which doubles its input:

\begin{align}
dbl\p{0} &= 0 \\
dbl\p{x} &= s\p{s\p{dbl\p{p\p{x}}}}
\end{align}

Consider furthermore a function $iterdbl$, which calls $dbl$ iteratively, the
same number of times as the length of its input:

\begin{align}
iterdbl\p{0} &= s\p{0} \\
iterdbl\p{x} &= dbl\p{iterdbl\p{p\p{x}}}
\end{align}

Calling $iterdbl$ with a string of length $n$, we obtain a string of length
$2^n$ due to iterated doubling of the input. It follows that $iterdbl$ does a
superpolynomial amount of work in the size of its input.\end{proof}

\begin{example}
We illustrate the above proof with an example:
\begin{align}
\underline{iterdbl\p{s\p{s\p{0}}}}
  &\leadsto dbl\p{iterdbl\p{\underline{p\p{s\p{s\p{0}}}}}} \\
  &\leadsto dbl\p{\underline{iterdbl\p{s\p{0}}}} \\
  &\leadsto dbl\p{dbl\p{iterdbl\p{\underline{p\p{s\p{0}}}}}} \\
  &\leadsto dbl\p{dbl\p{\underline{iterdbl\p{0}}}} \\
  &\leadsto dbl\p{\underline{dbl\p{s\p{0}}}} \\
  &\leadsto dbl\p{s\p{s\p{dbl\p{\underline{p\p{s\p{0}}}}}}} \\
  &\leadsto dbl\p{s\p{s\p{\underline{dbl\p{0}}}}} \\
  &\leadsto \underline{dbl\p{s\p{s\p{0}}}} \\
  &\leadsto s\p{s\p{dbl\p{\underline{p\p{s\p{s\p{0}}}}}}} \\
  &\leadsto s\p{s\p{\underline{dbl\p{s\p{0}}}}} \\
  &\leadsto s\p{s\p{s\p{s\p{dbl\p{\underline{p\p{s\p{0}}}}}}}} \\
  &\leadsto s\p{s\p{s\p{s\p{\underline{dbl\p{0}}}}}} \\
  &\leadsto s\p{s\p{s\p{s\p{0}}}} \\
\end{align}
\end{example}

In the proof above, we fall prey to the admission for the function $h$ in a
primitive recursive function $f$ to generate value larger in size than its
input.

We could try to mitigate for this, but this would be getting ahead of
ourselves. Bounded primitive recursion bounds the primitive recursion in a more
general way, by bounding $f$ itself.

recursion on notation can simulate primitive recursion in the absence of any
size bounds\cite{bellantoni-phd-1992}.

From this perspective we can fit Cobham's characterization into the pattern of
most (if not all) definitions of complexity classes: one defines a general
model of computation and then cuts oôhe computation when certain resource
bounds are exceeded.

An operator is a mapping from ground terms over a signature to ground terms
over a signature. An algebra is a signature coupled with a set of operators.

The semantics of composition is given by

$$\sem{\mathtt{COMP}\p{h,g_0,g_1,\ldots,g_i}} = f$$

\begin{align*}
%
f\p{\vect{x}} = \sem{h}\p{\sem{g_0}\p{\vect{x}}, \sem{g_1}\p{\vect{x}}, \ldots,
\sem{g_i}\p{\vect{x}}}
%
\end{align*}

\subsubsection{Projection}

\label{sec:generalised-projection}

\begin{definition} Let $\sem{\mathtt{PROJ}_j} : S_1 \times \cdots \times S_n
\rightarrow T$ be a \textbf{generalised projection} operator, having the
denotational semantics $\sem{\mathtt{PROJ}_j\p{e_1,\ldots,e_n}} = \sem{e_j}$,
for some $j,n \in \mathbb{N}$, such that $1 \leq j \leq n$.  \end{definition}

\begin{remark} $\sem{\mathtt{PROJ}}$ is parametric in the type $S_1,\ldots,S_n$
and $T$. In simpler cases, we might let $S_1 = \cdots S_n = S$, and have
$\sem{\mathtt{PROJ}}$ parametrized in $S$ and $T$; perhaps even let $S = T$.
\end{remark}

\begin{definition} Let $\mathtt{proj}_j$ be the operational interpretation of
$\sem{\mathtt{PROJ}_j}$, having in particular, the following
semantics:\begin{align*}
%
\axiom{ \mathtt{proj}_j\p{x_1,\ldots,x_{n}} \goesto x_j }
%
\end{align*}

\end{definition}

\subsubsection{Composition}

\label{sec:generalised-composition}

\begin{definition} Let $\sem{\mathtt{COMP}}$ be a \textbf{composition}
operator, having the denotational semantics
$\sem{\mathtt{COMP}\p{h,g_1,\ldots,g_m}} = f$, for some $m,n\in \mathbb{N}$,
where \begin{align*}
%
\sem{h} & : T_1 \times \cdots \times T_m \rightarrow U \\
%
\sem{g_1} & : S_1 \times \cdots \times S_n \rightarrow T_1 \\
%
& \ldots \\
%
\sem{g_m} & : S_1 \times \cdots \times S_n \rightarrow T_m \\
%
f & : S_1 \times \cdots \times S_n \rightarrow U
%
\end{align*} and $f$ is given by the following recursion scheme, where
$\vect{x} \triangleq x_1,\ldots,x_n$:\begin{align*}
%
f\p{\vect{x}} = \sem{h}\p{\sem{g_1}\p{\vect{x}}, \ldots,
\sem{g_m}\p{\vect{x}}}.
%
\end{align*}

\end{definition}

\begin{remark} $\sem{\mathtt{COMP}}$ is parametric in the types
$S_1,\ldots,S_n$, $T_1,\ldots,T_m$, and $U$. In simpler cases, we might let
$S_1 = \cdots = S_n = S$ and let $T_1 = \cdots T_m = T$, and have
$\sem{\mathtt{COMP}}$ parametrized in $S$, $T$, and $U$; perhaps even let
$S=T=U$. \end{remark}

\begin{definition} Let $\mathtt{comp}$ be the operational interpretation of
$\sem{\mathtt{COMP}}$, having in particular, the following operational
semantics, where $\vect{x} \triangleq x_1, \ldots, x_n$:\begin{align*}
%
\judgement{
%
  g_1\p{\vect{x}} \goesto y_1 \quad \cdots \quad g_m\p{\vect{x}} \goesto y_m
\quad h\p{y_1,\ldots, y_m} \goesto z
%
}{
%
  \mathtt{comp}\p{h,g_1,\ldots,g_m}\p{\vect{x}} \goesto z
%
}
%
\end{align*}

\end{definition}

\subsubsection{Explicit Transformation}

\begin{definition} \cite[p. 21]{smullyan-1961} Let $\sem{\mathtt{EXTR}}$ be an
\textbf{explicit transformation} operator, having the semantics
$\sem{\mathtt{EXTR}\p{h,e_1,\ldots,e_m}} = f$, where $\sem{h} : S^m \rightarrow
T$ and $\sem{e_1}, \ldots, \sem{e_m} : S + \set{1,\ldots,n}$, for some $m,n \in
\mathbb{N}$. So $\sem{e_i}$ is either $\mathtt{inl} \; s$ for some $s \in S$,
or $\mathtt{inr} \; j$ for some $j \in \set{1,\ldots, n}$.

The function $f : S^n \rightarrow S$ itself is given by the following recursion
scheme:\begin{alignat*}{3}
%
f\p{x_1,\ldots,x_n} &= \sem{h}\p{g\p{\sem{e_1}},\ldots,g\p{\sem{e_m}}} \\
%
& & \text{where} \; g\p{\mathtt{inl} \; s} &= s \\
%
& & g\p{\mathtt{inr} \; j} &= x_j
%
\end{alignat*}

\end{definition}

\begin{remark} $\sem{\mathtt{EXTR}}$ is parametric in the types $S$ and $T$.
\end{remark}

$\sem{\mathtt{EXTR}}$ is a generalised substitution operator. Using
$\sem{\mathtt{EXTR}}$, we can obtain not only $\sem{\mathtt{PROJ}}$ and
$\sem{\mathtt{COMP}}$, but also identity and permutation functions.  On the
other hand, we can obtain $\sem{\mathtt{EXTR}}$ from $\sem{\mathtt{PROJ}}$,
$\sem{\mathtt{COMP}}$, and the {\bfseries \color{red} initial operators}
\cite{rose-1984}. The theorem follows:

\begin{theorem}\label{thm:extr-comp-proj} $\p{\Sigma,\set{\mathtt{EXTR}} \cup
O} = \p{\Sigma,\set{\mathtt{COMP}, \mathtt{PROJ}} \cup O}$. \end{theorem}

\begin{proof} \todo{Missing} \end{proof}

\subsubsection{Primitive Recursion}

Motivated by \cite{leivant-1995}, we introduce a generalised form of primitive
recursion, also referred to as ``structural recursion''\cite{hofmann-2002}:

\index{recursion!primitive}
\index{recursion!structural|see {primitive recursion}}

\begin{definition} Let $\mathtt{PREC}$ be a \textbf{generalised primitive
recursion} operator over an inductively defined data type $S$, where
$s_1,\ldots,s_n$ are the value constructors of $S$, with the denotational
semantics $\sem{\mathtt{PREC}\p{g_{s_1},\ldots,g_{s_n}}} = f$ with the
type\begin{align*}
%
f : S \times T_1 \times \cdots \times T_k \rightarrow U
%
\end{align*}and defined as\begin{align*}
%
f\p{s_i\p{x_1,\ldots,x_m},y_1,\ldots,y_k} = \sem{g_{s_i}}\p{\vect{x}, \vect{y},
\vect{f}} & \quad \quad \p{\text{for all $i \in \set{1, \ldots, n}$}}
%
\end{align*} where

\begin{enumerate}

\item $\vect{x}$ is a subsequence of $x_1,\ldots,x_m$;

\item $\vect{y}$ is a subsequence of $y_1,\ldots,y_k$; and

\item $\vect{f}$ is a subsequence of $f\p{x_1, \vect{y}},\ldots, f\p{x_m,
\vect{y}}$.

\end{enumerate}

\end{definition}

In order to talk about primitive recursion, we also introduce the following
terminology:

\begin{itemize}

\item We say that the $s_i\p{x_1,\ldots,x_m}$ on the left hand side is the
\textbf{recurrence argument}, and $\vect{f}$ on the right hand side are
\textbf{critical arguments}.

\index{recurrence!argument}
\index{arguments!critical}

\item We als call the $g_{s_i}$'s above \textbf{recurrence functions}, where
$g_{s_i}$ is a \textbf{base function} if it takes no critical arguments, and is
a \textbf{step function} otherwise.

\index{recurrence!functions}
\index{functions!base|see {recurrence functions}}
\index{functions!step|see {recurrence functions}}

\item We say that the recurrence argument \textbf{drives} the recursion.

\index{recurrence!(drives a)}

\end{itemize}

\todo{All primitive recursions over inductive data types bottom out.}
