\chapter{Bounded Primitive Recursion on Notation}

\begin{quotation}

\footnotesize\sffamily\itshape

\begin{flushright}

If one oversteps the bounds of moderation, \\
the greatest pleasures cease to please.

\smallbreak

\upshape

--- Epictetus, \emph{late Stoic philosopher} (55--135 A.D.)

\end{flushright}

\end{quotation}

Perhaps the most obvious way to characterise \FPTIME{}, is to explicitly bound
the function defined to take time at most polynomial in the size of the input,
without further reflections on how such a bound may be instated. Even at this
level, numerous questions arise:

\begin{enumerate}[label=(\arabic*)]

\item What is a good measure of size?

\item What are the necessary and sufficient means to state a bound like
``polynomial in the size of the input''?

\item What does a bounded function definition look like?

\end{enumerate}  

Such a general characterisation, taking a shot at the questions above, appeared
in the seminal conference paper \cite{cobham-1965}. Consequently, this paper
helped to establish computational complexity as a field\cite{clote-1999}.

The original characterisation was given in terms of a single-sorted word
signature, in particular, the natural numbers in decimal notation. The
following characterisation equivalent up to choice of words and notation,
except:

\begin{enumerate}[label=(\arabic*)]

\item we give the denotational rather than recursion-theoretic semantics of the
operators in a step towards a more operational characterisation.

\end{enumerate}

\def\smashf{\ensuremath{\mathtt{\#}}}
\def\cdotnot{\ensuremath{\cdot}}

\begin{definition} \cite{cobham-1965}

\begin{enumerate}[label=(\arabic*)]

\item Let $\mathtt{N}$ be a base type of natural numbers with $\sem{\mathtt{N}}
= \mathbb{N}$.

\item Let $\mathtt{0} : \mathtt{N}$ be a \textbf{constant} operator with
$\sem{\mathtt{0}} = 0$.

\item Let $\mathtt{s_0}, \ldots, \mathtt{s_9} : \mathtt{N} \rightarrow
\mathtt{N}$ be the $10$ \textbf{successor} operators, having the following
semantics:\begin{align*}
%
\sem{\mathtt{s_0}}\p{x} &= 10x + 0 \\
%
&\ldots \\
%
\sem{\mathtt{s_9}}\p{x} &= 10x + 9
%
\end{align*} In particular, $\sem{\mathtt{s_0}}\p{0} = 0$,
$\sem{\mathtt{s_0}}\p{1} = 10$, $\sem{\mathtt{s_5}}\p{0} = 5$,
$\sem{\mathtt{s_5}}\p{1} = 15$, etc. 

\item Let $\mathtt{COMP}$ be a \textbf{composition} operator, defined as in
\refSec{generalised-composition}, parametrised by the types $S=T=U=\mathtt{N}$.
That is, any occurrence of $\mathtt{COMP}$ has, for  some $m,n\in\mathbb{N}$,
the following type:\begin{align*}
%
\p{\mathtt{N}^m \rightarrow \mathtt{N}} \rightarrow \p{\mathtt{N}^n \rightarrow
\mathtt{N}}^m \rightarrow \mathtt{N}^n \rightarrow \mathtt{N}
%
\end{align*} In particular, both occurrences of $\mathtt{COMP}$ in
$\mathtt{COMP}\p{\mathtt{s_4}, \mathtt{COMP}\p{\mathtt{s_2}, \mathtt{z}}}$,
have the type $\p{\mathtt{N} \rightarrow \mathtt{N}} \rightarrow \mathtt{N}
\rightarrow \mathtt{N}$. Recall, that $\mathtt{COMP}\p{f,\vect{x}}$ is also
written as $f\p{\vect{x}}$, and so $\mathtt{COMP}\p{\mathtt{s_4},
\mathtt{COMP}\p{\mathtt{s_2}, \mathtt{z}}}$ is more concisely written as
$\mathtt{s_4}\p{\mathtt{s_2}\p{\mathtt{z}}}$.

\item Let $\mathtt{EXTR}$ be an \textbf{explicit transformation} operator,
defined as in \refSec{explicit-transformation}, parametrised by the types
$S=T=\mathtt{N}$.

\item Let $\sem{\mathtt{BPRN}}$ be a \textbf{bounded primitive recursion on
notation} operator, having the semantics
$\sem{\mathtt{BPRN}\p{g,h_0,\ldots,h_9,k}} = f$, for some $\sem{g} :
\mathbb{N}^n \rightarrow \mathbb{N}$, $\sem{h_0}, \ldots, \sem{h_9} :
\mathbb{N} \times \mathbb{N}^n \times \mathbb{N} \rightarrow \mathbb{N}$ and
$\sem{k} : \mathbb{N} \times \mathbb{N}^n \rightarrow \mathbb{N}$, where $f$
has type $\mathbb{N} \times \mathbb{N}^n \rightarrow \mathbb{N}$, and $f$ is
given by the following recursion scheme, where $\vect{y} =
y_1,\ldots,y_n$:\begin{align*}
%
f\p{\mathtt{s_0}\p{0}, \vect{y}} &= \sem{g}\p{\vect{y}} \\
%
f\p{\mathtt{s_0}\p{x}, \vect{y}} &= \sem{h_0}\p{x, \vect{y}, f\p{x, \vect{y}}}
& \text{if $x\neq 0$}\\
%
f\p{\mathtt{s_1}\p{x}, \vect{y}} &= \sem{h_1}\p{x, \vect{y}, f\p{x, \vect{y}}}
\\
%
&\cdots \\
%
f\p{\mathtt{s_9}\p{x}, \vect{y}} &= \sem{h_9}\p{x, \vect{y}, f\p{x, \vect{y}}}
\\
%
f\p{x, \vect{y}} &\leq \sem{k}\p{x, \vect{y}}
%
\end{align*} where $f$ itself has the following type:\begin{align*}
%
f : \mathbb{N} \rightarrow \mathbb{N}^n \rightarrow \mathbb{N}.
%
\end{align*}

Notably, given the denotational semantics of
$\mathtt{s_0},\ldots,\mathtt{s_9}$, any $n \in \mathbb{N}$ can be represented
as \begin{align*}
%
x_1\p{\cdots x_{\card{n}}\p{0} \cdots} & \quad \quad \text{where $x_i \in
\set{\mathtt{s_0}, \ldots, \mathtt{s_9}}$}
%
\end{align*}


\todo{this is to be interpreted as}

\item Let $\card{\mathrel{}\cdotnot\mathrel{}} : \mathbb{N} \rightarrow
\mathbb{N}$ be a \textbf{size function} such that $\card{x} =
\ceil{\log_{10}\p{x+1}}$.

\item Let $\sem{\cdotnot \; \smashf{} \; \cdotnot} : \mathbb{N} \times
\mathbb{N} \rightarrow \mathbb{N}$ be a \textbf{smash} operator, having the
semantics $\sem{x \smashf{} y} = \sem{x}^\card{\sem{y}}$.

\item Let $\mathcal{L} = \seq{s_i, \smashf{}, \mathtt{EXTR}, \mathtt{COMP},
\mathtt{BPRN}}$. That is, $\mathcal{L}$ is the smallest class of functions
containing the generalised successor and smash functions, \todo{closed under}
the operations of explicit transformation, composition, and bounded primitive
recursion on notation.

\end{enumerate}

\end{definition}

The detailed proofs that $\mathcal{L}$ is \FPTIME-sound and \FPTIME-complete,
appeared later in e.g.  \cite{rose-1984,clote-1999} using binary notation, and
in \cite{tourlakis-1984} using $n$-ary notation, for any $n>1$.  The proofs are
sketched below.

Firstly, going from decimal to binary, in general, to $n$-ary notation, the
size function must be adequately modified:

\begin{definition} \label{def:n-ary-notation} For $n$-ary notation, for any
$n>1$, and $x\geq 1$, let the \textbf{size function} be given as $\card{x} =
\ceil{\log_n\p{x+1}}$. As for the special cases, for $n=1$, let $\card{x}=x$,
and for $n=0$, let $\card{x}=0$. As a matter of covenience, note the
following:\begin{align*}
%
\ceil{\log_n\p{x+1}} \leq \log_n\p{x+1} + 1 \leq \log_n\p{x} + 2 =
\bigOh{\log_n\p{x}}.
%
\end{align*} \index{size function!$n$-ary notation} \end{definition}

Two other prominent changes are often done to the definition of $\mathcal{L}$
(see e.g. \cite{rose-1984, buss-phd-1985-6, bellantoni-cook-1992, clote-1999},
but somewhat surprisingly, not \cite{tourlakis-1984}):

\begin{enumerate}[label=(\arabic*)]

\item $\sem{\mathtt{EXTR}}$ is replaced by $\sem{\mathtt{PROJ}}$. In the
presence of $\sem{\mathtt{COMP}}$, by \refThm{extr-comp-proj}, this is
admissible.

\item For binary notation, the smash operator, is instead given with the
semantics $\sem{x \smashf{} y} = 2^{\card{\sem{x}} \cdot \card{\sem{y}}}$. Of
course, for binary notation, we have\begin{align*}
%
x^\card{y} = \p{2^{\log_2{x}}}^\card{y} \leq
\p{2^{\ceil{\log_2\p{x+1}}}}^\card{y} = \p{2^\card{x}}^\card{y} = 2^{\card{x}
\cdot \card{y}}.
%
\end{align*}So this too, is admissible\footnote{The diligent reader might find
it peculiar that \FPTIME{} is stable under small perturbations of ``smash'';
contrast the proofs in \cite{tourlakis-1984} and \cite{clote-1999}.}.

\end{enumerate}

The latter choice of smash operator is perhaps best motivated by the fact that,
for binary notation, for any \todo{monotone} polynomial $p$, we
have\begin{align*}
%
\log_2\p{2^{p\p{\card{x}}}} = p\p{\card{x}}
%
\quad \quad \text{and so} \quad \quad
%
\card{2^{p\p{\card{x}}}} = \bigOh{p\p{\card{x}}}
%
\end{align*} This admits perhaps a more natural construction of a value
\emph{polynomial in size} in the size of the input, i.e. a value of size
$p\p{\card{x}}$ for a given value of size $\card{x}$, for any monotone
polynomial $p$. In particular, with this notion of smash,\begin{align*}
%
\sem{\underbrace{x \smashf{} \cdots \smashf{} x}_{\text{$k$ times}}} =
2^{\card{\sem{x}}^k}
%
\quad\quad \text{and so} \quad\quad
%
\card{\sem{\underbrace{x \smashf{} \cdots \smashf{} x}_{\text{$k$ times}}}} =
\bigOh{\card{\sem{x}}^k}.
%
\end{align*} At the same time, for $n$-ary notation, where $n>1$, we
have\begin{align*}
%
\card{x^\card{y}} &= \bigOh{\log_n\p{x^\bigOh{\log_n\p{y}}}} \\
%
                  &= \bigOh{\log_n\p{x^{\log_n\p{y}}}} \\
%
                  &= \bigOh{\log_n\p{x} \cdot \log_n\p{y}} \\
%
                  &= \bigOh{\card{x} \cdot \card{y}}
%
\end{align*} So the two smash functions $x^\card{y}$ and $2^{\card{x} \cdot
\card{y}}$ yield values equivalent in size, up to (fairly small) constant
factors (see also \refDef{n-ary-notation}). $x^\card{y}$ however, is easier to
generalise to $n$-ary notation.

\begin{lemma} \label{lem:l-fptime-sound} $\mathcal{L}$ is \FPTIME{}-sound.
\end{lemma}

\begin{proof} (Sketch, see also \cite{rose-1984, tourlakis-1984, clote-1999}.)
Proof by simulation of $\mathcal{L}$ on an \FPTIME{} TM.  \todo{Rest of sketch
missing.} \end{proof}

\begin{lemma} \label{lem:l-fptime-complete} $\mathcal{L}$ is
\FPTIME{}-complete. \end{lemma}

\begin{proof} (Sketch, see also \cite{rose-1984, tourlakis-1984, clote-1999}.)
Proof by simulation of an \FPTIME{} TM in $\mathcal{L}$. Performing primitive
recursion on notation on $u$, we can simulate $p\p{\card{t}}$-time TM, by
simulating the transition function of the TM with the step function of the
primitive recursion.  \todo{Rest of sketch missing.} \end{proof}

\begin{theorem} \label{thm:l-captures-fptime} $\mathcal{L}$ captures \FPTIME{}.
\end{theorem}

\begin{proof} Follows directly from \refLem{l-fptime-sound} and
\refLem{l-fptime-complete}. \end{proof}

% Key to the \FPTIME{}-completeness of the characterisation in
% \cite{cobham-1965} is to permit the construction of a term polynomial in size
% in the size of the input term.  That is, given a term $t \in
% \mathcal{T}\p{\Psi}$, to construct a term $u \in \mathcal{T}\p{\Psi}$, such
% that $p\p{\card{t}} \leq \card{u}$, for some monotone polynomial

% $$p\p{x}=a_0 \cdot x^0 + a_1 \cdot x^1 + \cdots + a_k \cdot x^k,$$

% where $a_0$, $a_1$, \ldots, $a_k$ are non-negative constants. This permits to
% simulate a $p\p{\card{t}}$-time TM using primitive recursion on notation over
% $u$, by simulating a TM transition at every step of the recursion.

\begin{remark} The technique of simulating a bounded TM with primitive
recursion appeared perhaps for the first time in \cite[\textsection
10.2.1/176]{minsky-1967}.  \end{remark}

% \begin{definition} Let $\mathcal{T}_\downarrow\p{\Psi}$ denote the
% \textbf{ground terms} over $\Psi$. \end{definition}

% \begin{definition} Let $\mathcal{T}_\downarrow\p{\Psi}^{\leq i}$ denote the
%ground terms over $\Psi$ of size at most $i$, m.f.

% $$\mathcal{T}_\downarrow\p{\Psi}^{\leq i} = \set{ t \in
% \mathcal{T}_\downarrow\p{\Psi} \st{ \card{t} \leq i }}.$$

% \end{definition}

% We consider the total number of ground terms over $\Psi$ of size at most $i$,
% for any given $i \in \mathbb{N}$, written
% $\card{\mathcal{T}_\downarrow\p{\Psi}^{\leq i}}$.

% Intuitively, the ground terms $\mathcal{T}_\downarrow\p{\Psi}$ of a
% single-sorted word signature $\Psi$ with $m$ nullary symbols and $\p{n-m}$
% unary symbols form the nodes in a forest $\mathcal{F}_\downarrow\p{\Psi}$ of
% $m$ complete $\p{n-m}$-ary trees of unbounded height.

% The number of terms of size $i\in\mathbb{N}$ is the number of nodes of tree
% depth at most $i-1$ in $\mathcal{F}_\downarrow\p{\Psi}$. This is also the total
% number of nodes in $\mathcal{F}_\downarrow\p{\Psi}$, when the trees are bounded
% to height $i-1$, written $\mathcal{F}_\downarrow\p{\Psi}^{< i}$.

% \begin{theorem} For a single-sorted word signature $\Psi$, having no unary
% symbols, so $\p{n-m}=0$, e.g. the booleans, the number of ground terms of size
% $i>0$ is $m$, m.f.

% $$\card{\mathcal{T}_\downarrow\p{\Psi}^{\leq i}} = m \cdot \iverson{i>0}$$

% \end{theorem}

% \begin{proof} For any $i$, consider the forest
% $\mathcal{F}_\downarrow\p{\Psi}^{< i}$. If $i=0$, the forest has $0$ nodes.
% If $i>0$, the forest has exactly $m$ nodes. \end{proof} of the methods
% surveyed to conventional data types in general, in particular, 

% \begin{theorem} For a single-sorted word signature $\Psi$, having exactly one
% unary symbol, so $\p{n-m}=1$, e.g. the Peano naturals, we have

% $$\card{\mathcal{T}_\downarrow\p{\Psi}^{\leq i}} = m \cdot i$$

% \end{theorem}

% \begin{proof} For any $i$, consider the forest
% $\mathcal{F}_\downarrow\p{\Psi}^{< i}$. This is a forest of $m$ chains of
% length $i$. The number of nodes in such a chain is $i$.  \end{proof}

% \begin{theorem} For a single-sorted word signature $\Psi$, having more than one
% unary symbol, so $\p{n-m}>1$, e.g. binary notation, we have

% $$\card{\mathcal{T}_\downarrow\p{\Psi}^{\leq i}} = m \cdot
% \frac{\p{n-m}^i-1}{\p{n - m} - 1}$$

% \end{theorem}

% \begin{proof} For any $i$, consider the forest
% $\mathcal{F}_\downarrow\p{\Psi}^{< i}$. This is a forest of $m$ complete
% $\p{n-m}$-ary trees of height $i-1$. It is well-known that the number of
% nodes in such a tree is $\frac{\p{n-m}^i-1}{\p{n - m} -
% 1}$\cite{cormen-et-al-2009}.  \end{proof}

% \begin{example} $\card{\mathcal{T}_\downarrow\p{B}^{\leq i}} = 1 \cdot 2^i$.
% (Recall that we also count the empty string.) \end{example}

% \begin{definition} Let the size of a term over $\Sigma$ be the number of
% symbols involved in the construction of the term:

% \begin{align*}
% \card{v_0}        &=  0 \\
% \card{v_1\p{x}}   &=  \card{x} + 1 \\
%                  &   \vdots \\
% \card{v_{i}\p{x}} &=  \card{x} + 1
% \end{align*}

% \end{definition}

% To get a closed form for the size function, we make a simplifying assumption
% that there is at most 

% \begin{lemma} If $\Sigma$ is a single-sorted signature, then
% $\card{x}<\card{v_k\p{x}}$ for all $\p{i+1} \leq k \leq j$.  \end{lemma}

% \begin{definition} The \textbf{smash function} over a single-sorted word
% signature $\Psi$, with $m$ nullary and $\p{n-m}$ unary symbols, is a binary
% symbol \smashf{}, written in infix notation, with the property that $\card{x
% \smashf{} y} = {\card{x}\cdot\card{y}}$.\end{definition}

% The usefulness of the smash function is illustrated by the fact that $\card{x
% \smashf{} y} = 2^3$

% \begin{definition} A single-sorted word signature $\Sigma$, extended with a
% smash function is $\Sigma^\smashf{} \triangleq \Sigma \cup \set { \smashf{} : 2
% }$. \end{definition}

% In 1964 Cobham [C] gave the first recursion-theoretic characterization of
% polynomial time, using a recursion whose depth was the binary length of the
% number being recursed on, and using explicit polynomial bounds to control the
% size of the output.\cite{bellantoni-phd-1992}.

% recursion scheme over an inductively or coinductively defined data type (an
% algebra).

% \begin{remark} It seems a bit superfluous to deal in the additional argument
% vector $\vect{y}$ which we simply pass down to our base and step functions.
% This could be handled by e.g. partially applying $\vect{y}$ on the given
% functions.  Although not important to the recursion scheme, we will soon take
% the stance that the arguments in $\vect{y}$ are of a different ``nature'' than
% the arguments we recur on.\end{remark}

% It seems straight forward to generalize bounded primitive recursion on notation
% to many-sorted free algebras.

\section{Generalisation}

% There is nothing special about the choice of natural numbers in decimal
% notation: we can make due with a single-sorted word signature, i.e. the
% language of strings over an alphabet.

In element (1) of the definition of $\mathcal{L}$, we let $\sem{n} = n$, for
all $n \in \mathbb{N}$. This can be considered a matter of ``syntactic sugar'':
any $n \in \mathbb{N}$ (including $0$) can (also) be represented in
$\mathcal{L}$ as follows:\begin{align*}
%
x_1\p{\cdots x_{\card{n}}\p{0} \cdots} & \quad \quad \text{where $x_i \in
\set{\mathtt{s_0}, \ldots, \mathtt{s_9}}$}
%
\end{align*}

\todo{How do I deal with the fact that $\log_2\p{1} = 0$, so \card{0} = 0?}

It follows that if we introduce a \textbf{zero} operator $\sem{\mathtt{z}} :
\mathbb{N}$ with the semantics $\sem{\mathtt{z}} = 0$, we can do away with (1)
in the definition of $\mathcal{L}$.

This demands to a further change to the special cases of $\mathtt{BPRN}$, i.e.
those where $0$ occurs:\begin{align*}
%
f\p{\mathtt{s_0}\p{0}, \vect{y}} &= \sem{g}\p{\vect{y}} \\
%
f\p{\mathtt{s_0}\p{x}, \vect{y}} &= \sem{h_0}\p{x, \vect{y}, f\p{x, \vect{y}}}
& \text{if $x\neq 0$}
%
\end{align*} are replaced by \begin{align*}
%
f\p{\mathtt{z}, \vect{y}} &= \sem{g}\p{\vect{y}} \\
%
f\p{\mathtt{s_0}\p{\mathtt{z}}, \vect{y}} &= \sem{g}\p{\vect{y}} \\
%
f\p{\mathtt{s_0}\p{x}, \vect{y}} &= \sem{h_0}\p{x, \vect{y}, f\p{x, \vect{y}}}
& \text{if $x\neq \mathtt{z}$}
%
\end{align*}

Except for the bounding condition, $\mathtt{BPRN}$ now looks like a regular
structural recursion, as in \refSec{primitive-recursion}. Thus, we are dealing
with a \textbf{single-sorted word signature} $\Psi = \p{\mathbb{A},
\set{\mathtt{z}, \mathtt{s_0}, \ldots, \mathtt{s_9}}}$, of which $\mathbb{N}$
is merely a model.

Dealing in terms, rather than natural numbers, requires to modify the size
function e.g. to the length of the term, as in \refSec{size}. Changing the size
function, requires changing the smash function to conform to the general
rule:\begin{align*}
%
\card{x\smashf{}y} = \bigOh{\card{x} \cdot \card{y}}
%
\end{align*}

\todo{Give the generalised version of $\mathcal{L}$ and prove (not sketch)
\FPTIME{} capture.}

\section{Examples}

\todo{Missing}

Of course, it remains to prove that the bound indeed is satisfied, which
segways neatly into the following discussion.

\section{Discussion}

Although bounded primitive recursion on notation has led to some interesting
work \cite{cook-1975, buss-phd-1985-6, cook-urquhart-1993}, it is generally
discarded\cite{bellantoni-cook-1992}, based on either that:

\begin{enumerate}[label=(\arabic*)]

\item the smash operator is not ``small''; or

\item the bounds are enforced in an ``extensional'' manner, rather than being
an ``intensional'' property of the system.

\end{enumerate}

The first trouble is perhaps best delineated by \cite{cobham-1965} himself.
Neither $x^\card{y}$ for $n$-ary notation, nor $2^{\card{x} \cdot \card{y}}$
for binary notation, is in $\mathcal{E}^2$, i.e. the second level of the
so-called Grzegorczyk hierarchy\cite{grzegorczyk-1953}. Specifically,
$\mathcal{E}^2$ is \todo{the smallest class of functions over $n$-ary notation,
containing multiplication}, and closed under the operations of explicit
transformation, composition and bounded primitive recursion. In other words, we
cannot smash in linear space, that is, using an $O\p{\card{x}+\card{y}}$-space
Turing machine\cite{ritchie-1963}. On the other hand, \cite{buss-phd-1985-6}
finds the use of \smashf{} ``natural and elegant'', as it has precisely the
growth rate necessary to fund the \todo{polynomial-time hierarchy}.

As to the other trouble, to put a $\mathtt{BPRN}$ operator to good use, we need
to prove that the bounding inequality is satisfied.  This leads to grave
complications in the cited work, but is sometimes outwitted (see e.g.
\cite{cook-urquhart-1993}) by changing the semantics of $\mathtt{BPRN}$ to
simply cut off at overflow. This is perhaps indifferent to a classical
characterisation of \FPTIME{}.

Considering the techniques one might employ for proving such a bound, one
quickly finds oneself counting and guessing, rather than the bound being
``guaranteed by construction''.  It is perhaps this quality that makes bounded
primitive recursion on notation inherently ``inintrinsic'', rather than the
mere fact that the bounds are stated explicitly, as criticised by e.g.
\cite{hofmann-2000a}.
