\chapter{Computational Complexity} \label{sec:computational-complexity}

\begin{quotation}

\footnotesize\sffamily\itshape

\begin{flushright}

This may be a good point to mention that, although I have so far been tacitly
equating computational difficulty with time and storage requirements, I don't
mean to commit myself to either of these measures. It may turn out that some
measure related to the physical notion of work will lead to the most
satisfactory analysis; or we may ultimately find that no single measure
adequately reflects our intuitive concept of difficulty.

\smallbreak

\upshape

--- ALAN COBHAM, {\itshape Logic, Methodology and Philosophy of Science} (1964)

\end{flushright}

\end{quotation}

\begin{quotation}

\footnotesize\sffamily\itshape

\begin{flushright}

In practice, the length of computer computations must be restricted, otherwise
the cost in time and money would be prohibitive.

\smallbreak

\upshape

--- H. E. ROSE, {\itshape Subrecursion: functions and hierarchies} (1984)

\end{flushright}

\end{quotation}

% Also known as Complexity, as in Computability & Complexity, and hence
% Complexity Theory

In the context of computing with Turing machines, there appeared two unbounded
resources: the number of state transitions involved in a computation, and the
number of squares on the machine tape. For any halting computation, only a
finite number of state transitions occur, and so only a finite number of tape
squares are ever scanned. It is perhaps natural to consider these two factors
as characterising the ``complexity'' of a halting computation.

\begin{definition} Let $M$ be a Turing machine that halts an all inputs.

\begin{enumerate}

\item The \textbf{time complexity} of $M$ is a function
$T:\mathbb{N}\rightarrow \mathbb{N}$, where $T\p{n}$ is the maximum number of
state transitions that $M$ makes for any input of length $n$.

\item The \textbf{space complexity} of $M$ is a function
$S:\mathbb{N}\rightarrow \mathbb{N}$, where $S\p{n}$ is the maximum number of
tape cells scanned by $M$ for any input of length $n$.

\end{enumerate}

\end{definition}

In the interest of classifying the halting computations by their time and space
complexities, it has become customary to use approximations rather than the
precise functions above.

The approximations are in the form \textbf{asymptotic notation}

\begin{definition} Let $f,g:\mathbb{N}\rightarrow \mathbb{R}^+$. We say that
$g\p{n}$ is an \textbf{asymptotic upper bound}, or simply upper bound, of
$f\p{n}$, written $f\p{n}=O\p{g\p{n}}$, if there exists a $c:\mathbb{R}^+$ and
$n_0:\mathbb{N}$, such that for every $n\geq n_0$,

$$f\p{n}\leq c\cdot g\p{n}.$$

\end{definition}


The theory of computational complexity deals in the analysis of the complexity
of halting computations. As such, it is 

Although the comparison of such precise functions can provide for more precise measures of complexity, it has become customary to use approximations rather than 

% Some complexity classes are interesting in their own right, as they are (more
% or less) the same, regardless of the choice of classical machine model.
% Perhaps the discussion of machine models really should be integrated with the
% discussion of complexity.

% It is perhaps at this point (in the discussion of machine models) that we
% mention our assumptions that a logical system is indifferent to a programming
% language.

\section{Machine models}

Although the Church-Turing thesis permits us to specify algorithms by means of
pseudo-code, this admits a pitfall. There is no guarantee that all the steps
involved in our algorithm are, in fact, computable.

To make sure that our algorithms are always computable (as by the Church-Turing
thesis), we have to formally specify our algorithm description language, aka.
our programming language, and show that it can be simulated by a Turing
machine.

...

% it is even better wrt. talking about complexity to specify the machine model,
% show it simulateable by a Turing machine.

% so it is okay to write crazy stuff? what you would really do, is show that
% your formal system can be simulated by a turing machine.

% maybe not as we survey other work.

% simulation

\section{Complexity}

...

\begin{notion} \emph{The Cobham-Edmonds thesis.}

The feasible functions are those which can be computed in time polynomial in
the size of the input.

\end{notion}

...

% measure of size is machine dependent.
% basic instructions are machine dependent.

In what follows, we refrain from quantification over (countably, or otherwise)
infinite domains. This is out of the consideration that there is no mechanical
procedure to check a property quantified over an infinite domain within a
finite amount of time.

\section{Feasibility}

\begin{notion} An algorithm is \textbf{feasible} if it is practical within its
domain of operation. \end{notion}

Cobham-Edmonds thesis

Superpolynomial bounds, say $n^{\log{\log{n}}}$ are lower than say $n^{100}$ for all
$n$ encountered in practice ($<n^{2^{100}}$).

\begin{conjecture} \label{cjt:hartmanis-baker} (Hartmanis-Baker conjecture) All
PTIME machine models are PTIME isomporphic. \end{conjecture}

P is the class of decision problems decidable in polynomial time.

FP is the class of functions computable in polynomial time. The trouble with FP
as opposed to mere P might be processing needs to generate the richer output.
That is other than that problems are formulated differently.

\section{Some informalities}

\begin{notion} An algorithm is \textbf{efficient} if ... \end{notion}

\begin{notion} An algorithm is \textbf{elegant} if ... \end{notion}

\begin{notion} An algorithm is \textbf{eloquent} if it is an acceptable
trade-off between an efficient and elegant algorithm for the matter at hand.
\end{notion}

\begin{notion} The \textbf{eloquent} programmer is interested in writing
eloquent algorithms. \end{notion}
