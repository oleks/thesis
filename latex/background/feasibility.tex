\chapter{Feasibility}

\begin{quotation}

\footnotesize\sffamily\itshape

\begin{flushright}

Anyone familiar with theory of computability will be aware that practical
conclusions from the theory must be drawn with caution.

\smallbreak

\upshape

--- MEYER \& RITCHIE, \emph{Proc. of the 22nd ACM National Conference} (1967)

\end{flushright}

\end{quotation}

\begin{hypothesis} \emph{(Cobham-Edmonds thesis)}

\FPTIME{} is the class of practically feasible functions.

\end{hypothesis}

Unfortunately, this thesis is easily disproven in practice.

A mere exponential growth rate does not always tell a sufficient story --- much
like the mean value does not always tell a sufficient story of a normal
distribution.  Exponential functions have perhaps a ``point of equilibrium'',
beyond which they ferociously outgrow any given polynomial. This point may only
come about at infeasibly large numbers.

For instance, 

As a particular example, $c_1 \cdot n^{\log\log{n}} \leq c_2 \cdot n^{100}$

A right off the bat argument is that the number of atoms

None-the-less, the thesis stands on firm ground when backed by the following conjecture:

Once we've found an \PTIME{} algorithm for a problem, we tend to find faster
algorithms. Whereas \NPTIME{} is often associated with ``brute-force''
techniques.

\todo{representable algorithm}
\index{algorithm!representable}
