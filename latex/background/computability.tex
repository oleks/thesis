\chapter{Computability} \label{sec:background-computability}

\begin{quotation}

\footnotesize\sffamily\itshape

\begin{flushright}

The ``computable'' numbers may be described briefly as the real numbers whose
expressions as a decimal are calculable by finite means.

\smallbreak

\upshape

--- ALAN TURING, {\itshape Proceedings of The London Mathematical Society} (1936-7)

\end{flushright}

\end{quotation}

The theory of computability is concerned with the nature of computation. It
originated in the 1930s with the celebrated works of G\"odel, Kleene, Church,
Post, Turing, et al.  Fundamental to the field is the Church-Turing thesis,
which proposes that the natural phenomena of computation is captured by the
notion of computation with a Turing machine.

We begin with a brief discussion about the intuitive notion of computation,
leading way for a formal\footnote{By ``formal'', in this chapter and the next,
we mean built atop the mathematical foundations as outlined in
\refSection{preface:mathematical-foundations}. It is an interesting thesis in
its own right to consider an exposition of computability and complexity theory
as based solely on the Church-Turing thesis as an axiom.} definition of
computation with a Turing machine, and an exposition of the Church-Turing
thesis.

\begin{notion} A problem is ``computable'' if it can be solved by transforming
a mathematical object, without ingenuity.
\end{notion}

Any attempt at a more definite notion of computability seems to arrive at a
philosophical impasse, where the notions of ``mathematical object'',
``transformation'', and ``ingenuity'' form a philosophical conundrum. The
indefinite notion however, is sufficient to state the following hypothesis:

\begin{hypothesis} \label{ths:problem-composition} If a problem $P$ can be
solved by solving a computable problem $Q$, immediately followed by solving a
computable problem $R$, then $P$ itself is computable. \end{hypothesis}

The hypothesis clearly holds as nothing is done other than solve the two
computable problems\footnote{Although, doing nothing, is itself a philosophical
conundrum.}.  So any problem which can be solved by solving a finite sequence
of computable problems, is itself computable. We now arrive at an intuitive
notion of an algorithm:

\begin{notion} \label{ntn:problem-algorithm} An ``algorithm'' is a
specification of how a problem can be solved by solving a finite sequence of
computable problems, called ``steps''.\end{notion}

Such indefinite notions are useful for little else. We are left to fall prey to
some choice of formalism. To this end, the notion of computation with a Turing
machine, due to Alan Turing\cite{turing-1936-7}, forms perhaps the most
classical conceptualization of computation. The conceptualization below is
heavily inspired by the those that appear in \cite{tourlakis-1984},
\cite{jones-1997}, \cite{homer-selman-2011}, and \cite{sipser-2013}, but
slightly differs from all of them.

A \textbf{Turing machine} is a symbol-manipulating machine, working with a
1-dimensional, unbounded tape divided into discrete, equally sized squares.
Each square is occupied by a symbol drawn from the \textbf{tape alphabet} of
the machine.  The tape alphabet consists of the \textbf{input/output alphabet},
usually denoted $\Sigma$, and a special \textbf{blank symbol}, written \blank,
which is not in $\Sigma$. The tape has a left edge, but extends into infinity
to the right, filled with blank symbols.

\begin{figure}[h!]
\centering
\begin{tikzpicture}
  \tikzset{tape/.style={minimum size=.7cm, draw}}
  \begin{scope}[start chain=0 going right, node distance=0mm]
   \foreach \x [count=\i] in {\symb{1}, \symb{0}, \symb{1}, \symb{0}, \symb{1}, \blank, \blank, \blank} {
    \ifnum\i=8 % if last node reset outer sep to 0pt
      \node [on chain=0, tape, outer sep=0pt] (n\i) {\x};
      \draw (n\i.north east) -- ++(.1,0) decorate [decoration={zigzag, segment length=.12cm, amplitude=.02cm}] {-- ($(n\i.south east)+(+.1,0)$)} -- (n\i.south east) -- cycle;
     \else
      \node [on chain=0, tape] (n\i) {\x};
     \fi
     \ifnum\i=1 % if first node draw a thick line at the left
      \draw [line width=.1cm] (n\i.north west) -- (n\i.south west);
     \fi
   }
   \node [right=.25cm of n8] {$\cdots$};
  \end{scope}
\end{tikzpicture}
\caption[]{Schematic of a Turing tape with $\Sigma=\set{\symb{0},\symb{1}}$.}
\label{fig:turing-tape}
\end{figure}

A Turing machine is always in one of a finite, nonempty set of \textbf{states},
also called the \textbf{state alphabet}, and has a tape head, always
overlooking a particular square on the tape. A Turing machine ``computes'' by
transitioning between states and working with the tape head.

The transition from one state to the next is uniquely determined by the current
state, and the symbol read off the overlooked square. A state transition bears
with it writing a new symbol to the overlooked square, and moving the tape head
one square to the left, or one square to the right.

There are two distinguished states, the \textbf{starting state}, usually
denoted \qstart, and the \textbf{halting state}, usually denoted \qhalt. A
Turing machine \textbf{starts} a computation by entering the state \qstart,
overlooking the square at the left end of the tape. A Turing machine
\textbf{halts}, i.e.  performs no further state transitions, if it ever
transitions to the state \qhalt. A halting computation is a finite sequence of
state transitions.

We now formalise these notions as follows:

\begin{definition} A Turing machine (TM) $M$ is a 5-tuple
$\p{Q,\qstart,\qhalt,\Gamma, \delta}$, where $\Gamma\supseteq\Sigma\cup\set{\blank}$, $\blank\notin\Sigma$,

\begin{enumerate}

\item $Q$ is the state alphabet,

\item $\qstart \in Q$ is the starting state,

\item $\qhalt \in Q$ is the halting state,

\item $\Gamma$ is the tape alphabet, where $\blank\notin
\Sigma$ and $\Sigma$ is the input/output alphabet, and

\item $\delta : Q \setminus \set{\qhalt} \times \Gamma \rightarrow Q \times
\Gamma \times \set{\symb{L}, \symb{R}}$ is the (total) transition function.

\end{enumerate}

We will omit stating the components of a TM, when they are clear from context.

\end{definition}

\begin{definition} An $M$-\textbf{configuration} of a TM $M$, is a 3-tuple,
written $uqv$, where $u\in\Gamma^*$, $q\in Q$, and $v\in\Gamma^*$. If
$v=\varepsilon$, it can be regarded as if $v=\blank$. \end{definition}

We will omit stating that $u\in\Gamma^*$, $q\in Q$, and $v\in\Gamma^*$ when
this is clear from context. The additional permission to regard an empty $v$ as
a $v$ containing a blank symbol facilitates the notion of an infinite tape.
This is illustrated by an example further below.

\begin{definition} An $M$-configuration $C_x$ \textbf{yields} an
$M$-configuration $C_y$, written $C_x\leadsto C_y$, iff either

\begin{enumerate}

\item $\delta\p{q_x,a}=\p{q_y,b,\goright}$, $C_x=uq_xav$, and $C_y=ubq_yv$,

\item $\delta\p{q_x,a}=\p{q_y,b,\goleft}$, $C_x=ucq_xav$, and $C_y=uq_ycbv$,
or

\item $\delta\p{q_x,a}=\p{q_y,b,\goleft}$, $C_x=q_xbv$,  and $C_y=q_ybv$;

where $a,b,c\in\Sigma$, and $q_x,q_y\in Q$.

\end{enumerate}

\end{definition}

\begin{definition}

A TM $M$ halts with output $y\in\Sigma^*$ on input $x\in\Sigma^*$, written
$M\p{x}\downarrow y$, iff there exists a finite sequence of $M$-configurations,
$\chev{C_1,C_2,\ldots, C_k}$, of length $k\in\mathbb{N}$, such that

\begin{enumerate}

\item $C_1=\qstart x$,

\item $C_i\leadsto C_{i+1}$ for all $1\leq i < k$,

\item $C_k=u\qhalt yv$.

\end{enumerate}

\end{definition}

Note, in particular that the ``tape'' of a TM in such a formalisation is not
infinite, and rather, finite but extensible. As an example, consider a TM $M$,
where $\Sigma=\set{\symb{\#}}$ and $\delta\p{\qstart, \blank} =
\p{\qhalt,\#,\goleft}$. We have that $\qstart\leadsto\qhalt\#$ and
$M\p{}\downarrow\#$.

\section{Minor variants}

In this section, we give some justification for our definition of Turing
machines by considering how it can simulate various variants often found the
literature. The ability of our Turing machines to simulate these variants
builds confidence in our definition, and permits us to stand on the shoulders
of giants, and state the Church-Turing thesis based on our notion of Turing
machines.

\begin{itemize}

\item $\Sigma\cup\set{\blank}\subseteq\Gamma$.

\item $\set{\symb{L},\symb{R},\symb{S}}$.

\item Multiple halting states.

\item Multiway and multiple tapes.

\end{itemize}

\section{Major variants}

Major variants differ from minor variants in that they tend to play a major
role in the theory of computability and complexity.

\begin{itemize}

\item Nondeterministic Turing machines.

\item Enumerators.

\end{itemize}

\section{Church-Turing thesis}

\begin{hypothesis} \textit{Church-Turing thesis.}

No intuitive notion of an algorithm can compute a function that a Turing
machine cannot compute.

\end{hypothesis}

The Church-Turing thesis does not lend itself to mathematical proof. Rather, it
is a proposed explanation of the natural phenomena of computation. The
hypothesis remains to be disproven, and in the mean-while, computational models
more convenient than the Turing machine have emerged, some of them, even useful
in practice.

Some of these models have a formal specification, and so it lends itself to
formal proof that these models compute no more than a Turing machine does.

Our belief in the Church-Turing thesis has however, a notational advantage: to
show that a function $f$ is computable, it is sufficient to informally
describe, an algorithm for computing $f$. Due to the hypothesis, any
formalisation of the informal algorithm, will yield a computable algorithm,

This is sometimes called providing the ``pseudo-code'' for computing $f$, but can be said
of most ``code'' today in general.

Formal methods, due to their inherent complexity are seldom employed.

\section{What does Turing machines compute?}

The introduction to this chapter talked about ``problems'' being computable. We
went on to define computation by means of a Turing machine. We now formally
discuss what it is that a Turing machine computes, effectively giving a
definition of what we mean by the notion of a ``problem''.

\begin{itemize}

\item Recursively enumerable.

\item Recursive.

\end{itemize}

\begin{theorem} For every element $x\in X$, of an enumerable set $X$, there is
a unique \textbf{representation} of $x$ in $\Sigma^*$, written
$\overline{x}\in\Sigma^*$, where $\Sigma$ is an alphabet. \end{theorem} 

\begin{definition} A TM $M$ \textbf{computes} a partial function
$f:X\nrightarrow Y$, where $X$ and $Y$ are enumerable sets, if given a
$\overline{x}$ as input, $M$ halts with $\overline{f\p{x}}$ as output, whenever
$f\p{x}$ is defined, and does not halt otherwise.  \end{definition}

\begin{definition} A function is \textbf{computable}, or \textbf{recursively
enumerable} if there exists a Turing machine that computes it. \end{definition}

\section{Some uncomputable problems}
