% vim: set spell:

\chapter{Complexity}

\begin{quotation}

\footnotesize\sffamily\itshape

\begin{flushright}

This may be a good point to mention that, although I have so far been tacitly
equating computational difficulty with time and storage requirements, I don't
mean to commit myself to either of these measures. It may turn out that some
measure related to the physical notion of work will lead to the most
satisfactory analysis; or we may ultimately find that no single measure
adequately reflects our intuitive concept of difficulty.

\smallbreak

\upshape

--- ALAN COBHAM, {\itshape Logic, Methodology and Philosophy of Science} (1964)

\end{flushright}

\end{quotation}

\begin{quotation}

\footnotesize\sffamily\itshape

\begin{flushright}

In practice, the length of computer computations must be restricted, otherwise
the cost in time and money would be prohibitive.

\smallbreak

\upshape

--- H. E. ROSE, {\itshape Subrecursion: functions and hierarchies} (1984)

\end{flushright}

\end{quotation}

\begin{definition} The computational complexity of a function $f$, wrt. a
particular resource, quantifies the use of that resource as a function of the
length of the input string.\end{definition}

\section{Time}

\subsection{Polynomial Time}

\begin{itemize}

\item Recursive characterization of polytime functions in \cite{rose-1984},
proving certain claims by \cite{cobham-1965}. Both question the relation to the
Grzegorczyk hierarchy \cite{grzegorczyk-1953}.

\item Leivant's paper - A Foundational Delineation of Computational Feasibility.

\item Bellantoni and Cook paper - A NEW RECURSION-THEORETIC CHARACTERIZATION OF
THE POLYTIME FUNCTIONS

\item Niel Jones paper.

\item Caporaso

\item Upper bounds (algorithms) can be produced by expressing the property of
interest in one of our languages. Lower bounds proven elsewhere can be used as
a proof that the language is expressive enough.

\end{itemize}

\subsection{Subpolynomial Time}

In what follows, we delineate a hierarchy of complexity classes, strictly under
polynomial time. That is, we present a sequence $C_1(n),$ $t(n)=o\p{n^{\bigOh{1}}}$. For each
complexity class $C$, we present a representative problem $P_C$. We aim to find
problems which are known to be $t(n)=\Theta\p{C\p{n}}$.

\begin{itemize}

\item Some problems, although computable in polynomial time, are still hard to
compute in practice (ICALP'2014, Amir Abboud).

\item Remind of the definitions of $O$, $\Omega$, etc.

\item For each of the below show that every subsequent class is distinct from
the proceeding, and exhibit some ``complete'' problems for these classes.

\end{itemize}

\begin{description}

\item[$\bigOh{1}$ --- Constant]

\item[$\bigOh{\alpha\p{n}}$ --- Inverse Ackermann]

\item[$\bigOh{\log^*{\p{n}}}$ --- Log star]

\item[$\bigOh{\log{\log{\p{n}}}}$ --- Log-log]

\item[$\bigOh{\log{\p{n}}}$ --- Log]

\item[$\bigOh{\log{\p{n}^{\bigOh{1}}}}$ --- Polylog]

\item[$\bigOh{n^c}$, for $0<c<1$ --- Fractional power]

\item[$\bigOh{n}$ --- Linear time]

\item[$\bigOh{n\log^*{\p{n}}}$ --- n log star]

\item[$\bigOh{n\log{\log{\p{n}}}}$ --- n log-log]

\item[$\bigOh{n\log{\p{n}}}$ --- n log n]

Comparison-based sorting $o\p{n\log{n}}$.

\item[$\bigOh{n^2}$ --- quadratic]

\item[$\bigOh{n^3}$ --- cubic]

\item[$\bigOh{n^O(1)}$ --- polynomial]

\end{description}

\subsection{Space}
