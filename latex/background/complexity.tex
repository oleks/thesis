% vim: set spell:

\chapter{Complexity}

\begin{quotation}

\footnotesize\sffamily\itshape

\begin{flushright}

This may be a good point to metion that, although I have so far been tacitly
equating computational difficulty with time and storage requirements, I don't
mean to commit myself to either of these measures. It may turn out that some
measure related to the physical notion of work will lead to the most
satisfactory analysis; or we may ultimately find that no single measure
adequately reflects our intuitive concept of difficulty.

\smallbreak

\upshape

--- ALAN COBHAM, {\itshape Logic, Methodology and Philosophy of Science} (1964)

\end{flushright}

\end{quotation}

\section{Time}

\subsection{Polynomial Time}

\begin{itemize}

\item Problems computable in polynomial time on a TM lie in the second level of
the Grzegorczyk hierarchy (Cobham'1965 - definition of feasibility, proven in Rose'1984). 

\item 

\end{itemize}

\subsection{Subpolynomial Time}

Some problems, although computable in polynomial time, are still hard to
compute in practice (ICALP'2014, Amir Abboud).

\begin{description}

\item[$\bigOh{1}$ --- Constant]

\item[$\bigOh{\alpha\p{n}}$ --- Inverse Ackermann]

\end{description}

\subsection{Space}
