\section{Shorthands} \label{sec:introduction:shorthands}

To shorten various proofs and discussions, a range of shorthands are employed
throughout this document. These are listed and expanded below.

\begin{enumerate}

\item ``iff'' stands for ``if and only if''.

\item Something is said to be ``well-known'', if the reader is assumed to be
familiar with it as per \refSec{introduction:audience}. For instance, ``it is
well-known that a finite tree has a finite number of leafs''. \index{well-known}

\item Something is said to be ``perhaps'' if it is to be taken as an informal
interpretation (by the author) of a particular phenomenon. For instance, ``the
general idea of recursion is perhaps as follows''. Such statements are
off-the-hip statements, and seek perhaps only to warp the reader's
intuition.\index{perhaps}

\end{enumerate}
