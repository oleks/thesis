\chapter{Introduction}

%This chapter serves to motivate and preface the following thesis.

\section{Motivation}

In most matters of life, it is of principal concern to represent and transform
data. In high profile applications, such as where human fortunes are at stake,
it is important that data transformations occur in a reliable manner.  This
means both that the data is transformed in an expected way, but also that this
is done in a timely manner.

The advent of digital computers has delivered much reliability, provided that
their human programmers can deliver correct and efficient programs for the
matter at hand. Much has been done in the field of data representation and
transformation, i.e. Computer Science, in terms of guaranteeing the correctness
and judging the efficiency of programs. This thesis is about guaranteeing
both the correctness and efficiency of programs.

In support of the fundamental human faculty of managing data, it is natural in
Computer Science to consider data representations and transformations as data
in themselves. Accompanied by additional descriptors, this leads way for basic
buildings blocks for the more eloquent management of data. Such
``abstractions'' in Computer Science both alleviate the mundane details of data
representation and transformation, and enable more efficient communication
about programs amongst the humans themselves.

\begin{example} \label{example:alice-bob-add} \ 

It may be sufficient for a Alice to know that Bob has written a program
\textsc{Add}, which given two integers, $x$ and $y$, in the range $[0;31]$,
after a handful machine instructions, returns an integer, $z$, such that $z=x+y
\mod 32$.

Alice could now use \textsc{Add} to compute the sum of an array of integers in
the range $[0;31]$, in a number of machine instructions proportional to the
number of elements in the array. That is, Alice didn't need to know the
internal details of how \textsc{Add} works, merely some high-level descriptors
of what it does.

\end{example}

\begin{example} \label{example:alice-bob-email} \

It may be sufficient for Alice to know that the network of computers set up
between her and Bob, provided sufficient means from the electricity grid, will
make sure that Alice's email will eventually be delivered to Bob. 

\end{example}

Abstractions incorporate into the interface amongst humans and computers.

% A good programming language puts the tools of the language designer at the
% programmers fingertips. And so anything written in a programming language
% defines a programming language.

% We approach the problem from a programming languages perspective. A digital
% computer typically understands only one language --- machine language. As
% programming is principally a human activity, 

% Programming languages serve to communicate the intent of the programmer both
% to a computer, other programmers, and the programmer themselves.

\begin{itemize}

\item Programming Languages

\item Human-computer interaction

\item Computability and Complexity

\end{itemize}

The rest of this chapter is concerned with prefacing matters.

\section{External references}

\section{Audience} \label{sec:introduction:audience}

The audience of this thesis is anyone interested in the connection of
computability and complexity to the theory and practice of programming
languages.

In particular, what the admittance of useful programming language constructs
implies for the time and space complexity of the programs that you can write. A
programming language construct is ``useful'' if its admission permits to write
a practical class of programs in an elegant manner.

The thesis is directed towards the level of a Computer Science graduate student
at the time of writing: The reader is assumed to be familiar with the basics of
discrete mathematics, as in \cite[\ch~0]{sipser-2013}, and \cite[Appendices A,
B, and C]{cormen-et-al-2009}. The analysis of time and space complexity of
algorithms, as in \cite[\chs~1--17 and \chs~21--24]{cormen-et-al-2009}. Regular
and context-free languages, as in \cite[\chs~1--2]{sipser-2013}, and their use
for programming language design, as in \cite{mogensen-2010}. The reader should
also be familiar with Logic in Computer Science, as in
\cite[\chs~1--4]{huth-ryan-2004}.


\section{Internal references} \label{sec:introduction:internal-references}

All references within this document to parts of this document are qualified
with the page on which the referenced part begins. For instance, we refer to
this section as \refSection{introduction:internal-references}, read ``section
zero point two, on page two''.

\section{Named segments}

Named segments are text segments with a role and an address. The role seeks to
distinguish the text segment from prose, and the address permits to
subsequently reference the segment. For instance, \refExample{alice-bob-add}.

Addresses are of the format \mbox{$x$.$y$.$z$}, where $z$ is the index of the
named segment in section indexed $y$, in chapter indexed $x$. Where
appropriate, we'll give a named segment a name for even easier reference, for
instance, when defining the concept of a Turing machine. The various roles are
described below, showcasing also how each named segment will be formatted.

\begin{notion} An informal definition of a concept, appealing either to the
reader's intuition or formerly stated notions. \end{notion}

Some concepts are too disputed to be given by a notion. In the presence of
multiple competing notions, we will merely specify the features that we are
interested in.

\begin{specification} An informal specification of a concept, appealing either
to the reader's intuition or formerly stated notions or specifications.
\end{specification}

\begin{hypothesis} An informal statement, proposed as an explanation of a
natural phenomenon. A hypothesis does not admit itself to formal proof, but can
be disproven by observation. A hypothesis is always followed by a discussion of
its validity. \end{hypothesis}

Some notions are ``primitive'', in the sense that they underpin subsequent
formal definitions. It is a fundamental epistemological impasse, that in any
formal system, some concepts must remain distinguished, but undefined, these
form the \textbf{primitive notions}, or \textbf{axioms}, of the formal system.

\begin{definition} A formal definition of a concept, appealing to either
formerly stated primitive notions or definitions. Definitions seek to underpin
subsequent formal definitions and analyses. \end{definition}

This thesis only seeks to be self-contained beyond what can be expected of the
intended audience. Some notions, hypotheses, and definitions appeal to
knowledge outside this thesis. See also \refSection{introduction:audience}.

\begin{theorem} \label{thm:theorem} A formal statement using formerly
defined concepts. A theorem is always proven to hold by an immediately
proceeding proof.\end{theorem}

\begin{proof} A formal or informal proof the preceding theorem. \end{proof}

As proofs occur immediately after a theorem, proofs are not decorated with an
address. A proof is then referred to by the address of the theorem it proves.
For instance, we refer to the proof above as \refProof{theorem}.

\begin{example} An example of a concept discussed above. Examples seek to
facilitate the reader's understanding on the concept in question.
\end{example}

\begin{notation}

A choice of notation for some formally or informally defined concepts.
Typically, notation is introduced to aid further discussion.

\end{notation}

A concept is put in quotations, for instance ``computable'', if it is
introduced informally as part of a notion. A concept is put in bold, for
instance \textbf{computable}, if it is introduced formally as part of a
definition.

\section{Mathematical Foundations}

\label{sec:preface:mathematical-foundations}

\def\strue{\ensuremath{\text{\textsc{True}}}}
\def\sfalse{\ensuremath{\text{\textsc{False}}}}

It is beyond the scope of thesis to justify a choice of mathematical
foundations other than \emph{classical mathematics}. This is the sort of
mathematics often taught prior to higher education, and throughout; so the
reader is assumed to be familiar with it. This choice is also natural as most
related literature assumes a classical foundation.

This is certainly not the only possible foundation of mathematics, as advocated
by various schools of \emph{constructive mathematics}. In a nutshell,
constructive mathematics takes the stance that a proof that a mathematical
object exists, constitutes specifying a way of (mentally) constructing that
object.

Although we assume a classical foundation, we will seek to keep ourselves to
constructive arguments --- the approach often taken in related literature.

\subsection{Sequences}

We begin with a brief discussion of some typographical, rather than
mathematical matters. Sequences have a particular notational significance in
the remainder of this document.

\begin{notion}

A \textbf{sequence} is a ``juxtaposition'' of \textbf{elements}.

\end{notion}

Some ``short'' sequences can be denoted typographically. For instance,

$$\symb{1}\symb{0}\symb{1}\symb{0}\symb{1}\symb{0}$$

To make the distinction between the elements of a sequence clear, and to
distinguish various types of sequences, some sequences will appear ``separated
by'' a distinguished symbol. For instance,

$$\symb{1},\symb{0},\symb{1},\symb{0},\symb{1},\symb{0} \quad \text{or} \quad
X\rightarrow Y \rightarrow Z.$$

Some sequences will be ``too long'' to be sequenced typographically. Such
sequences will be given by stating a ``sequencing algorithm'', by which the
elements can be sequenced, one-by-one, at the reader's leisure. There need not
be a point at which we have sequenced all the elements of a sequence, and so
the reader should take care not to get carried away.

One informal way of specifying a sequencing algorithm is to typographically
sequence the first couple of elements, and to terminate the sequence by an
ellipses ($\ldots$) We then rely on the reader's intuition to deduce how the
subsequent elements are to be sequenced, one-by-one\footnote{Here, we ignore
Wittgenstein's rule-following paradox \cite[\textsection~201, p.
69]{wittgenstein-1953}.}. For instance,

$$\symb{1},\symb{0},\symb{1},\symb{0},\symb{1},\symb{0},\ldots \quad \text{or} \quad
A\rightarrow B \rightarrow C\rightarrow\cdots$$

We say that the elements of a sequence appear in a particular ``order'', and
sometimes, the elements, or sequences of elements, ``repeat'' within a
sequence. Such order and repetition will sometimes be significant.

Some sequences will be otherwise decorated to further guide the reader's
intuition about the sequence. For instance, the one-dimensional ``Turing
tape'', as discussed in \refSection{background-computability}, is a sequence of
``tape squares'' having a left edge, but extending to infinity to the right.  A
Turing tape may be illustrated as follows:
 
\begin{center}
\begin{tikzpicture}
  \tikzset{tape/.style={minimum size=.7cm, draw}}
  \begin{scope}[start chain=0 going right, node distance=0mm]
   \foreach \x [count=\i] in {\symb{1}, \symb{0}, \symb{1}, \symb{0}, \symb{1}, \blank, \blank, \blank} {
    \ifnum\i=8 % if last node reset outer sep to 0pt
      \node [on chain=0, tape, outer sep=0pt] (n\i) {\x};
      \draw (n\i.north east) -- ++(.1,0) decorate [decoration={zigzag, segment length=.12cm, amplitude=.02cm}] {-- ($(n\i.south east)+(+.1,0)$)} -- (n\i.south east) -- cycle;
     \else
      \node [on chain=0, tape] (n\i) {\x};
     \fi
     \ifnum\i=1 % if first node draw a thick line at the left
      \draw [line width=.1cm] (n\i.north west) -- (n\i.south west);
     \fi
   }
   \node [right=.25cm of n8] {$\cdots$};
  \end{scope}
\end{tikzpicture}
\end{center}

\subsection{Sets}

It is beyond the scope of this thesis to develop an axiomatic theory of sets.
We assume a notion of sets as given, subject to the following specification:

\begin{specification} A \textbf{set} is a mathematical object that is distinct
from, but completely determined by its \textbf{elements}. \end{specification}

\begin{notation} By $X\triangleq Y$, read ``$X$ is by definition the same as
$Y$'', we mean that $X$ is henceforth interchangeable with $Y$, until $X$
occurs on the left side of another $\triangleq$, that is, until $X$ is
redefined. \end{notation}

A set will always be denoted enclosed in braces. For instance, the simplest set
is the empty set, having no elements. We define the empty set as follows:

\begin{definition} The \textbf{empty set}, written $\emptyset$, is

$$\emptyset\triangleq\set{}.$$

\end{definition}

Some sets can be given by denoting their elements in a typographical sequence,
separated by commas. For instance, we define the set of boolean values as follows:

\begin{definition} The set of \textbf{boolean values} is

$$\mathbb{B}\triangleq\set{\strue,\sfalse}$$

\end{definition}

Some sets have too many elements to be sequenced typographically. Some of these
``large'' sets can be given by stating an algorithm\footnote{In the informal,
mathematical sense of an algorithm, as in e.g. Egyptian multiplication.}, by
which the elements can be sequenced at the reader's leisure.  Every element of
such a set will eventually appear in the sequence, but there need not be a
point at which we have sequenced all the elements of the set.

One way of specifying such an algorithm is to typographically sequence the
first couple of elements, and to terminate this sequence by an ellipses
($\ldots$) We then rely on the reader's intuition to deduce how the subsequent
elements can be sequenced one-by-one\footnote{Here we ignore Wittgenstein's
rule-following paradox.}.

For instance, we define the set of natural numbers as follows:

\begin{definition} The set of \textbf{natural numbers} is

$$\mathbb{N} \triangleq \set{0,1,2,3,4,5,6,7,8,9,10,\ldots}.$$

\end{definition}

As we shall later prove, there are sets which have too many elements to be
sequenced by an algorithm.  Such sets will be given by a rule regarding what
constitutes an element of the set, relying ever more on the reader's intuition.
A set like this is written $\set{x\st{\text{rule regarding $x$}}}$, read ``all
$x$ such that the rule regarding $x$ holds''.

For instance, we define the set of real numbers as follows:

\begin{definition} The set of \textbf{real numbers} is

$$\mathbb{R}\triangleq\set{x\st{\text{$x$ is value along a continuous
line}}}.$$

\end{definition}

We will use typical mathematical notation to denote the elements of
$\mathbb{R}$. That is, fractions or adequate decimal expansions for the
\emph{rational numbers}, and known mathematical functions for the
\emph{irrational numbers}, e.g. $\sqrt{2}$, $\pi$, $\phi$. When adequate,
an irrational number will be given by a rational approximation.

\begin{notation} By $x\in X$, read (foreach) ``$x$ in $X$'', (if) ``$x$ is in
$X$'', or (let) ``$x$ be in $X$'',  for some set $X$, we mean that $x$ shall
henceforth refer to an element of $X$, if any. \end{notation}

% Russel's paradox comes after functions, predicates and the last step of
% set-builder notation with such that rules.

\subsection{Functions}

For the lack of an axiomatic approach to the notion of sets, we also take a
notion of functions as given, subject to the following specification:

\begin{specification} A \textbf{function} is either

\begin{enumerate}

\item a \textbf{partial function} $f$ with a \textbf{domain} $X$, and
\textbf{codomain} $Y$, written $f:X\rightharpoonup Y$, where $X$ and $Y$ are
sets, and

\begin{enumerate}

\item for each $x\in X$, $f$ either has a \textbf{value} at $x$ in $Y$, written
$f\p{x}\in Y$, or $f\p{x}$ is undefined;

\item $X$, $Y$, and $f\p{x}$ for each $x\in X$ are completely determined by $f$;

\item $X$, $Y$, and $f\p{x}$ for some $x\in X$ are completely determine $f$; or

\end{enumerate}

\item a \textbf{constant} $\overline{x}$, where $x\in X$, for some nonempty set
$X$, written $\overline{x}:X$, with domain $X$, written $x:X$, where $X$ is a
nonempty set.

\end{enumerate}

\end{specification}

\begin{definition} A function is \textbf{total}, if it is either a constant, or
a partial function $f:X\rightharpoonup Y$, where $f\p{x}$ is defined for all
$x\in X$, written $f:X\rightarrow Y$.\end{definition}

For the lack of a formal definition of either sets or partial functions, but in
tune with the specifications of these notions above, we can regard partial
functions as sets. In such an interpretation, the codomain (and domain) of a
partial function may itself be a partial function. This leads way to some
convenient notation:

\begin{notation} A partial function $f:X\rightharpoonup G$, where $G$ is a
function $g:Y\rightharpoonup Z$ is also written $f:X\rightharpoonup
\p{Y\rightharpoonup Z}$, or simply $f:X\rightharpoonup Y\rightharpoonup Z$.
Similarly for total functions.  \end{notation}

\def\codomain#1{co\textsuperscript{$#1$}domain}

\begin{notation} For a function $f:X_0\rightharpoonup X_1 \rightharpoonup
\cdots \rightharpoonup X_n \rightharpoonup Y$, for some $n\in \mathbb{N}$, we
say $Y$ is the \codomain{n} of $f$, and $f$ has $n$ \textbf{arguments}, or $f$
is an $n$-argument function, and furthermore

\begin{itemize}

\item $f$ is \textbf{nullary} if it has $0$ arguments;

\item $f$ is \textbf{unary} if it has $1$ argument;

\item $f$ is \textbf{binary} if it has $2$ arguments.

\end{itemize}

\end{notation}

\begin{definition} A \textbf{relation} is a total binary function with a
\codomain{2} $\mathbb{B}$. \end{definition}

\begin{notation} When denoted by a symbol, the application of a relation will
sometimes be denoted in infix notation. \end{notation}

\begin{notation} By $=$, we denote the \textbf{equality relation}.
\end{notation}

 A partial function can hereby be given by a set of mappings,
where a mapping from $x$ to $f\p{x}$ is written $x\mapsto f\p{x}$.

\begin{definition} A function $f:X\rightarrow Y$ is a \textbf{bijection}, or
``is bijective'', iff there exists a function $g:Y\rightarrow X$ such that for
each $x\in X$, $g\p{f\p{\overline{x}}}=\overline{x}$. \end{definition}

\subsection{Cardinality}

The \textbf{cardinality} of a set $X$, written $\card{X}$, is a measure of its
size. Two sets $X$ and $Y$ have the same cardinality, written
$\card{X}=\card{Y}$, if there exists a bijective function $f:X\rightarrow Y$.

\begin{notation} When we consider an element $a$ of a set $A$, we write $a\in
A$. \end{notation}

\begin{definition} The \textbf{union} of sets $A$ and $B$, written $A\cup B$,
is the set of all $a\in A$ and all $b\in B$. \end{definition}

\begin{specification} A \textbf{property} is a mathematical statement that is
either true or false. \end{specification}

\begin{definition} A set $X$, is \textbf{enumerable} iff
$\card{X}\leq\card{\mathbb{N}}$. \end{definition}

%\begin{definition}

%A \textbf{language} $\mathcal{L}=\chev{\mathcal{C},\mathcal{F}, \mathcal{R}}$
%is given by specifying a set of constant symbols $\mathcal{C}$, a set of
%function symbols $\mathcal{F}$, and a set of relation symbols $\mathcal{R}$.

%\end{definition}

% a sequence is something formed by the concatenation of elements.

%\begin{definition}

%A \textbf{string} is a finite sequence of elements drawn from an alphabet.

%\end{definition}

%\begin{definition}

%A \textbf{string language} is a set of \textbf{strings}.

%\end{definition}

%..

%\begin{definition}

%A \textbf{closure} of $A$ under the operations $O$, is the smallest class $C$,
%containing $A$, and such that the operations of $O$, operating on elements of
%$C$ yield elements of $C$.

%\end{definition}

%\begin{definition}

%A \textbf{function algebra} is a closure of a class $A$.

%\end{definition}

%\begin{definition}

%A characteristic function 

%\end{definition}

\subsection{Tuples}

\begin{notation} We denote a tuple by listing the elements separated by commas,
and enclosed in parentheses. We use this notation out of inspiration from
Haskell syntax. We can regard this as in line with regular mathematical
notation as the arguments to a non-curried function are typically passed in
parentheses. That is unless a more convenient notation is given for particular
kinds of tuples, e.g. the configurations of a Turing machine.\end{notation}

\subsection{Strings and Languages}

\begin{definition} An \textbf{alphabet} is a finite, nonempty set of
\textbf{symbols}.  \end{definition}

\begin{definition} A \textbf{string} over an alphabet is a finite sequence of
symbols from that alphabet. \end{definition}

%\begin{notation} We take $b\notin A$ to mean the same as $\set{b}\cup A =
%\emptyset$. \end{notation}

