\chapter{Implicit Computational Complexity}

% not really aka, but in the spirit of the previous chapter, Implicit
% Complexity, and hence Implicit Complexity Theory

% Also, better suited as an introduction as it doesn't really say anything
% about the choice of polynomial time from the get go. That's more a
% coincidence, the following definitions will hold regardless. Perhaps it is
% worth it calling this a Part I: Introduction and Background

% Fragment of a logic means that we are restricted to a subset of the syntax,
% but retain the same semantics. If we choose the combination of words -
% subsystem of system, we are perhaps a little more free, but in either case,
% we probably have to further specify what we keep and what we take out.

% for the purposes of this thesis, we may regard logical systems equivalent to
% programming languages - draw inspiration from Types of Crash Prevention - it
% has a nice, human readable introduction to the whole thing.

The correspondence between an ICC system and a complexity class is extensional,
i.e. the class of function (or problems) representable in the system equals the
complexity class.

% by this point we should have more precisely defined the notions of function
% and problem.

The systems that we will consider here will be subsystems of a larger base
system, in which other functions, besides those in the complexity class of the
system can be represented.
