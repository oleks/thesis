\section{Mathematical Foundations}

\label{sec:preface:mathematical-foundations}

\def\strue{\ensuremath{\text{\textsc{True}}}}
\def\sfalse{\ensuremath{\text{\textsc{False}}}}

It is beyond the scope of thesis to justify a choice of mathematical
foundations other than \emph{classical mathematics}. This is the sort of
mathematics often taught prior to higher education, and throughout. The reader
is assumed to be familiar with this style of mathematics. This choice is also
natural as most related literature assumes a classical foundation.

This is certainly not the only possible foundation of mathematics, as advocated
by various schools of \emph{constructive mathematics}. In a nutshell,
constructive mathematics takes the stance that a proof that a mathematical
object exists, constitutes specifying a way of (mentally) constructing that
object.

Although we assume a classical foundation, we will seek to keep ourselves to
constructive arguments --- the approach often taken in related literature.

\subsection{Sets}

It is beyond the scope of this thesis to develop an axiomatic theory of sets.
We assume a notion of sets as given, subject to the following specification:

\begin{specification} A \textbf{set} is a mathematical object that is distinct
from, but completely determined by its \textbf{elements}. \end{specification}

\begin{notation} By $X\triangleq Y$, read ``$X$ is by definition the same as
$Y$'', we mean that $X$ is henceforth interchangeable with $Y$, until $X$
occurs on the left side of another $\triangleq$, that is, until $X$ is
redefined. \end{notation}

A set is given by characterising its elements, enclosed in braces. For
instance, the simplest set is the empty set, having no elements. We define the
empty set as follows:

% Note to self: we don't start with sequences as sets will also be defined by
% other means than sequences.

\begin{definition} The \textbf{empty set}, written $\emptyset$, is

$$\emptyset\triangleq\set{}.$$

\end{definition}

Some ``small'' sets can be given by denoting their elements in a typographical
sequence, separated by commas. For instance, we define the set of boolean
values as follows:

\begin{definition} The set of \textbf{boolean values} is

$$\mathbb{B}\triangleq\set{\strue,\sfalse}$$

\end{definition}

Some sets have too many elements to be sequenced typographically. Some of these
``large'' sets can be given by a sequencing algorithm. In
\refSection{background-computability} we will formalise the notion of an
algorithm. For now, we appeal to the less formal, ellipses notation. For
instance, we define the set of natural numbers as follows:

\begin{definition} The set of \textbf{natural numbers} is

$$\mathbb{N} \triangleq \set{0,1,2,3,4,5,6,7,8,9,10,\ldots}.$$

\end{definition}

As we shall later prove, there are sets which have too many elements to be
sequenced by an algorithm.  Such sets will be given by a rule regarding what
constitutes an element of the set, relying ever more on the reader's intuition.
A set like this is written $\set{x\st{\text{rule regarding $x$}}}$, read ``all
$x$ such that the rule regarding $x$ holds''. For instance, we define the set
of real numbers as follows:

\begin{definition} The set of \textbf{real numbers} is

$$\mathbb{R}\triangleq\set{x\st{\text{$x$ is value along a continuous
line}}}.$$

\end{definition}

We will use typical mathematical notation to denote the elements of
$\mathbb{R}$. That is, fractions or adequate decimal expansions for the
\emph{rational numbers}, and known mathematical functions for the
\emph{irrational numbers}, e.g. $\sqrt{2}$, $\pi$, $\phi$. When adequate,
an irrational number will be given by a rational approximation.

\begin{notation} By $x\in X$, read ``$x$ in $X$'', (if) ``$x$ is in $X$'', or
(let) ``$x$ be in $X$'',  for some set $X$, we mean that $x$ shall henceforth
refer to an element of $X$, if any. \end{notation}

% Russel's paradox comes after functions, predicates and the last step of
% set-builder notation with such that rules.

\subsection{Tuples}

\begin{definition} An $n$-\textbf{tuple} over some given sets $X_0$, $X_1$,
\ldots, $X_n$, where $n\in\mathbb{N}$, is a sequence of $n$ elements enclosed
in parentheses, $\p{x_0,x_1,\ldots,x_n}$, where $x_i\in X_i$ for all
$i\in\set{0,1,\ldots, n}$. \end{definition}

\begin{definition} The \textbf{Cartesian product} of the sets $X_0$, $X_1$,
\ldots, $X_n$, for some $n\in\mathbb{N}$, is the set of all $n$-tuples over the
sets:

$$X_0\times X_1\times \cdots \times X_n \triangleq
\set{\p{x_0,x_1,\ldots,x_n}\st{ \text{$x_i\in X_i$ for all
$i\in\set{0,1,\ldots, n}$}}}.$$

\end{definition}

\begin{notation} We denote a tuple by listing the elements separated by commas,
and enclosed in parentheses. We use this notation out of inspiration from
Haskell syntax. We can regard this as in line with regular mathematical
notation as the arguments to a non-curried function are typically passed in
parentheses. That is unless a more convenient notation is given for particular
kinds of tuples, e.g. the configurations of a Turing machine.\end{notation}

\subsection{Functions}

For the lack of an axiomatic approach to the notion of sets, we also take a
notion of functions as given, subject to the following specification:

\begin{specification} A \textbf{function} is either

\begin{enumerate}

\item a \textbf{partial function} $f$ with a \textbf{domain} $X$, and
\textbf{codomain} $Y$, written $f:X\rightharpoonup Y$, where $X$ and $Y$ are
sets, and

\begin{enumerate}

\item for each $x\in X$, $f$ either has a \textbf{value} at $x$ in $Y$, written
$f\p{x}\in Y$, or $f\p{x}$ is undefined;

\item $X$, $Y$, and $f\p{x}$ for each $x\in X$, are completely determined by $f$;

\item $X$, $Y$, and $f\p{x}$ for some $x\in X$, are completely determine $f$; or

\end{enumerate}

\item a \textbf{constant} $\overline{x}$, where $x\in X$, for some nonempty set
$X$, written $\overline{x}:X$, with domain $X$, written $x:X$, where $X$ is a
nonempty set.

\end{enumerate}

\end{specification}

\begin{definition} A function is \textbf{total}, if it is either a constant, or
a partial function $f:X\rightharpoonup Y$, where $f\p{x}$ is defined for all
$x\in X$, written $f:X\rightarrow Y$.\end{definition}

For the lack of a formal definition of either sets or partial functions, but in
tune with the specifications of these notions above, we can regard partial
functions as sets. In such an interpretation, the codomain (and domain) of a
partial function may itself be a partial function. This leads way to some
convenient notation:

\begin{notation} A partial function $f:X\rightharpoonup G$, where $G$ is a
function $g:Y\rightharpoonup Z$ is also written $f:X\rightharpoonup
\p{Y\rightharpoonup Z}$, or simply $f:X\rightharpoonup Y\rightharpoonup Z$.
Similarly for total functions.  \end{notation}

\def\codomain#1{co\textsuperscript{$#1$}domain}

\begin{notation} For a function $f:X_0\rightharpoonup X_1 \rightharpoonup
\cdots \rightharpoonup X_n \rightharpoonup Y$, for some $n\in \mathbb{N}$, we
say $Y$ is the \codomain{n} of $f$, and $f$ has $n$ \textbf{arguments}, or $f$
is an $n$-argument function, and furthermore

\begin{itemize}

\item $f$ is \textbf{nullary} if it has $0$ arguments;

\item $f$ is \textbf{unary} if it has $1$ argument;

\item $f$ is \textbf{binary} if it has $2$ arguments.

\end{itemize}

\end{notation}

\begin{definition} A \textbf{relation} is a total binary function with a
\codomain{2} $\mathbb{B}$. \end{definition}

\begin{notation} When denoted by a symbol, the application of a relation will
sometimes be denoted in infix notation. \end{notation}

\begin{notation} By $=$, we denote the \textbf{equality relation}.
\end{notation}

 A partial function can hereby be given by a set of mappings,
where a mapping from $x$ to $f\p{x}$ is written $x\mapsto f\p{x}$.

\begin{definition} A function $f:X\rightarrow Y$ is a \textbf{bijection}, or
``is bijective'', iff there exists a function $g:Y\rightarrow X$ such that for
each $x\in X$, $g\p{f\p{\overline{x}}}=\overline{x}$. \end{definition}

\subsection{Cardinality}

The \textbf{cardinality} of a set $X$, written $\card{X}$, is a measure of its
size. Two sets $X$ and $Y$ have the same cardinality, written
$\card{X}=\card{Y}$, if there exists a bijective function $f:X\rightarrow Y$.

\begin{notation} When we consider an element $a$ of a set $A$, we write $a\in
A$. \end{notation}

\begin{definition} The \textbf{union} of sets $A$ and $B$, written $A\cup B$,
is the set of all $a\in A$ and all $b\in B$. \end{definition}

\begin{specification} A \textbf{property} is a mathematical statement that is
either true or false. \end{specification}

\begin{definition} A set $X$, is \textbf{enumerable} iff
$\card{X}\leq\card{\mathbb{N}}$. \end{definition}

%\begin{definition}

%A \textbf{language} $\mathcal{L}=\chev{\mathcal{C},\mathcal{F}, \mathcal{R}}$
%is given by specifying a set of constant symbols $\mathcal{C}$, a set of
%function symbols $\mathcal{F}$, and a set of relation symbols $\mathcal{R}$.

%\end{definition}

% a sequence is something formed by the concatenation of elements.

%\begin{definition}

%A \textbf{string} is a finite sequence of elements drawn from an alphabet.

%\end{definition}

%\begin{definition}

%A \textbf{string language} is a set of \textbf{strings}.

%\end{definition}

%..

%\begin{definition}

%A \textbf{closure} of $A$ under the operations $O$, is the smallest class $C$,
%containing $A$, and such that the operations of $O$, operating on elements of
%$C$ yield elements of $C$.

%\end{definition}

%\begin{definition}

%A \textbf{function algebra} is a closure of a class $A$.

%\end{definition}

%\begin{definition}

%A characteristic function 

%\end{definition}

\subsection{Strings and Languages}

\begin{definition} An \textbf{alphabet} is a finite, nonempty set of
\textbf{symbols}.  \end{definition}

\begin{definition} A \textbf{string} over an alphabet is a finite sequence of
symbols from that alphabet. \end{definition}

%\begin{notation} We take $b\notin A$ to mean the same as $\set{b}\cup A =
%\emptyset$. \end{notation}
