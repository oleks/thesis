\section{Mathematical Foundations}

\label{sec:preface:mathematical-foundations}

\def\strue{\ensuremath{\text{\textsc{True}}}}
\def\sfalse{\ensuremath{\text{\textsc{False}}}}

It is beyond the scope of this thesis to justify a choice of mathematical
foundations other than \emph{classical mathematics}. This is the sort of
mathematics often taught prior to higher education, and throughout. The reader
is assumed to be familiar with this style of mathematics. This choice is
natural as a lot of related literature assumes a classical foundation.

This is certainly not the only possible foundation of mathematics, as advocated
by various schools of \emph{constructive mathematics}. In a nutshell,
constructive mathematics takes the stance that a proof that a mathematical
object exists, constitutes specifying a way of (mentally) constructing that
object.

Although we assume a classical foundation, we will seek to keep ourselves to
constructive arguments --- the approach often taken in related literature. This
is however, not the case for this chapter in general.

\subsection{Sets}

It is beyond the scope of this thesis to develop an axiomatic theory of sets.
We assume the concept as given, subject to the following specification:

\begin{specification} \label{spec:set} A \textbf{set} is a mathematical object
that is distinct from, but completely determined by its \textbf{elements}. For
any conceivable set $S$ and object $x$, either the statement $x\in S$, read
``$x$ refers to an element of $S$'', holds, or the statement $x\notin S$, read
``$x$ does not refer to an element of $S$'', does. \end{specification}

% omnipresent universe

A set is given by characterising its elements, enclosed in braces. For
instance, the simplest set is the empty set, having no elements. We define the
empty set as follows:

% Note to self: we don't start with sequences as sets will also be defined by
% other means than sequences.

\begin{definition} The \textbf{empty set} is

$$\emptyset\triangleq\set{}.$$

\end{definition}

Some ``small'' sets can be given by denoting their elements in a typographical
sequence, separated by commas. For instance, we define the set of boolean
values as follows:

\begin{definition} The set of \textbf{boolean values} is

$$\mathbb{B}\triangleq\set{\strue,\sfalse}$$

\end{definition}

Some sets have too many elements to be sequenced typographically. Some of these
``large'' sets can be given by a sequencing algorithm. In
\refSec{background-computability} we will formalise the notion of an algorithm.
For now, we appeal to the less formal, ellipses notation. For instance, we
define the set of natural numbers as follows:

\begin{definition} The set of \textbf{natural numbers} is

$$\mathbb{N} \triangleq \set{0,1,2,3,4,5,6,7,8,9,10,\ldots}.$$

\end{definition}

\begin{remark} When sequenced typographically, or given by a sequencing
algorithm, the order and repetition of elements in the sequence is
insignificant to the denotation of a set. \end{remark}

For instance, $\set{\strue,\sfalse}$ and $\set{\sfalse, \strue, \sfalse}$
denote the same sets, as do $\set{0,1,2,3,4,5,6,7,8,9,10,\ldots}$ and
$\set{1,0,3,2,5,4,7,6,9,8,11,10,\ldots}$.

As we shall later prove, there are sets which have too many elements to be
sequenced by an algorithm.  Such sets will be given by a rule regarding what
constitutes an element of the set, relying ever more on the reader's intuition.
A set like this is written $\set{x\st{\text{rule regarding $x$}}}$, read ``all
$x$ such that the rule regarding $x$ holds''. For instance, we define the set
of real numbers as follows:

\begin{definition} The set of \textbf{real numbers} is

$$\mathbb{R}\triangleq\set{x\st{\text{$x$ is value along a continuous
line}}}.$$

\end{definition}

We will use typical mathematical notation to denote the elements of
$\mathbb{R}$. That is, fractions or adequate decimal expansions for the
\emph{rational numbers}, and known mathematical functions for the
\emph{irrational numbers}, e.g. $\sqrt{2}$, $\pi$, $\phi$. When adequate,
an irrational number will be given by a rational approximation.

The statement $x \in S$ may be more simply read as ``$x$ in $S$'', or (let)
``$x$ be in $S$'', or (if) ``$x$ is in $S$''. Similarly for $x \notin S$.

\begin{definition} For any set $S$, if $x\in S$, we say that $S$
\textbf{contains} $x$, or $x$ is \textbf{drawn} from $S$.\end{definition}

\begin{definition} It follows from \refSpec{set}, that the sets $S$ and $T$ are
\textbf{equal}, written $S=T$, iff they contain the same elements.
\end{definition}

\begin{remark} We will often exploit $\mathbb{N}$ in ellipses
notation.\end{remark}

For instance, a set of $n$ natural numbers, for some $n\in\mathbb{N}$, may be
given as $\set{1,2,\ldots,n}$. In particular, if $n$ is $5$, this denotes the
set $\set{1,2,3,4,5}$; if $n$ is $0$, this denotes the set $\set{}$.

\begin{notation} If we would like to draw a sequence of elements from a set
$X$, as a shorthand, we sometimes denote a sequence of variables before the
$\in$ sign. \end{notation}

For instance, if we let $x,y,z\in X$, then $x$, $y$, and $z$ now refer to three
arbitrary elements of $X$. In general, if we let $x_1,x_2,\ldots,x_n\in X$, for
some $n\in\mathbb{N}$, then $x_1$, $x_2$, up to $x_n$, refer to $n$ arbitrary
elements of $X$.

\subsubsection{Element equality}

\begin{notation} Two given elements $x\in X$ and $y\in X$, by virtue of
definition, or choice, may refer to the ``same'' element of $X$, written $x=y$,
read ``$x$ is \textbf{equal} to $y$'', for some notion of ``equality''.
Conversely, $x$ and $y$ may refer to ``different'' elements of $X$, written
$x\neq y$, read ``$x$ is \textbf{not equal} to $y$''.\end{notation}

\begin{remark} Some $x\in X$ and $y\in X$ need neither refer to ``same''
nor ``different'' elements. For some sets, the notion of ``equality'' will
remain undefined, or will even be known to be ``undecidable''. \end{remark}

\subsubsection{Set operations}

\begin{definition} The \textbf{union} of sets $S$ and $T$ is

$$S\cup T\triangleq \set{x\st{\text{$x\in S$ or $x\in T$}}}.$$

\end{definition}

\begin{definition} The \textbf{intersection} of sets $S$ and $T$ is

$$S\cap T\triangleq \set{x\st{\text{$x\in S$ and $x\in T$}}}.$$

\end{definition}

\begin{definition} A set $S$ is a \textbf{subset} of a set $T$, written
$S\subseteq T$ if for each $s\in S$, we have $s\in T$. \end{definition}

\begin{definition} A set $S$ is \textbf{elementarily equal} to a set $T$,
written $S=T$, if $S\subseteq T$ and $T\subseteq S$. \end{definition} 

Sometimes, a subset of a set $S$, will be given by predicating the elements of
$X$ with a rule. We write this as $\set{x\in X\st{\text{rule regarding $x$}}}$,
read ``all $x$ in $X$ such that the rule regarding $x$ holds''. For instance,
we define the set minus operation as follows:

\begin{definition} A set $S$ \textbf{minus} a set $T$ is

$$S\setminus T \triangleq \set{s\in S\st{s\notin T}}.$$

\end{definition}

A less formal use of the notation is this:

\begin{definition} The set of \textbf{non-negative} real numbers is

$$\mathbb{R}^+ \triangleq \set{ x\in\mathbb{R} \st{\text{$x$ is greater than or
equal to $0$}}}.$$

\end{definition}

\begin{notation} When denoted by a symbol, the application of a relation will
sometimes be denoted in infix notation. \end{notation}

\begin{notation} By $=$, we denote the \textbf{equality relation}.
\end{notation}

\subsection{Tuples}

\begin{definition} An $n$-\textbf{tuple} over some given sets $X_1$, $X_2$,
\ldots, $X_n$, where $n\in\mathbb{N}$, is a sequence of $n$ elements, written
$\p{x_1,x_2,\ldots,x_n}$, where $x_i\in X_i$ for all $i\in\set{1,2,\ldots, n}$.
\end{definition}

\begin{definition} A $0$-tuple is also called \textbf{unit}, a $1$-tuple
\textbf{singleton}, a $2$-tuple \textbf{pair}, and a 3-tuple \textbf{triple}.
\end{definition}

\begin{definition} The \textbf{cartesian product} of the sets $X_1$, $X_2$,
\ldots, $X_n$, for some $n\in\mathbb{N}$, is the set of all $n$-tuples over the
given sets:

$$X_1\times X_2\times \cdots \times X_n \triangleq
\set{\p{x_1,x_2,\ldots,x_n}\st{ \text{$x_i\in X_i$ for all
$i\in\set{1,2,\ldots, n}$}}}.$$

\end{definition}

\begin{remark} The empty cartesian product (when $n=0$) is unit. \end{remark} 

% Russel's paradox comes after functions, predicates and the last step of
% set-builder notation with such that rules.

\begin{definition} A \textbf{relation} $R$, between sets $S$ and $T$, is a
subset of the cartesian product of $S$ and $T$, that is $R \subseteq S\times
T$. \end{definition}

% \begin{definition} A \textbf{diagonal relation} $\Delta_S$, is a relation
% between a set $S$ and itself, i.e. $\Delta_S \subseteq S \times S$.
% \end{definition}

\begin{definition} A relation $R$ is \textbf{single-valued} iff $\p{x,y}\in R$
and $\p{x,z}\in R$ implies that $y=z$. \end{definition}

\subsection{Set Functions}

For the lack of an axiomatic approach to the concept of sets, and because we
will consider several different ``implementations'' of functions, we take the
concept of functions as given, subject to the following specification:

\begin{specification} \label{spec:function} A \textbf{total function}, or
simply \textbf{function}, $f$, is a mathematical object with \textbf{domain}
$S$, and \textbf{codomain} $T$, written $f : S\rightarrow T$, where

\begin{enumerate}

\item [F-1.] $S$ and $T$ are sets;

\item [F-2.] for each $s \in S$, $f$ has a (unique) \textbf{value} at $s$ in
$T$, written $f\p{s}\in T$;

\item [F-3.] for any $s,t \in S$, if $s = t$, then $f\p{s} = f\p{t}$;

\item [F-4.] $S$, $T$, and $f\p{s}$ for each $s \in S$, are completely
determined by $f$;

\item [F-5.] $S$, $T$, and $f\p{s}$ for each $s \in S$, completely determine
$f$.

\end{enumerate}

\end{specification}

Having functions at hand, we can now expand on the notion of sets:

\begin{definition} \label{def:subset} A set $S$ is a \textbf{subset} of a set
$T$, written $S \subseteq T$, iff there is an \textbf{inclusion function}, $i :
S \rightarrow T$, such that $i\p{s}=s$ for all $s \in S$. \end{definition}

\begin{definition} \label{def:seteq} A set $S$ is \textbf{equal} to a set $T$,
written $S=T$, iff $S\subseteq T$ and $T\subseteq S$. \end{definition}

Having the notion of equality for sets, functions can now be composed:

\begin{definition} \label{def:compose} Given two functions $f : S \rightarrow
T$ and $g : U \rightarrow V$, if $T = U$, let the \textbf{composite function},
written $\p{g\circ f} : S \rightarrow V$, be the function such that $\p{g\circ
f}\p{s}=g\p{f\p{s}}$ for all $s \in S$.  \end{definition}

\begin{notation} The symbol $\circ$ is read ``after''.\end{notation}

% equality of functions

\begin{definition} \label{def:graph} The \textbf{graph} of a function $f : S
\rightarrow T$ is

$$G_f\triangleq\set{\p{s,t}\st{\text{$t=f\p{s}$ for some $s\in S$}}}.$$

\end{definition}

The graph of a function is a single-valued relation, with the
\textbf{functional property}, i.e. for each $s\in S$, we have $\p{s,f\p{s}} \in
G_f$. Although by collecting the first components of the pairs in $G_f$, we get
$S$ --- by collecting the second components, we \emph{do not necessarily} get
$T$.  The graph $G_f$ is insufficient to implement the function $f :
S\rightarrow T$.  One common implementation is therefore the triple
$\p{S,T,G_f}$. In general however, this leads to the following theorem:

\begin{theorem} \label{thm:codomain-subtyping} If there exists a function $f :
S \rightarrow T$, where $T\subseteq U$, then there exists a function $g : S
\rightarrow U$, such that $f\p{s}=g\p{s}$ for all $s\in S$. \end{theorem}

\begin{proof} By \refDef{subset}, we have an inclusion function $i : T
\rightarrow U$. Let $g \triangleq \p{i\circ f}$. \end{proof}

\begin{definition} \label{def:id-fun} Given a set $S$, let the \textbf{identity
function}, written $id_S: S \rightarrow S$, be the function such that
$id_S\p{s}=s$ for all $s\in S$. \end{definition}

\begin{definition} \label{def:injective} A function $f : S \rightarrow T$ is
\textbf{injective} if $f\p{s}=f\p{t}$ iff $s=t$ for any $s,t\in S$.
\end{definition}

\begin{corollary} \label{cor:id-is-injective} The identity function of any set
$S$ is injective.  \end{corollary}

\begin{proof} Follows directly from \refDef{id-fun} and \refDef{injective}.
\end{proof}

\begin{definition} \label{def:image} The \textbf{image} of a function $f : S
\rightarrow T$, is

$$I_f \triangleq \set{t\st{\text{$t=f\p{s}$ for some $s\in S$}}}.$$

\end{definition}

It is perhaps worth noting why we don't just define the function $f : S
\rightarrow T$ as the triple $\p{S, T, G_f}$.  One reason will become apparent
in \refSec{background-what-tm-computes}, where we observe that a Turing machine
\emph{implements} a partial function (specified below) for particular kinds of
sets.

\begin{specification} \label{spec:partial-function} A \textbf{partial function}
$f_\bot$, with domain $S$ and codomain $U$, written $f_\bot : S \rightharpoonup
U$, is a (total) function $f : T \rightarrow U$, where $T\subseteq S$. We say
that $T$ is the \textbf{domain of definition} of $f_\bot$, and $f_\bot\p{s}$
is \textbf{undefined} for all $s \in S \setminus T$.\end{specification}

\begin{remark} For some partial functions, the domain of definition, or the
underlying total function may be unknown. \end{remark}

\begin{theorem} \label{thm:total-has-partial} For every (total) function $f : S
\rightarrow T$, there is a partial function $g : S \rightharpoonup T$ with the
domain of definition $S$.  \end{theorem}

\begin{proof} Follows from \refSpec{partial-function}, as for any $S$, we have
$S\subseteq S$.  \end{proof}

We give the notion of partial function by specification rather than by
definition to leave the notion of ``undefined'' undefined, until we have an
implementation of (total) functions at hand.

For instance, if we consider the implementation of (total) functions by
triples, we can implement partial functions by dropping the functional
requirement from the graph in the triple. In this implementation, $f_\bot\p{s}$
is undefined for all those $s\in S$, for which $s$ does not occur as the first
component of any pair in $G_{f_\bot}$. This way, not only do we not necessarily
get $Y$ by collecting the second components of the pairs in $G_{f_\bot}$, we
also \emph{do not necessarily} get $X$ by collecting the first components of
$G_{f_\bot}$.

\begin{definition} The graph $G_{f_\bot}$, of a partial function $f_\bot : S
\rightharpoonup T$, is the set of pairs

$$G_{f_\bot} \triangleq \set{\p{x,f_\bot\p{x}}\in X \times Y}.$$

\end{definition}

For similar reasons, we leave the notion of composition of partial functions as
a specification.

\begin{specification} \label{spec:partial-composition} Given two partial
functions $f_\bot : S \rightharpoonup T$ and $g_\bot : T \rightharpoonup U$,
with domains of definition of $S'$, and $T'$, respectively, let the composite
partial function, written $\p{g_\bot \circ f_\bot} : S \rightharpoonup U$, be
the partial function such that $\p{g \circ f}\p{s} = g\p{f\p{s}}$ for all $s\in
S'$ where $f\p{s} \in T'$, and $\p{g \circ f}\p{s}$ is \textbf{undefined}
otherwise.  \end{specification}

% \begin{definition} \label{def:codomain-coersion} A function $f:S\rightarrow
% T$ may be \textbf{coerced} to a function $g:S\rightarrow U$ if $T\subseteq
% U$. We write $f \leq g$.  \end{definition}

% For instance, a function $f:S\rightarrow T$ implemented as the triple
% $\p{S,T,G_f}$, can be coerced to a function $g:S\rightarrow U$, by replacing
% $T$ with $U$ in the triple to get $\p{S,U,G_f}$. 

For the lack of a formal definition of either sets or functions, but in tune
with the specifications of these notions above, we can regard \emph{functions
as sets}. In such an interpretation, the codomain (and domain) of a function
may itself be a function. This demands some convenient notation:

\begin{notation} A function $f:X\rightarrow G$, where $G$ is a function
$g:Y\rightarrow Z$ is also written $f:X\rightarrow \p{Y\rightarrow Z}$, or
simply $f:X\rightarrow Y\rightarrow Z$. That is, $\rightarrow$ is
right-associative. \end{notation}

% \def\codomain#1{(co)\textsuperscript{$#1$}domain}

%\begin{notation} For a function $f:X_1\rightarrow X_2 \rightarrow \cdots
%\rightarrow X_n \rightarrow Y$, for some $n\in \mathbb{N}$, we say $Y$ is the
%\codomain{n} of $f$, and $f$ has $n$ \textbf{arguments}, or $f$ is an
%$n$-argument function, and furthermore

%\begin{itemize}

%\item $f$ is \textbf{nullary} if it has $0$ arguments;

%\item $f$ is \textbf{unary} if it has $1$ argument;

%\item $f$ is \textbf{binary} if it has $2$ arguments.

%\end{itemize}

%\end{notation}

\begin{definition} \label{def:countable} A set $S$ is \textbf{countable} iff
there exists an injective function $f : S \rightarrow \mathbb{N}$.
\end{definition}

\begin{definition} \label{def:surjective} A function $f:S\rightarrow T$ is
\textbf{surjective} iff for each $t\in T$ there is a corresponding $s\in S$,
such that $f\p{s}=t$.  \end{definition}

\begin{definition} \label{def:bijective} A function $f : S \rightarrow T$ is
\textbf{bijective}, iff $f$ is both injective and surjective, or (equivalently)
there exists a function $g : T \rightarrow S$, such that $g\p{f\p{s}}=s$ for
all $s\in S$, and $f\p{g\p{t}}=t$ for all $t\in T$.\end{definition}

\begin{theorem} \label{thm:injective-compose} An injective function $f : S
\rightarrow T$, composed with an injective function $g : T \rightarrow U$,
written $\p{g \circ f} : S \rightarrow U$ is also injective. \end{theorem}

\begin{proof} By \refDef{compose}, we have $\p{g \circ f}\p{s}=g\p{f\p{s}}$ for
all $s \in S$. By \refDef{injective}, we have $f\p{s}=f\p{t}$ iff $s=t$ for any
$s,t\in S$, and $g\p{u}=g\p{v}$ iff $u=v$ for any $u,v \in T$. By transitivity,
we must have $g\p{f\p{s}}=g\p{f\p{t}}$ iff $f\p{s}=f\p{t}$ iff $s=t$ for any
$s,t \in S$.\end{proof}

\begin{theorem} \label{thm:surjective-compose} A surjective function $f : S
\rightarrow T$, composed with a surjective function $g : T \rightarrow U$,
written $\p{g \circ f} : S \rightarrow U$ is also surjective. \end{theorem}

\begin{proof} By \refDef{compose}, we have $\p{g \circ f}\p{s}=g\p{f\p{s}}$ for
all $s \in S$. By \refDef{surjective}, for each $t \in T$, there is a
corresponding $s \in S$, such that $f\p{s} = t$, and for each $u \in U$, there
is a corresponding $t \in T$, such that $g\p{t} = u$. By transitivity, for each
$u \in U$, there is a corresponding $t \in T$, for which there is a
corresponding $s \in S$, such that $g\p{f\p{s}} = g\p{t} = u$.\end{proof}

\begin{theorem} \label{thm:bijective-compose} A bijective function $f : S
\rightarrow T$, composed with a bijective function $g : T \rightarrow U$,
written $\p{g \circ f} : S \rightarrow U$, is also bijective.
\end{theorem}

\begin{proof} By \refDef{bijective}, $f$ and $g$ are both injective and
surjective. By \refThm{injective-compose} and \refThm{surjective-compose},
$\p{g \circ f}$ is both injective and surjective, and so similarly, also
bijective. \end{proof}

\begin{definition} \label{def:countably-infinite} A set $S$ is
\textbf{countably infinite} iff there exists a bijective function $f : S
\rightarrow \mathbb{N}$, or (equivalently) a bijective function $g : \mathbb{N}
\rightarrow S$. \end{definition}

\begin{theorem} \label{thm:countable-to-countably-infinite} If $S$ is a
countable set, then there exists an injective function $f : S \rightarrow T$,
where $T$ is a countably infinite set.  \end{theorem}

\begin{proof} By \refDef{countable}, there is an injective function $g : S
\rightarrow \mathbb{N}$. By \refDef{countably-infinite}, there is a bijective
function $h : \mathbb{N} \rightarrow T$, which by \refDef{bijective} is also
injective. Let $f \triangleq \p{h \circ g}$. By \refThm{injective-compose}, $f$
is also injective. \end{proof}

\begin{theorem} \label{thm:countably-infinite-to-countably-infinite} If $S$ and
$T$ are a countably infinite sets, then there exists a bijective function $f :
S \rightarrow T$, or (equivalently) a bijective function $g : T \rightarrow S$.
\end{theorem}

\begin{proof} By \refDef{countably-infinite}, there is a bijective function $g
: S \rightarrow \mathbb{N}$, and a bijective function $h : \mathbb{N}
\rightarrow T$. Let $f \triangleq \p{h \circ g}$. By
\refThm{bijective-compose}, $f$ is also bijective.\end{proof}

\begin{definition} \label{def:id-function} The \textbf{identity function} for a
set $S$, written $id_S : S\rightarrow S$, is a function such that $id\p{s}=s$
for any $s\in S$. \end{definition}

\begin{corollary} An identity function is injective. \end{corollary}

\begin{corollary} The set $\mathbb{N}$ is countable. \end{corollary}

\begin{theorem} \label{thm:subset-of-countably-infinite} If $S\subset T$, for
some countably infinite set $T$, $S$ must either be countable or countably
infinite. \end{theorem}

\begin{proof} Requires an inclusion function from $S$ to $T$, which is
injective, and done by transitivity. \end{proof}

\subsection{Cardinality}

\begin{notion} The \textbf{cardinality} of a set $S$, written $\card{S}$, is a
measure of its size. \end{notion}

\begin{definition} \label{def:same-card} Two sets $S$ and $T$ have the same
cardinality, written $\card{S}=\card{T}$, iff there exists a bijective function
$f : S\rightarrow T$.\end{definition}

\begin{definition} \label{def:leq-card} A set $S$ has a cardinality less than
or equal to the cardinality of a set $T$, written $\card{S}\leq\card{T}$, iff
there exists an injective function $f : S \rightarrow T$.\end{definition}

\begin{theorem} \label{thm:injective-image-card} If $f : S \rightarrow T$ is an
injective function, then $\card{I_f}=\card{S}$. \end{theorem}

\begin{proof} Proof by contradiction. Let $\card{I_f} \neq \card{S}$. We
must have one of two cases: 

\begin{enumerate}

\item There is an $s\in S$ such that $f\p{s}\notin f\seq{S}$, but this
contradicts the assumption that $f\seq{S}$ forms the image of $f$.

\item There is a $f\p{s}\in f\seq{S}$ such that $s\notin S$, which again
contradicts the assumption that $f\seq{S}$ forms the image of $f$. 

\end{enumerate}

\end{proof}

\begin{definition} \label{def:card-leq} A set $X$ has a cardinality less than
or equal to the cardinality of set $Y$, written $\card{X}\leq\card{Y}$, if
there exists an injective function $f:X\rightarrow Y$. \end{definition}

\begin{theorem}\label{thm:card-enumerable} A set $X$ is enumerable iff
$\card{X}\leq\card{\mathbb{N}}$.  \end{theorem}

\begin{proof} Follows directly from \refDef{enumerable} and \refDef{card-leq}.
\end{proof}



%\begin{definition}

%A \textbf{language} $\mathcal{L}=\chev{\mathcal{C},\mathcal{F}, \mathcal{R}}$
%is given by specifying a set of constant symbols $\mathcal{C}$, a set of
%function symbols $\mathcal{F}$, and a set of relation symbols $\mathcal{R}$.

%\end{definition}

% a sequence is something formed by the concatenation of elements.

%\begin{definition}

%A \textbf{string} is a finite sequence of elements drawn from an alphabet.

%\end{definition}

%\begin{definition}

%A \textbf{string language} is a set of \textbf{strings}.

%\end{definition}

%..

%\begin{definition}

%A \textbf{closure} of $A$ under the operations $O$, is the smallest class $C$,
%containing $A$, and such that the operations of $O$, operating on elements of
%$C$ yield elements of $C$.

%\end{definition}

%\begin{definition}

%A \textbf{function algebra} is a closure of a class $A$.

%\end{definition}

%\begin{definition}

%A characteristic function 

%\end{definition}

\subsection{Strings and Languages}

\begin{definition} An \textbf{alphabet} is a finite, nonempty set of
\textbf{symbols}. \end{definition}

\begin{notation} We will typically use the symbol $\Sigma$ to denote an
alphabet. \end{notation}

\begin{definition} A \textbf{string} over an alphabet is a finite sequence of
symbols from that alphabet. \end{definition}

\begin{definition} The set of all strings over an alphabet $\Sigma$ is written
$\Sigma^*$. \end{definition}

\begin{theorem} \label{thm:kleene-star-countably-infinite} The set $\Sigma^*$
is countably infinite. \end{theorem}

\begin{proof} \end{proof}

\section{Categories}

Classical and constructive mathematics are not the only possible foundations
for the theory of computability and complexity.

\emph{Category theory} allows a systematic discussion of the behaviour of a
system in terms of the admissible transformations of system objects, without
dealing in the issues of representation.

\begin{specification} A \textbf{category} $\mathbb{C}$ is a collection of
\textbf{objects} and \textbf{morphisms} between them. A morphism $f$ between
objects $A$ and $B$ is denoted $f:A\rightarrow B$. We say that $A$ is the
\textbf{domain}, and $B$ the \textbf{codomain} of $f$.

For each object $A$, there exists an \textbf{identity morphism}
$id_A:A\rightarrow A$. For any pair of morphisms $f : A\rightarrow B$ and $g :
B \rightarrow C$, there exists a \textbf{composite morphism} $\p{g\circ f} : A
\rightarrow C$.  Furthermore, there is a notion of equality amongst the
morphisms of a category, such that:

\begin{enumerate}

\item [C-1] For any morphism $f:A\rightarrow B$, we have $f\circ id_A = f =
id_B \circ f$.

\item [C-2] For any triple of morphisms $f : A \rightarrow B$, $g : B
\rightarrow C$, and $h : C \rightarrow D$, we have $f\circ \p{g \circ h} =
\p{f\circ g} \circ h$.

\end{enumerate}

\end{specification}

\begin{notation} Except when in conflict with other definitions, we'll use the
alphabet $\mathbb{C},\mathbb{D},\mathbb{E},\ldots,\mathbb{Z}$ to denote
categories, $A,B,C,\ldots,Z$ to denote objects, and $f,g,h,\ldots,z$ to denote
morphisms.  \end{notation}

% order of characters is significant in an alphabet, often we define an order
% on the letters leading to a lexicographical order on strings.

Category theory is sufficient to deduce many, but not all results in the theory
of computability and complexity\cite{di-paola-heller-1987}. The main hurdle to
this is that a category is classically defined with morphisms being
\emph{total}, whereas the notion of computation asks for partiality. That is,
for morphisms $f : A \rightarrow B$ and $g : C \rightarrow D$ to be composable,
we must have $B=C$. If $g$ is partial, it is not defined over the entire domain
$B=C$, and so the ``compositionality'' of these morphisms is questionable.

We still give a basic account of Category theory, as it will never-the-less
prove useful in modeling terminating and otherwise naturally constrained
computation.
