\section{Mathematical Foundations}

\label{sec:preface:mathematical-foundations}

\def\strue{\ensuremath{\text{\textsc{True}}}}
\def\sfalse{\ensuremath{\text{\textsc{False}}}}

It is beyond the scope of this thesis to justify a choice of mathematical
foundations other than \emph{classical mathematics}. This is the sort of
mathematics often taught prior to higher education, and throughout. The reader
is assumed to be familiar with this style of mathematics. This choice is
natural as a lot of related literature assumes a classical foundation.

This is certainly not the only possible foundation of mathematics, as advocated
by various schools of \emph{constructive mathematics}. In a nutshell,
constructive mathematics takes the stance that a proof that a mathematical
object exists, constitutes specifying a way of (mentally) constructing that
object.

Although we assume a classical foundation, we will seek to keep ourselves to
constructive arguments --- the approach often taken in related literature. This
is however, not the case for this chapter in general.

\subsection{Sets}

It is beyond the scope of this thesis to develop an axiomatic theory of sets.
We assume the concept as given, subject to the following specification:

\begin{specification} A \textbf{set} is a mathematical object that is distinct
from, but completely determined by its \textbf{elements}. For any conceivable
set $S$ and object $s$, either the statement $s\in S$, read ``$s$ refers to an
element of $S$'', holds, or the statement $s\notin S$, read ``$s$ does not
refer to an element of $S$'', does. \end{specification}

A set is given by characterising its elements, enclosed in braces. For
instance, the simplest set is the empty set, having no elements. We define the
empty set as follows:

% Note to self: we don't start with sequences as sets will also be defined by
% other means than sequences.

\begin{definition} The \textbf{empty set}, written $\emptyset$, is

$$\emptyset\triangleq\set{}.$$

\end{definition}

Some ``small'' sets can be given by denoting their elements in a typographical
sequence, separated by commas. For instance, we define the set of boolean
values as follows:

\begin{definition} The set of \textbf{boolean values} is

$$\mathbb{B}\triangleq\set{\strue,\sfalse}$$

\end{definition}

Some sets have too many elements to be sequenced typographically. Some of these
``large'' sets can be given by a sequencing algorithm. In
\refSection{background-computability} we will formalise the notion of an
algorithm. For now, we appeal to the less formal, ellipses notation. For
instance, we define the set of natural numbers as follows:

\begin{definition} The set of \textbf{natural numbers} is

$$\mathbb{N} \triangleq \set{0,1,2,3,4,5,6,7,8,9,10,\ldots}.$$

\end{definition}

\begin{remark} When sequenced typographically, or given by a sequencing
algorithm, the order and repetition of elements in the sequence is
insignificant to the denotation of a set. \end{remark}

For instance, $\set{\strue,\sfalse}$ and $\set{\sfalse, \strue, \sfalse}$
denote the same sets, as do $\set{0,1,2,3,4,5,6,7,8,9,10,\ldots}$ and
$\set{1,0,3,2,5,4,7,6,9,8,11,10,\ldots}$.

As we shall later prove, there are sets which have too many elements to be
sequenced by an algorithm.  Such sets will be given by a rule regarding what
constitutes an element of the set, relying ever more on the reader's intuition.
A set like this is written $\set{x\st{\text{rule regarding $x$}}}$, read ``all
$x$ such that the rule regarding $x$ holds''. For instance, we define the set
of real numbers as follows:

\begin{definition} The set of \textbf{real numbers} is

$$\mathbb{R}\triangleq\set{x\st{\text{$x$ is value along a continuous
line}}}.$$

\end{definition}

We will use typical mathematical notation to denote the elements of
$\mathbb{R}$. That is, fractions or adequate decimal expansions for the
\emph{rational numbers}, and known mathematical functions for the
\emph{irrational numbers}, e.g. $\sqrt{2}$, $\pi$, $\phi$. When adequate,
an irrational number will be given by a rational approximation.

\subsubsection{Containment}

The statement $x\in X$ may be more simply as ``$x$ in $X$'', or (let) ``$x$ be
in $X$'', or (if) ``$x$ is in $X$''. Similarly for $x\notin X$.

\begin{definition} If $x\in X$, we say that $X$ \textbf{contains} $x$, or $x$
is \textbf{drawn} from $X$.\end{definition}

\begin{remark} We will often exploit $\mathbb{N}$ in ellipses
notation.\end{remark}

For instance, a set of $n$ natural numbers, for some $n\in\mathbb{N}$, may be
given as $\set{1,2,\ldots,n}$. In particular, if $n$ is $5$, this denotes the
set $\set{1,2,3,4,5}$; if $n$ is $0$, this denotes the set $\set{}$.

\begin{notation} If we would like to draw a sequence of elements from a set
$X$, as a shorthand, we sometimes denote a sequence of variables before the
$\in$ sign. \end{notation}

For instance, if we let $x,y,z\in X$, then $x$, $y$, and $z$ now refer to three
arbitrary elements of $X$. In general, if we let $x_1,x_2,\ldots,x_n\in X$, for
some $n\in\mathbb{N}$, then $x_1$, $x_2$, up to $x_n$, refer to $n$ arbitrary
elements of $X$.

\subsubsection{Element equality}

\begin{notation} Two given elements $x\in X$ and $y\in X$, by virtue of
definition, or choice, may refer to the ``same'' element of $X$, written $x=y$,
read ``$x$ is \textbf{equal} to $y$'', for some notion of ``equality''.
Conversely, $x$ and $y$ may refer to ``different'' elements of $X$, written
$x\neq y$, read ``$x$ is \textbf{not equal} to $y$''.\end{notation}

\begin{remark} Some $x\in X$ and $y\in X$ need neither refer to ``same''
nor ``different'' elements. For some sets, the notion of ``equality'' will
remain undefined, or will even be known to be ``undecidable''. \end{remark}

\subsubsection{Set operations}

\begin{definition} The \textbf{union} of sets $S$ and $T$ is

$$S\cup T\triangleq \set{x\st{\text{$x\in S$ or $x\in T$}}}.$$

\end{definition}

\begin{definition} The \textbf{intersection} of sets $S$ and $T$ is

$$S\cap T\triangleq \set{x\st{\text{$x\in S$ and $x\in T$}}}.$$

\end{definition}

\begin{definition} A set $S$ is a \textbf{subset} of a set $T$, written
$S\subseteq T$ if for each $s\in S$, we have $s\in T$. \end{definition}

\begin{definition} A set $S$ is \textbf{elementarily equal} to a set $T$,
written $S=T$, if $S\subseteq T$ and $T\subseteq S$. \end{definition} 

Sometimes, a subset of a set $S$, will be given by decorating the elements of
$X$ with a rule. We write this as $\set{x\in X\st{\text{rule regarding $x$}}}$,
read ``all $x$ in $X$ such that the rule regarding $x$ holds''. For instance,
we define the set of non-negative real numbers as follows:

\begin{definition} The set of \textbf{non-negative} real numbers is

$$\mathbb{R}^+ \triangleq \set{ x\in\mathbb{R} \st{\text{$x$ is greater than or
equal to $0$}}}.$$

\end{definition}

Perhaps a more eloquent use of the notation is for the definition of the set
minus operation:

\begin{definition} A set $S$, \textbf{minus} a set $T$ is

$$S\setminus T \triangleq \set{s\in S\st{s\notin T}}.$$

\end{definition}

\subsection{Tuples}

\begin{definition} An $n$-\textbf{tuple} over some given sets $X_1$, $X_2$,
\ldots, $X_n$, where $n\in\mathbb{N}$, is a sequence of $n$ elements, written
$\p{x_1,x_2,\ldots,x_n}$, where $x_i\in X_i$ for all $i\in\set{1,2,\ldots, n}$.
\end{definition}

\begin{definition} A $0$-tuple is also called \textbf{unit}, a $1$-tuple
\textbf{singleton}, a $2$-tuple \textbf{pair}, and a 3-tuple \textbf{triple}.
\end{definition}

\begin{definition} The \textbf{cartesian product} of the sets $X_1$, $X_2$,
\ldots, $X_n$, for some $n\in\mathbb{N}$, is the set of all $n$-tuples over the
given sets:

$$X_1\times X_2\times \cdots \times X_n \triangleq
\set{\p{x_1,x_2,\ldots,x_n}\st{ \text{$x_i\in X_i$ for all
$i\in\set{1,2,\ldots, n}$}}}.$$

\end{definition}

\begin{remark} The empty cartesian product is unit. \end{remark} 

% Russel's paradox comes after functions, predicates and the last step of
% set-builder notation with such that rules.

\begin{definition} A \textbf{relation} $R$, between sets $S$ and $T$, is $R
\subseteq S\times T$. \end{definition}

\begin{definition} A \textbf{diagonal relation} $\Delta_S$, is a relation
between a set $S$ and itself, i.e. $\Delta_S \subseteq S \times S$.
\end{definition}

\subsection{Functions on Sets}

For the lack of an axiomatic approach to the concept of sets, we also take a
concept of functions as given, subject to the following specification:

\begin{specification} A \textbf{total function}, or simply \textbf{function},
$f$, is a mathematical object with \textbf{domain} $X$, and \textbf{codomain}
$Y$, written $f:X\rightarrow Y$, where

\begin{enumerate}

\item [F-1.] $X$ and $Y$ are sets;

\item [F-2.] for each $x\in X$, $f$ has a \textbf{value} at $x$ in $Y$, written
$f\p{x}\in Y$;

\item [F-3.] for any $x,y\in X$, if $x=y$, then $f\p{x}=f\p{y}$;

\item [F-4.] $X$, $Y$, and $f\p{x}$ for each $x\in X$, are completely determined
by $f$;

\item [F-5.] $X$, $Y$, and $f\p{x}$ for each $x\in X$, completely determine $f$.

\end{enumerate}

\end{specification}

\begin{definition} Given functions $f:X\rightarrow Y$ and $g:Y\rightarrow Z$,
let the \textbf{composite} function, written $g\circ f : X \rightarrow Z$, be
the function such that $\p{g\circ f}\p{x}=g\p{f\p{x}}$ for all $x\in X$.
\end{definition}

\begin{definition} Given a set $S$, let the \textbf{identity} function, written
$id_S: S \rightarrow S$, be the function such that $id_S\p{s}=s$ for all $s\in
S$. We omit $S$ from $id_S$, when it is clear from context. \end{definition}

\begin{definition} The \textbf{graph} $G_f$, of a function $f:X\rightarrow Y$, is the
set of pairs

$$G_f\triangleq\set{\p{x,f\p{x}}\st{x\in X}}.$$

\end{definition}

The graph of a function is a relation: a subset of the cartesian product
$X\times Y$, with the \textbf{functional property}, i.e. for each $x\in X$, we
have $\p{x,f\p{x}} \in G_f$. Although by collecting the first components of the
pairs in $G_f$, we get $X$, by collecting the second components, we \emph{do
not necessarily} get $Y$. The graph $G_f$ is insufficient to implement a
function $f:X\rightarrow Y$. One common implementation is therefore the triple
$\p{X,Y,G_f}$.

It is perhaps worth noting why we don't just define functions as such triples.
One reason will become apparent in \refSection{background-what-tm-computes},
where we observe that a Turing machine \emph{implements} a partial function, as
defined below:

\begin{definition} A \textbf{partial function} $f$, with domain $X$ and
codomain $Y$, written $f:X_\bot\rightarrow Y_\bot$, is a total function $f :
\p{X\cup\set{\bot}} \rightarrow \p{Y\cup\set{\bot}}$, where $\bot\notin X$,
$\bot\notin Y$, and $f\p{\bot}=\bot$.\end{definition}

\begin{definition} We say that $\bot$, read ``bottom'', denotes an
\textbf{undefined} value, in the sense that it is not an element of any
distinguished set we will consider. \end{definition}

\begin{definition} For any partial function $f:S_\bot\rightarrow T_\bot$, if
$f\p{x}=\bot$, for some $x\in X$, we say that $f\p{x}$ is
\textbf{undefined}.\end{definition}

This definition of partial functions is perhaps a bit unconventional, but it
permits us to directly compose partial functions, and provides perhaps a better
model of computation (as discussed in
\refSection{background-what-tm-computes}). Indeed, $\bot$ will represent
non-terminating computation.

For the lack of a formal definition of either sets or functions, but in tune
with the specifications of these notions above, we can regard \emph{functions
as sets}. In such an interpretation, the codomain (and domain) of a function
may itself be a function. This demands some convenient notation:

\begin{notation} A function $f:X\rightarrow G$, where $G$ is a function
$g:Y\rightarrow Z$ is also written $f:X\rightarrow \p{Y\rightarrow Z}$, or
simply $f:X\rightarrow Y\rightarrow Z$. That is, $\rightarrow$ is
right-associative. \end{notation}

\def\codomain#1{(co)\textsuperscript{$#1$}domain}

%\begin{notation} For a function $f:X_1\rightarrow X_2 \rightarrow \cdots
%\rightarrow X_n \rightarrow Y$, for some $n\in \mathbb{N}$, we say $Y$ is the
%\codomain{n} of $f$, and $f$ has $n$ \textbf{arguments}, or $f$ is an
%$n$-argument function, and furthermore

%\begin{itemize}

%\item $f$ is \textbf{nullary} if it has $0$ arguments;

%\item $f$ is \textbf{unary} if it has $1$ argument;

%\item $f$ is \textbf{binary} if it has $2$ arguments.

%\end{itemize}

%\end{notation}

\begin{definition} A function $f:X\rightarrow Y$ is \textbf{injective} if
$f\p{x}=f\p{y}$ iff $x=y$. \end{definition}

\begin{definition} A function $f:X\rightarrow Y$ is \textbf{surjective} iff for
each $y\in Y$ there is a corresponding $x\in X$, such that $f\p{x}=y$.
\end{definition}

\begin{definition} \label{def:enumerable} A set $X$ is \textbf{enumerable} if
there exists an injective function $f:X\rightarrow \mathbb{N}$.
\end{definition}

\begin{definition} The \textbf{identity function} for a set $X$, written
$id:X\rightarrow X$, is the (unique) function such that $id\p{x}=x$.
\end{definition}

\begin{corollary} Every set $X$ has an identity function. \end{corollary}

\begin{corollary} An identity function is injective. \end{corollary}

\begin{corollary} The set $\mathbb{N}$ is enumerable. \end{corollary}

\begin{definition} A \textbf{partial function} $f$, with domain $X$ and
codomain $Y$, written $f:X\rightarrow Y_{\bot}$, is a total function
$f:X\rightarrow Y \cup \set{\bot}$, where $Y\cap \set{\bot} = \emptyset$.
\end{definition}

\begin{notation} When denoted by a symbol, the application of a relation will
sometimes be denoted in infix notation. \end{notation}

\begin{notation} By $=$, we denote the \textbf{equality relation}.
\end{notation}

 A partial function can hereby be given by a set of mappings,
where a mapping from $x$ to $f\p{x}$ is written $x\mapsto f\p{x}$.

\begin{definition} A function $f:X\rightarrow Y$ is a \textbf{bijection}, or
``is bijective'', iff there exists a function $g:Y\rightarrow X$ such that for
each $x\in X$, $g\p{f\p{x}}=x$. \end{definition}

\begin{definition} A set $X$ is \textbf{enumerable} if there exists a injective
function $f:X\rightarrow \mathbb{N}$. \end{definition}

\subsection{Cardinality}

\begin{notion} The \textbf{cardinality} of a set $X$, written $\card{X}$, is a
measure of its size. \end{notion}

\begin{definition} Two sets $X$ and $Y$ have the same cardinality, written
$\card{X}=\card{Y}$, if there exists a bijective function $f:X\rightarrow
Y$.\end{definition}

\begin{definition} \label{def:card-leq} A set $X$ has a cardinality less than
or equal to the cardinality of set $Y$, written $\card{X}\leq\card{Y}$, if
there exists an injective function $f:X\rightarrow Y$. \end{definition}

\begin{theorem}\label{thm:card-enumerable} A set $X$ is enumerable iff
$\card{X}\leq\card{\mathbb{N}}$.  \end{theorem}

\begin{proof} Follows directly from \refDef{enumerable} and \refDef{card-leq}.
\end{proof}



%\begin{definition}

%A \textbf{language} $\mathcal{L}=\chev{\mathcal{C},\mathcal{F}, \mathcal{R}}$
%is given by specifying a set of constant symbols $\mathcal{C}$, a set of
%function symbols $\mathcal{F}$, and a set of relation symbols $\mathcal{R}$.

%\end{definition}

% a sequence is something formed by the concatenation of elements.

%\begin{definition}

%A \textbf{string} is a finite sequence of elements drawn from an alphabet.

%\end{definition}

%\begin{definition}

%A \textbf{string language} is a set of \textbf{strings}.

%\end{definition}

%..

%\begin{definition}

%A \textbf{closure} of $A$ under the operations $O$, is the smallest class $C$,
%containing $A$, and such that the operations of $O$, operating on elements of
%$C$ yield elements of $C$.

%\end{definition}

%\begin{definition}

%A \textbf{function algebra} is a closure of a class $A$.

%\end{definition}

%\begin{definition}

%A characteristic function 

%\end{definition}

\subsection{Strings and Languages}

\begin{definition} An \textbf{alphabet} is a finite, nonempty set of
\textbf{symbols}.  \end{definition}

\begin{definition} A \textbf{string} over an alphabet is a finite sequence of
symbols from that alphabet. \end{definition}

%\begin{notation} We take $b\notin A$ to mean the same as $\set{b}\cup A =
%\emptyset$. \end{notation}

\section{Categories}

Classical and constructive mathematics are not the only possible foundations
for the theory of computability and complexity.

\emph{Category theory} allows a systematic discussion of the behaviour of a
system in terms of the admissible transformations of system objects, without
dealing in the issues of representation.

\begin{specification} A \textbf{category} $\mathbb{C}$ is a collection of
\textbf{objects} and \textbf{morphisms} between them. A morphism $f$ between
objects $A$ and $B$ is denoted $f:A\rightarrow B$. We say that $A$ is the
\textbf{domain}, and $B$ the \textbf{codomain} of $f$. For each object $A$,
there is an \textbf{identity morphism} $id_A:A\rightarrow A$.

There is a notion of equality amongst the objects of a category, such that if
$B=C$, for some morphisms $f : A\rightarrow B$ and $g : C \rightarrow D$, then
there exists a \textbf{composite morphism} $\p{g\circ f} : A \rightarrow D$.
There is also a notion of equality amongst the morphisms of a category, such
that:

\begin{enumerate}

\item [C-1] For any morphism $f:A\rightarrow B$, we have $f\circ id_A = f =
id_B \circ f$.

\item [C-2] For any triple of morphisms $f : A \rightarrow B$, $g : B
\rightarrow C$, and $h : C \rightarrow D$, we have $f\circ \p{g \circ h} =
\p{f\circ g} \circ h$.

\end{enumerate}

\end{specification}

\begin{notation} Except when in conflict with other definitions, we'll use the
alphabet $\mathbb{C},\mathbb{D},\mathbb{E},\ldots,\mathbb{Z}$ to denote
categories, $A,B,C,\ldots,Z$ to denote objects, and $f,g,h,\ldots,z$ to denote
morphisms.  \end{notation}

% order of characters is significant in an alphabet, often we define an order
% on the letters leading to a lexicographical order on strings.

Category theory is sufficient to deduce many, but not all results in the theory
of computability and complexity\cite{di-paola-heller-1987}. The main hurdle to
this is that a category is classically defined with morphisms being
\emph{total}, whereas the notion of computation asks for partiality. That is,
for morphisms $f : A \rightarrow B$ and $g : C \rightarrow D$ to be composable,
we must have $B=C$. If $g$ is partial, it is not defined over the entire domain
$B=C$, and so the ``compositionality'' of these morphisms is questionable.

We still give a basic account of Category theory, as it will never-the-less
prove useful in modeling terminating and otherwise naturally constrained
computation.
