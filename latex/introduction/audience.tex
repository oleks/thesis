\section{Audience} \label{sec:introduction:audience}

The audience of this thesis is anyone interested in the connection of
computability and complexity to the theory and practice of programming
languages.

In particular, what the admittance of useful programming language constructs
implies for the time and space complexity of the programs that you can write. A
programming language construct is ``useful'' if its admission permits to write
a practical class of programs in an elegant manner.

The thesis is directed towards the level of a Computer Science graduate student
at the time of writing: The reader is assumed to be familiar with the basics of
discrete mathematics, as in \cite[\ch~0]{sipser-2013}, and \cite[Appendices A,
B, and C]{cormen-et-al-2009}. The analysis of time and space complexity of
algorithms, as in \cite[\chs~1--17 and \chs~21--24]{cormen-et-al-2009}. Regular
and context-free languages, as in \cite[\chs~1--2]{sipser-2013}, and their use
for programming language design, as in \cite{mogensen-2010}. The reader should
also be familiar with Logic in Computer Science, as in
\cite[\chs~1--4]{huth-ryan-2004}.
