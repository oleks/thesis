\chapter{Computability}

% Aka. Computability Theory

We begin with an intuitive notion of computation, calling for a formal
definition of Turing machines and an exposition of the Church-Turing thesis.

\begin{notion} A problem is ``computable'' if it can be solved by transforming
a mathematical object, without ingenuity.
\end{notion}

Any attempt at a more definite notion of computability seems to arrive at a
philosophical impasse, where the notions of ``mathematical object'',
``transformation'', and ``ingenuity'' form a philosophical conundrum. The
indefinite notion however, is sufficient to make the following statement:

\begin{statement} If a problem $P$ can be solved by solving a computable
problem $Q$, followed by solving a computable problem $R$, then $P$ itself is
computable. \end{statement}

The statement holds vacuously\footnote{Although this too, can be a
philosophical conundrum.} as no transformations are performed other than to
solve the two computable problems. Consequently, any problem which can be
solved by solving a finite sequence of computable problems, is itself
computable. Thus we arrive at an intuitive notion of an algorithm:

\begin{notion} An ``algorithm'' is a specification of how a problem can be
solved by solving a finite sequence of computable problems, called
steps.\end{notion}

Such indefinite notions are useful for little else. We are left to take a
philosophical leap of faith, and fall prey to some choice of formalism. To this
end, the notion of a Turing machine, due to Alan Turing\cite{turing-1937},
forms perhaps the most well-founded, classical, and intuitive notion of
computation.

\begin{definition} \emph{Turing machine (TM).}

A Turing machine is a 7-tuple
$\chev{Q,\Sigma,\Gamma,\delta,q_0,q_{\text{accept}},q_{\text{reject}}}$, where $Q$, $\Sigma$,
and $\Gamma$ are all finite sets and

\begin{enumerate}

\item $Q$ is a set of states,

\item $\Sigma$ is an alphabet not containing the \textbf{blank symbol}
\textvisiblespace,

\item $\Gamma$ is the tape alphabet, $\text{\textvisiblespace}\in\Gamma$ and
$\Sigma\subseteq\Gamma$,

\item $\delta:Q\times\Gamma \rightarrow Q\times\Gamma\times\set{L,R}$, is the
transformation function,

\item $q_0$ is the starting state,

\item $q_{\text{accept}}$ is the accept state.

\item $q_{\text{accept}}$ is the reject state, and $q_{\text{accept}}\neq q_{\text{reject}}$.

\end{enumerate}

\end{definition}

\begin{notion} \textit{The Church-Turing thesis.}

No intuitive notion of an algorithm can compute a problem that a Turing
machine cannot compute.

\end{notion}

The Church-Turing thesis does not lend itself to mathematical proof. It has
however, a notational significance: to show that a problem P can be computed
by a Turing machine M, it is sufficient to informally describe an algorithm for
computing P.
