\section{Matters of Typography}

\subsection{Internal references} \label{sec:introduction:internal-references}

All references within this document to parts of this document are qualified
with the page on which the referenced part begins. For instance, we refer to
this section as \refSection{introduction:internal-references}, read ``section
zero point two, on page two''.

\subsection{External references} \label{sec:introduction:external-references}

All external references..

\subsection{Named segments}

Named segments are text segments with a role and an address. The role seeks to
distinguish the text segment from prose, and the address permits to
subsequently reference the segment. For instance, \refExample{alice-bob-add}.

Addresses are of the format \mbox{$x$.$y$.$z$}, where $z$ is the index of the
named segment in section indexed $y$, in chapter indexed $x$. Where
appropriate, we'll give a named segment a name for even easier reference, for
instance, when defining the concept of a Turing machine. The various roles are
described below, showcasing also how each named segment will be formatted.

\begin{notion} An informal definition of a concept, appealing either to the
reader's intuition or formerly stated notions. \end{notion}

Some concepts are too disputed to be given by a notion. In the presence of
multiple competing notions, we will merely specify the features that we are
interested in.

\begin{specification} An informal specification of a concept, appealing either
to the reader's intuition or formerly stated notions or specifications.
\end{specification}

\begin{hypothesis} An informal statement, proposed as an explanation of a
natural phenomenon. A hypothesis does not admit itself to formal proof, but can
be disproven by observation. A hypothesis is always followed by a discussion of
its validity. \end{hypothesis}

Some notions are ``primitive'', in the sense that they underpin subsequent
formal definitions. It is a fundamental epistemological impasse, that in any
formal system, some concepts must remain distinguished, but undefined, these
form the \textbf{primitive notions}, or \textbf{axioms}, of the formal system.

\begin{definition} A formal definition of a concept, appealing to either
formerly stated primitive notions or definitions. Definitions seek to underpin
subsequent formal definitions and analyses. \end{definition}

This thesis only seeks to be self-contained beyond what can be expected of the
intended audience. Some notions, hypotheses, and definitions appeal to
knowledge outside this thesis. See also \refSection{introduction:audience}.

\begin{theorem} \label{thm:theorem} A formal statement using formerly
defined concepts. A theorem is always proven to hold by an immediately
proceeding proof.\end{theorem}

\begin{proof} A formal or informal proof the preceding theorem. \end{proof}

As proofs occur immediately after a theorem, proofs are not decorated with an
address. A proof is then referred to by the address of the theorem it proves.
For instance, we refer to the proof above as \refProof{theorem}.

\begin{example} An example of a concept discussed above. Examples seek to
facilitate the reader's understanding on the concept in question.
\end{example}

\begin{notation}

A choice of notation for some formally or informally defined concepts.
Typically, notation is introduced to aid further discussion.

\end{notation}

A concept is put in quotations, for instance ``computable'', if it is
introduced informally as part of a notion. A concept is put in bold, for
instance \textbf{computable}, if it is introduced formally as part of a
definition.

\subsection{Sequences}

Sequences have notational significance in the remainder of this document.

\begin{notion}

A \textbf{sequence} is a juxtaposition of elements.

\end{notion}

Some short sequences can be denoted typographically. For instance,

$$\symb{1}\symb{0}\symb{1}\symb{0}\symb{1}\symb{0}$$

To make the distinction between the elements of a sequence clear, and to
distinguish various types of sequences, some sequences will appear ``separated
by'' a distinguished symbol. For instance,

$$\symb{1},\symb{0},\symb{1},\symb{0},\symb{1},\symb{0} \quad \text{or} \quad
X\rightarrow Y \rightarrow Z.$$

Some sequences will be too long to be sequenced typographically. Such sequences
will be given by stating their ``sequencing algorithm'', by which the elements can
be sequenced, one-by-one, at the reader's leisure. There need not be a point at
which we have sequenced all the elements, and so the reader should take care
not to get carried away.

We will soon formalise the notion of an ``algorithm''. For now, we make use of
the less formal, ellipses notation:

\begin{notation} A sequence is given by \textbf{ellipses notation} by
typographically sequencing the first couple of elements, and terminating the
sequence with a distinguished ellipses symbol ($\ldots$) We then rely on the
reader's intuition to deduce how the subsequent elements are to be
sequenced\footnote{We generally ignore Wittgenstein's rule-following paradox
\cite[\textsection~201, p.  69]{wittgenstein-1953}.}. For instance,

$$\symb{1},\symb{0},\symb{1},\symb{0},\symb{1},\symb{0},\ldots$$ 

If we wish for the sequencing algorithm to stop when a particular ``final
element'' is reached, we follow the ellipses by the final element. For instance,

$$A\rightarrow B \rightarrow \cdots \rightarrow Z.$$

\end{notation}

We say that the elements of a sequence appear in a particular ``order'', and
sometimes, the elements, or sequences of elements, ``repeat'' within a
sequence. For instance, the elements \symb{1} and \symb{0} repeat, in that
order, in a sequence above. Such order and repetition will sometimes be
significant.

Some sequences will be further decorated to guide the reader's intuition. For
instance, a ``Turing tape'', as discussed in
\refSection{background-computability}, is a sequence of tape squares, occupied
by symbols drawn from an alphabet, followed by an infinite sequence of blank
symbols. We illustrate a Turing tape as in \refFigure{informal-turing-tape}.
 
\begin{figure}[h!]
\centering
\begin{tikzpicture}
  \tikzset{tape/.style={minimum size=.7cm, draw}}
  \begin{scope}[start chain=0 going right, node distance=0mm]
   \foreach \x [count=\i] in {\symb{1}, \symb{0}, \symb{1}, \symb{0}, \symb{1}, \symb{0}, \blank, \blank, \blank} {
    \ifnum\i=9 % if last node reset outer sep to 0pt
      \node [on chain=0, tape, outer sep=0pt] (n\i) {\x};
      \draw (n\i.north east) -- ++(.1,0) decorate [decoration={zigzag, segment length=.12cm, amplitude=.02cm}] {-- ($(n\i.south east)+(+.1,0)$)} -- (n\i.south east) -- cycle;
     \else
      \node [on chain=0, tape] (n\i) {\x};
     \fi
   }
   \node [right=.25cm of n9] {$\cdots$};
  \end{scope}
\end{tikzpicture}
\caption[]{A Turing tape with an alphabet of \symb{0} and \symb{1}.
\blank{} denotes the blank symbol.}
\label{fig:informal-turing-tape}
\end{figure}
