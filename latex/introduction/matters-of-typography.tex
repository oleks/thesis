\section{Matters of Typography}

\subsection{Internal references} \label{sec:introduction:internal-references}

All references within this document to parts of this document are qualified
with the page on which the referenced part begins. For instance, we refer to
this section as \refSection{introduction:internal-references}, read ``section
zero, point three, point one, on page three''.

\subsection{External references} \label{sec:introduction:external-references}

All external literature is referenced enclosed brackets by citing the author(s)
and year of the (first) publication. The citations are further expanded in the
references at the end of this document. For instance, \cite{sipser-2013}.

If you are reading a PDF version of this document, the citation is a hyperlink
to the expanded citation. For easy backreference, the expanded citation lists
all the pages where the citation shorthand occurs. Feel free to try this now.

Sometimes we further qualify the citation with a particular section, or page
number for a more precise citation.  For instance,
\cite[\chs~1--2]{sipser-2013}.

\subsection{Named segments}

Named segments are text segments with a role and an address. The role seeks to
distinguish the text segment from prose, and the address permits to
subsequently reference the segment. For instance, \refExample{alice-bob-add}.

Addresses are of the format \mbox{$x$.$y$.$z$}, where $z$ is the index of the
named segment in section indexed $y$, in chapter indexed $x$. Where
appropriate, we'll give a named segment a name for even easier reference, for
instance, when defining the concept of a Turing machine. The various roles are
described below, showcasing also how each named segment will be formatted.

\begin{notion} An informal definition of a concept, appealing either to the
reader's intuition or other informal concepts. \end{notion}

\begin{hypothesis} An informal statement, proposed as an explanation of a
natural phenomenon. A hypothesis does not admit itself to formal proof, but can
be disproven by observation. A hypothesis is always followed by a discussion of
its validity. \end{hypothesis}

A concept is ``put in quotations'', if it is introduced informally, as part of
a notion, notation, or a mere discussion. A concept is \textbf{put in bold}, if
it is introduced formally, as part of a definition.

Some notions are ``primitive'', in the sense that they underpin subsequent
formal definitions. It is a fundamental epistemological impasse, that in any
formal system, some concepts must remain distinguished, but undefined, these
form the ``primitive notions'', or ``axioms'', of the formal system.

\begin{definition} A formal definition of a concept, appealing to either
formerly stated primitive notions or definitions. Definitions seek to underpin
subsequent formal definitions and analyses. \end{definition}

This thesis only seeks to be self-contained beyond what can be expected of the
intended audience. Some notions, hypotheses, and definitions appeal to
knowledge outside this thesis. See also \refSection{introduction:audience}.

\begin{theorem} \label{thm:theorem} A formal statement using formerly defined
concepts. A theorem is always proven to hold by an immediately succeeding
proof.\end{theorem}

\begin{proof} A formal or informal proof the preceding theorem. \end{proof}

As proofs occur immediately after a theorem, proofs are not decorated with an
address. A proof is then referred to by the address of the theorem it proves.
For instance, we refer to the proof above as \refProof{theorem}.

Some concepts may have several competing formal or informal definitions in
literature.  When such a choice of fundamentals is of little importance, we
will give a concept by a ``specification'' rather than an informal or formal
definition. 

\begin{specification} An informal specification of a concept, appealing either
to the reader's intuition or other informal concepts. The reader is free to
choose any known formal or informal definition of the concept, provided that it
adheres to the given specification. \end{specification}

\begin{example} An example of a concept discussed above. Examples seek to
facilitate the reader's understanding of the concept in question.
\end{example}

\begin{notation}

A choice of notation for some formally or informally defined concepts.
Typically, notation is introduced to aid further discussion.

\end{notation}

\subsection{Sequences}

Sequences have notational significance in the remainder of this document.

\begin{notion}

A (typographical) \textbf{sequence} is a juxtaposition of elements.

\end{notion}

Some short sequences can be denoted typographically. For instance,

$$\symb{1}\symb{0}\symb{1}\symb{0}\symb{1}\symb{0}$$

To make the distinction between the elements of a sequence clear, and to
distinguish various types of sequences, some sequences will appear ``separated
by'' a distinguished symbol. For instance,

$$\symb{1},\symb{0},\symb{1},\symb{0},\symb{1},\symb{0} \quad \text{or} \quad
X\rightarrow Y \rightarrow Z.$$

Some sequences will be too long to be sequenced typographically. Such sequences
will be given by stating their ``sequencing algorithm'', by which the elements
can be sequenced, one-by-one, at the reader's leisure. There need not be a
point at which we have sequenced all the elements, and so the reader should
take care not to get carried away.

We will soon formalise the notion of an ``algorithm''. For now, we make use of
the less formal, ellipses notation:

\begin{notation} A sequence is given by \textbf{ellipses notation} by
typographically sequencing the first couple of elements, and terminating the
sequence with a distinguished ellipses symbol ($\ldots$) We then rely on the
reader's intuition to deduce how the subsequent elements are to be
sequenced\footnote{We generally ignore Wittgenstein's rule-following paradox
\cite[\textsection~201, p.  69]{wittgenstein-1953}.}. For instance,

$$\symb{1},\symb{0},\symb{1},\symb{0},\symb{1},\symb{0},\ldots$$

If we wish for the sequencing algorithm to stop when a particular ``final
element'' is reached, we follow the ellipses by the final element. For instance,

$$A\rightarrow B \rightarrow \cdots \rightarrow Z.$$

\end{notation}

We say that sequences having a final element are ``finite'', and ``infinite''
otherwise. For instance, the last sequence above is finite, whereas the one
before it is infinite.  Furthermore, the elements of a sequence appear in a
particular ``order'', and sometimes, the elements, or sequences of elements,
``repeat'' within a sequence. For instance, the elements \symb{1} and \symb{0}
repeat, in that order, in a sequence above. Such order and repetition will
sometimes be significant.

Some sequences will be further decorated to guide the reader's intuition. For
instance, a ``Turing tape'', as discussed in
\refSection{background-computability}, is a finite sequence of tape squares,
occupied by symbols drawn from an alphabet, followed by an infinite sequence of
tape squares, each occupied by special ``blank symbol''. We illustrate a Turing
tape as in \refFigure{informal-turing-tape}.
 
\begin{figure}[h!]
\centering
\begin{tikzpicture}
  \tikzset{tape/.style={minimum size=.7cm, draw}}
  \begin{scope}[start chain=0 going right, node distance=0mm]
   \foreach \x [count=\i] in {\symb{1}, \symb{0}, \symb{1}, \symb{0}, \symb{1}, \symb{0}, \blank, \blank, \blank} {
    \ifnum\i=9 % if last node reset outer sep to 0pt
      \node [on chain=0, tape, outer sep=0pt] (n\i) {\x};
      \draw (n\i.north east) -- ++(.1,0) decorate [decoration={zigzag, segment length=.12cm, amplitude=.02cm}] {-- ($(n\i.south east)+(+.1,0)$)} -- (n\i.south east) -- cycle;
     \else
      \node [on chain=0, tape] (n\i) {\x};
     \fi
   }
   \node [right=.25cm of n9] {$\cdots$};
  \end{scope}
\end{tikzpicture}
\caption[]{A Turing tape with an alphabet of \symb{0} and \symb{1}.
\blank{} denotes the blank symbol.}
\label{fig:informal-turing-tape}
\end{figure}

\subsection{Definitions}

\begin{notation} By $X\triangleq Y$, read ``$X$ by definition the same as
$Y$'', we mean that $X$ is henceforth interchangeable with $Y$, until $X$
occurs on the left side of another $\triangleq$, that is, until $X$ is
``redefined''. \end{notation}

This is not all the matters of typography, but is sufficient to lift some
mathematical definitions off the ground. The remaining matters of typography
are interspersed throughout the remainder of this document.
