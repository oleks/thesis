\chapter{Future Work}

\section{Implementation}

What has not been discussed is the possibility of an implementation.

See pola.

\section{Coinductive data types}

The litterature and techniques considered here took only into account types
that correspond (at most) to any initial algebra, that is, inductively defined
data types. This includes all finite types (e.g. booleans), and recursively
defined types (e.g. lists and trees). The dual, final coalgebras, that is,
coinductively defined data types, were not considered.

This is an active area of research, where e.g. ramified recursion
\cite{ramayaa-leivant-2011} plays a role.

\section{Category theory}

\cite{rutten-1998}

\section{Generalized Reedy categories}

% To this end we take a leap of faith into the realm of generalized abstract nonsense:

One popular category-theoretic view of programming is to identify the realm of
programming with some fixed category $\mathcal{C}$. In the context of
``successor'' and ``predecessor'' operations, it is perhaps worthwhile to
identify the realm of programming with a slightly more restrictive category:

\begin{definition} A \textbf{generalized Reedy structure}
\cite{berger-moerdijk-2011} on a \emph{small} category $\mathcal{R}$ consists
of two \emph{wide} subcategories $\mathcal{R}^+$ and
$\mathcal{R}^-$,\footnote{That is, both $Obj\p{\mathcal{R}}$ and
$Mor\p{\mathcal{R}}$ form a set, and $Obj\p{\mathcal{R}} = Obj\p{\mathcal{R}^+}
= Obj\p{\mathcal{R}^-}$.} and a degree function $d : Obj\p{\mathcal{R}}
\rightarrow \mathbb{N}$, satisfying the follow axioms:

\begin{enumerate}

\item non-invertible morphisms in $\mathcal{R}^+$ increase degree;

\item non-invertible morphisms in $\mathcal{R}^-$ decrease degree;

\item isomorphisms in $\mathcal{R}$ preserve degree;

\item $Iso\p{\mathcal{R}} = Mor\p{R^+} \cap Mor\p{R^-}$;

\item every $f \in Mor\p{\mathcal{R}}$ factors as $f = h \circ g$ for some $h
\in Mor\p{\mathcal{R}^-}$ and $g \in Mor\p{\mathcal{R}^+}$, uniquely up to
isomorphism;

\item if $h \circ i = h$ for some $i \in Iso\p{\mathcal{R}}$ and $h \in
Mor\p{\mathcal{R}^-}$, then $i$ is an identity.

\end{enumerate}

The structure is \textbf{dualizable} if the following axiom also holds:

\begin{enumerate}
\setcounter{enumi}{6}

\item if $i \circ g = g$ for some $i \in Iso\p{\mathcal{R}}$ and $g \in
Mor\p{\mathcal{R}^+}$, then $i$ is an identity.

\end{enumerate}

A \textbf{(dualizable) generalized Reedy category} is a small category equipped
with a (dualizable) generalized Reedy structure. 

\end{definition}

Generalized Reedy categories are interesting as they can form various ``shape
categories'', such as cycle and tree categories.

\begin{definition} A \textbf{successor} is a morphism which increases degree.
\end{definition}

\begin{definition} A \textbf{predecessor} is a morphism which decreases degree.
\end{definition}

\section{Natural numbers}

We begin by defining a theory of natural numbers, $\mathcal{N}$. This is
distinct from the set of natural numbers, $\mathbb{N}$, which may form the
universe of a model of $\mathcal{N}$.

\begin{definition} \textit{The theory of natural numbers, $\mathcal{N}$.}

Let the language
$\mathcal{L}_{\mathcal{N}}=\chev{\set{0},\set{s(x)},\set{x<y,x=y}}$, consist of
a constant symbol $0$ (read: zero), a unary function $s$ (read: successor), and
a binary relation $<$ (read: less than). The class of natural numbers is
axiomatized by the $\mathcal{L}_{\mathcal{N}}$-sentences

\begin{align}
&\vdash 0 < s(x) \\
x < y &\vdash s(x) < s(y)
\end{align}

\end{definition}

\begin{example} \textit{A model $\mathcal{M}$ of $\mathcal{N}$.}

Let the universe be the set of natural numbers, $\mathbb{N}$. Let the constant
zero be the numeral $0$, and the successor function be the mathematical
function $s(x)=x+1$. Let the less than relation be the usual $<$ relation on
$\mathbb{N}$.

Notably, if $\mathcal{M}$ is a model of $\mathcal{N}$, satisfying the above
axioms, $s$ is an \emph{injective} function, and $<$ is an \emph{irreflexive}
and \emph{transitive} relation.

\end{example}

\begin{remark}

The theory of natural numbers as defined above, closely resembles the notion of
natural numbers as originally given by Dedekind\cite{beman-1901,joyce-2005}.
Unlike the more conventional Peano/Dedekind axioms, which you might be familiar
with, this notion focuses on the ordered nature of the natural numbers.

\end{remark}

\begin{definition} \emph{The theory of strings over an alphabet $\Sigma$}

Let the language
$\mathcal{L}_\Sigma=\chev{\set{\varepsilon}\cup\Sigma,\set{x\cdot y},\set{}}$,
consist of the constant symbols $\Sigma$, the empty symbol $\varepsilon \notin
\Sigma$, and a binary function $\cdot$ (read: concatenate). The class of
strings over an alphabet $\Sigma$ is axiomatized by the
$\mathcal{L}_\Sigma$-sentences

\begin{align}
&\vdash \varepsilon \cdot x = x & \text{(left identity)} \\
&\vdash x \cdot \varepsilon = x & \text{(right identity)} \\
&\vdash \p{ x \cdot y } \cdot z = x \cdot \p{ y \cdot z } & \text{(associativity)}
\end{align}

\end{definition}

