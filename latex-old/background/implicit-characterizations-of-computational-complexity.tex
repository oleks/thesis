% vim: set spell:

\chapter{Implicit Characterizations of P}

\section{Primitive Recursion on Notation}

When dealing with the manipulation of symbolic strings, there is a natural
total order on the values of the formal parameters --- the length of their
representing string. Primitive recursion on notation utilizes this order,
requiring that the length of the input string be decreased before a recursive
call.

\begin{definition} A function $f$ is defined by primitive recursion on notation
from functions $g$, $h_1$, $h_2$, \ldots, $h_{\card{\Sigma}}$ iff 

\begin{align}
f\p{\varepsilon, \vect{t}}  &= g\p{\vect{t}} \\
f\p{s_i \cdot s,\vect{t}} &= h_i\p{s,\vect{t},f\p{s,\vect{t}}} & \forall s_i: \Sigma
\end{align}

\end{definition}

We say that a function is primitive recursive on notation (PRN), if it is defined by
primitive recursion on notation from non-recursive or PRN functions.

Unfortunately, not all PRN functions take polynomial time.

\begin{theorem} There exist PRN functions which do a superpolynomial amount of
work.\end{theorem} 

\begin{proof}
Consider a function $g$, which duplicates every symbol in the input string:

\begin{align}
g\p{\varepsilon} &= \varepsilon \\
g\p{\mathtt{0} \cdot s} &= \mathtt{0} \cdot \mathtt{0} \cdot g\p{s}
\end{align}

Consider furthermore a function $h$, which calls $g$ iteratively, the same
number of times as the length of its input string:

\begin{align}
h\p{\varepsilon} &= \mathtt{0} \\
h\p{\mathtt{0} \cdot s} &= g\p{h\p{s}}
\end{align}

Calling $g$ with a string of length $n$, we obtain a string of length $2^n$ due
to iterated duplication. It follows that $g$ does a superpolynomial amount of
work. \end{proof}

\begin{example}
We illustrate the above proof with an example:
\begin{align}
h\p{\mathtt{0} \cdot \mathtt{0}}
  &\leadsto g\p{h\p{\mathtt{0}}} \\
  &\leadsto g\p{g\p{h\p{\varepsilon}}} \\
  &\leadsto g\p{g\p{\mathtt{0}}} \\
  &\leadsto g\p{\mathtt{0} \cdot \mathtt{0} \cdot g\p{\varepsilon}} \\
  &\leadsto g\p{\mathtt{0} \cdot \mathtt{0}} \\
  &\leadsto \mathtt{0} \cdot \mathtt{0} \cdot g\p{\mathtt{0}} \\
  &\leadsto \mathtt{0} \cdot \mathtt{0} \cdot \mathtt{0} \cdot \mathtt{0} \cdot g\p{\varepsilon} \\
  &\leadsto \mathtt{0} \cdot \mathtt{0} \cdot \mathtt{0} \cdot \mathtt{0}
\end{align}
\end{example}

\section{Bounded Primitive Recursion on Notation}

With reference to the ``extended rudimentary functions'' of
\cite{bennett-1962}, \cite{cobham-1965} defined bounded primitive recursion on
integers in decimal notation. For a similar definition on binary notation, see
\cite[p.  127]{rose-1984}.

We generalize this to an arbitrary alphabet $\Sigma$, building upon PRN above:

\begin{definition}

A function $f$ is defined by bounded primitive recursion on notation from
functions $g$, $h_1,h_2,\ldots,h_{\card{\Sigma}}$, and $k$ iff

\begin{align}
f\p{\varepsilon, \vect{y}}  &= g\p{\vect{y}} \\
f\p{s_i \cdot x,\vect{y}} &= h_i\p{x,\vect{y},f\p{x,\vect{y}}} & \forall s_i : \Sigma \\
f\p{x,\vect{y}} &\leq k\p{x,\vect{y}}\label{def-brn-poly-bound}
\end{align}

\end{definition}

We say that a function is bounded primitive recursive on notation (BPRN) if it
is defined by bounded primitive recursion on notation from non-recursive, or
BPRN functions. The scheme is also known as limited primitive recursion on
notation.

The addition that we make to PRN is that the function $f$ must be bounded from
above by function $i$. Letting $i$ characterize polynomial time functions, we
obtain perhaps the earliest implicit characterization of P due to
\cite{cobham-1965}.

This straight-forward approach has an obvious limitation: it requires defining
an ordering relation on our functions --- a problem that is undecidable in
general. Furthermore 

\section{Finite Model Theory}

Finite model theory is 

\section{Ramification}

\subsection{Safe Recursion}

\subsection{Tiering}
